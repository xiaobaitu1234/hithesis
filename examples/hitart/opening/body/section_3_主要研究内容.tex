% !Mode:: "TeX:UTF-8"

\section{主要研究内容}
本文将围绕CDN服务质量,尤其是一种面向时延特征的CDN服务质量测绘
及部署优化策略研究展开。研究内容组织结构如图 \ref{fig:研究内容组织结构图}所示,展示了主要研究内容及各内容间关系。首先,设计实现CDN服务商识别系统,基于探测空间约减方案来降低信息源获取对实网环境的污染,使用CDN特征集扩充方案发现并存储潜在CNAME特征串,为后续的时延测量提供服务商特征;其次,设计实现时延测量系统,制定基本探测流程,如单点探测顺序、探测内容等。并构建分布式探测系统,用于保障探测数据的完整性、可靠性。根据探测得到的CDN边缘服务器时延结果进行分析,得到CDN的部署效果,为后续放置优化策略提供优化方向;最后,设计一种权衡CDN部署成本和QoS的部署方案,为CDN服务商、ICP和终端用户三方的权益提供一种均衡策略。

\begin{figure}[ht]
  \centering
  \includegraphics{主要研究内容.pdf}
 
  \caption{研究内容组织结构图}
  \label{fig:研究内容组织结构图}
\end{figure}



\subsection{基于探测空间约减的CDN服务商识别研究}

为实现对CDN服务商进行高效、准确的发现与识别,并降低探测流量对实网环境的影响,本文设计基于探测空间约减的CDN服务商识别系统,向使用CDN可能性较高的域名进行探测,并利用CDN服务的就近解析原理,即DNS解析时,综合考量请求方ISP、网络位置、拥塞情况等因素,返回较优边缘服务器IP地址子集,使用地理分布广泛的递归DNS服务器完成探测工作。识别方案上,根据奥卡姆剃刀定律(Occam's Razor),本文试图剔除对识别精度影响小、或识别效率低的特征因子,以较小代价达到较优识别精度。具体的,本文选取域名的HTTP特征来辅助CNAME特征进行识别,根据识别发现的潜在使用CDN服务域名,提取其CNAME特征,结合人工判定来扩充CDN识别特征库,从而迭代提高CDN服务商识别精度。


\subsection{基于分布式代理的CDN时延测量研究}

为发现CDN服务在部署方面存在的不足,需要对CDN服务进行外部测量,面向时延特征分析其部署特点。本文设计基于分布式代理的CDN时延测量系统,根据CDN就近解析原则,通过地域分布广泛的代理(如Socket代理、HTTP代理等),获取CDN边缘服务器在不同地域,不同时间的时延特征。通过构建分布式平台解决探针与代理距离过远、瞬时网络拥塞或DNS服务器缓存等原因,导致的数据噪声问题,控制测量的准确性、完整性。根据测量结果进行多维度分析,得出CDN边缘服务器放置策略的评价值。

% 当用户选择OpenDNS和Google Public DNS等公共DNS解析器。其解析结果不可能接近所有用户\cite{fu2018},因此,本研究拟尝试FDNS的转发功能或代理服务器的方式模拟多地解析请求过程。

% 由于DNS拥有缓存机制\cite{Moura2019},在TTL时间内,即时首选边缘服务器存在更新,也无法及时传播到终端用户。因此,还需要通过间隔多次连续解析消除等情况引起的边缘服务器更新,导致服务器收集不全面的问题。

% 针对CDN边缘服务器时延测量问题,本文拟采用Socket代理的方式进行测量。胡治国\cite{胡治国2017IP}等人指出,当前网络测量的精准度有待提高,通用操作性系统无法测量高速网络的时延情况。通常,使用代理的测量方式可以保证覆盖面请求的覆盖面问题,理想情况下,使用单点结合分布广泛的代理即可完成探测任务。然而,由于单点不可能临近所有代理,与探测点地理位置过远,或网络距离较远的代理,其探测结果势必存在较大误差。因此,本研究拟采取构建分布式系统结合代理探测方式进行时延测量,并对测量结果进行分析。

\subsection{基于多目标优化博弈的CDN放置策略研究}
CDN放置优化可视为以降低CDN成本,保证用户QoS为目的的多目标优化问题。当前多数研究从服务器放置位置,或者是缓存策略方面独立进行研究。服务器放置问题通常是NP-Hard问题,多采用设施位置、K-median、最小K中心、K-Cache等模型,进行整数线性规划,并结合贪心算法等启发式或近似算法降低计算量。缓存策略可以由内容分发路径进行考量,选取路径长度较短、拥塞成本低的分发路径。本研究拟从服务器放置和缓存内容分发路径两方面进行综合考量,基于多目标优化技术结合博弈论权衡边缘服务器放置成本和QoS。

该算法使用两阶段方案对CDN放置策略进行研究。首先,对边缘服务器放置以及缓存内容分发路径两方面分别进行建模。构建相应的优化策略,随后,使用博弈思想对策略进行权衡,从而得出一种均衡方案。通过仿真软件对优化前后放置方案评价,对算法优化CDN放置效果进行评估。

 
% Local Variables:
% TeX-master: "../report"
% TeX-engine: xetex
% End: