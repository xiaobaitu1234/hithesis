% !Mode:: "TeX:UTF-8"

\chapter{基于多阶段学习的CDN服务域名识别方法}[jiyuTSVMdeCDNfuwuyumingshibiefangfa]




\section{引言}[yinyan]


\section{相关工作}[xiangguangongzuo]



\section{基于半监督学习的数据集构建方法}[pulearn]

\subsection{域名数据集选择方法}[]
当前较为出色的域名排行有Traco\cite{Pochat2018}排名以及SecRank\cite{Xie2022}排名。其中,Traco融合了多种国际排名,如Alexa、Umbrella、Majestic以及Farsight,具有较好的研究价值。但由于其提供域名多为SLD域名,不适合用来做CNAME检测及HTTP获取。因此,本文选择具有包含多种标签长度域名的SecRank排名作为研究对象。

这里我们考虑使用CNAME的二级域名(SLD)的WHOIS信息进行识别,判定是否为公司CDN入口域名。

于是方法就变成了CNAME特征串判定+CNAME的SLD的WHOIS判定补充+HTTP响应头判定。
\subsection{特征选取}[]

先前研究表明,中国大陆主要以以DNS全局负载均衡系统作为CDN IP选择依据,在这里,我们选取SecRank数据集的www域名以及SLD域名作为研究对象,因为他们大概率是网站,共计筛选出20W个。其次,我们选择来自不同地区的***个LDNS对域名集合进行DNS解析,并对包含解析结果的域名进行HTTP Head命令解析,获取其响应头信息字典。根据上述条件进行CDN代理标注。标注规则如下

\begin{description}
	\item[CDN代理域名] 域名CNAME记录包含CDN服务商关键字
	\item[非CDN代理域名] 域名多地解析结果相同,不含CNAME记录,且HTTP响应过程中不存在类似302、307等跳转状态码。
\end{description}

根据上述条件对样本筛选后,我们保守得判定出***个CDN域名,以及***个非CDN域名。

根据当前数据集,我们对HTTP响应头进行特征提取。

提取后排序如下:

因此,我们选取前10个特征作为特征提取依据。

\subsection{基于PU learn的小样本标注方法}[]

在对特征提取过程中,我们选取过程较为保守,即仅根据一些热点厂商的CNAME特征进行CDN代理域名判定,对使用一些CNAME特征白名单不包含厂商特征的域名,没有完整加入到正例集,且含CNAME记录的域名也存在部分未使用CDN服务。因此,这里采用PU learn方法对数据集中未标注的数据进行判定。



\subsection{基于TSVM的数据集扩充方法}[]
由于本文研究内容面向使用DNS作为全局负载均衡的CDN,其数据可从含CNAME记录的域名中产生。
NB-tree方法\cite{Hou2021}指出,使用CNAME、WHOIS以及HTTP响应头可以较好识别CDN服务商。然而,现网环境下,域名与IP的映射关系存在一定误差\cite{Ma2021}。其方法过于依赖IP的相关信息,IP变量的引入势必会对识别精度产生影响。

本文在NB-tree方法\cite{Hou2021}基础上进行改进,舍弃IP的相关信息。选取识别效果较好的CNAME及其二级域名(SLD)的WHOIS信息,结合HTTP响应头的特征分类进行识别器构造。首先,当前CDN识别数据集在构建时,大多采用监督学习,对标注后的数据进行训练。在实际场景中,根据符合厂商特征的CNAME特征串对HTTP响应头标注为正例准确率较高,但对未明确识别的数据进行类别划分较为困难。因此,本文使用PU learning\cite{Bekker2020}方法(一种从正例和未标注数据集学习的二元分类方法)进行小样本数据集标注工作。得到一批存在正例和负例的样本。随后,使用TSVM\cite{Joachims1999}(一种经典半监督 SVM)方法来构造大样本集,增加分类器的准确率。


\section{基于半监督集成学习的CDN节点识别方法}[pulearn]




算法在使用之前,要测一下数据集精度!!!!!!!,数据集可用之后,还要拿别人的算法对比。


使用决策树结合机器学习方法评估 CDN 对站点返回 HTTP报文的 headers 字典特征。此处可以结合多种方法对比,如SVM,随机森林,朴素贝叶斯等。(其中朴素贝叶斯用于跟侯锐杰的方法对比)

\section{实验结果}[shiyanujieguo]



\subsection{数据集构建精度结果}[]
在这里验证数据集构建方法的精度,如果达到什么条件???可以采纳???



\subsection{CNAME关键字发现方法结果}[]
拿出来看看找到多少家的特征串,说明一下,找找官方公布的数据对比一下,说一下我发现了百分之多少的厂商串



\subsection{CDN域名识别结果}[]
这里给出一个F1分数什么的作为结果验证。

效果评估部分: 在这里,可以先考虑不使用 CNAME,作为极端情况的特征串发现方法。

应用部分:给出一个使用 SecRank 数据集\cite{Xie2022}情况下,解析出CDN 服务的情况,在这里给出《互联网基础设施与软件安全》类似的 CDN 报告图片。(给出厂商数量,厂商使用量,排名等)



\section{本章小结}[benzhangxiaojie]




本章旨在提取CDN服务域名特征,给出一种CDN入口地址特征串发现方法。为后续测量分析部分提供厂商分类支撑。



% Local Variables:
% TeX-master: "../main"
% TeX-engine: xetex
% End:
