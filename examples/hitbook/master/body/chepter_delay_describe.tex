% !Mode:: "TeX:UTF-8"

\chapter{基于***的D???NS解析效率测量}[Example]


\section{引言}[yinyan]
\section{相关工作}[xiangguangongzuo]

\section{方法}[fangfa]

\section{实验结果}[shiyanujieguo]
\section{本章小结}[benzhangxiaojie]


1.公共DNS选取策略(很多论文都是用LDNS,用运营商提供的递归来判定,由于大约有13\%的用户使用公共DNS\cite{Callejo-2019-Measuring},因此引入部分公共DNS进行测量) 
选取TOP域名测一遍,观察准确率,成功率等(出一个算法)
(研究表明,在开启EDNS解析时,会附带终端用户的部分IP信息。此使,DNS全局负载均衡可以根据这些信息来返回对当前用户更优的选择。由ECS推出的Akamai映射方案为公共DNS服务的用户提供了显著的性能优势\cite{Chen2015}。相反,当终端用户未启用EDNS时,DNS负载均衡时会根据LDNS的地理位置进行智能解析\cite{Hao2018})
同时,根据用户举例LDNS的距离不同,最终结果的效果也不尽人意\cite{pan-2003-dns},目前也有研究提供一种用户选择最近开放DNS服务\cite{Zhang-2021-Scale-platform}方法。因此,在本次测量过程中,以LDNS作为聚类中心分配用户请求,理想化假定用户请求与自己距离最近的LDNS。同时为减少探测点网络性能的差距,我们使用请求时间减去探测点到DNS解析器的时间,减少探测点网络情况对实验结果的影响(考虑是否引用专利、论文~!!!!)。
 
解释方面:CDN入口CNAME域名的TTL比较长,在使用时可以忽略这部分时延。可以找一些CDN域名,向权威CNAME,以及向CDN解析CNAME请求对比一下TTL。
这个需要自己搭建递归做一下实验吧!!! 研究\cite{Moura2019}建议对DNS负载均衡使用较短的TTL而其他的建议长一些,且根据Li\cite{Li2020}的基于机器学习识别CDN方案,IP和TTL相关特征对域名CDN识别较为有效。   

2.基于约减策略,观察CDN厂商是否根据使用频率来修改TTL值?


<为什么要测量DNS呢,因为看CDN厂商对LDNS的支持程度,一定程度上对用户有一定好处,
我们也可以绘制一个延迟地图,给用户作为选择>
我们选择secrank的合适域名,对全国常用公共递归进行判定,对每一个递归拿到一个平均值,以及解析成功率
根据这些对每一个递归进行评分,最终显示到地图上。同时给出一个总分,供用户选择CDN时有一定区域偏向,或选择多款CDN来弥补偏向


\chapter{基于***的IP通信时延测量方法}[IPtongxinshiyanceliang]
\section{引言}[yinyan]
\section{相关工作}[xiangguangongzuo]
\section{实验结果}[shiyanujieguo]
\section{本章小结}[benzhangxiaojie]

是否探测节点数量????????

1.使用全球ping等工具抓数据
2.对数据进行清晰(聚类分析等)
3.对数据进行分析(想一个算法)




\chapter{基于***的数据分析方法}[shujufenxi]

或者是分时段、分地区测量(上下文,用于服务第三章、第四章)

1.想一个方法来构造评估策略。

层次分析?有客观部分,加一个算法


2.研究一下CDN测量结果评价标准,有没有相应的国标等。重点关注测量方法,测量效率,评估方面。







% Local Variables:
% TeX-master: "../main"
% TeX-engine: xetex
% End:
