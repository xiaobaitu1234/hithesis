% !Mode:: "TeX:UTF-8"

\section{主要研究内容}
本节对CDN服务商识别技术、CDN节点分布及时延测量、CDN节点分布优化策略相关研究内容进行介绍。


\subsection{基于探测空间约减的CDN服务商识别技术研究}

为完成CDN服务商的识别过程,首先要确定其识别算法的输入与识别算法执行流程。根据奥卡姆剃刀定律(Occam's Razor),本文试图剔除对识别精度影响较小、或识别效率较低的特征因子,从而以最小代价达到较优的识别精度。对当前主流识别算法分析可知,CNAME记录的匹配结果\cite{Huang2008,Adhikari2014,Guo2018},是判定域名是否使用CDN的主要手段之一。同时,在匹配规则基础上引入其他辅助特征,对识别算法的精度提高有一定帮助。Hou\cite{Hou2021}指出,使用CNAME、IP WHOIS以及HTTP响应头可以较好识别CDN服务商。然而,现网环境下,域名与IP的映射关系存在一定误差\cite{Ma2021}。引入IP的相关变量信息,势必会对识别效率和精度产生影响。因此,本文重点考虑HTTP特征来辅助识别。通常,针对使用CDN服务的站点,其经CDN边缘服务器交付给用户的HTTP响应,会在头部信息字典包含一些特征,如Cache、Hit from、Miss from、CDN等标识字段。因此,本文拟采用HTTP的响应头部字典,对CNAME记录匹配识别进行一定扩充,以期使用较少的信息源达到较优的识别精度。


同时,针对特征的探测空间约减,需要对探测对象和探测方法两方面进行处理。针对探测对象,本文主要考虑重点域名列表,或流量较大的域名,因为这些域名有较高概率使用CDN服务。在探测方法上,目前存在两种方式,一种为借助地域分布广泛的探针对域名进行解析,以获取多地域的解析结果
。该方案对探针规模及位置合理性有较高的要求。另一种方式为借助公开区域递归服务器(LDNS open)作为探测媒介\cite{Huang2008},从而获取多地解析结果。该方法借助现有DNS系统,但LDNS open规模有限,无法保证覆盖率。因此,本文拟采用转发DNS服务器(FDNS)作为媒介,从而覆盖更多的区域递归服务器(LDNS),期望使用较少资源和发包量,获取较全面的特征数据,降低对实网环境的污染。

综上所述,本文拟提出一种基于探测空间约减的CDN服务商识别技术,期望使用较少探测资源,以及实网环境低污染情况下,完成CDN服务商的识别。



\subsection{CDN节点分布及时延测量研究}








\subsection{CDN节点分布优化策略研究}
 

% Local Variables:
% TeX-master: "../report"
% TeX-engine: xetex
% End: