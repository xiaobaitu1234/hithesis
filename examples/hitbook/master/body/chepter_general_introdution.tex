% !Mode:: "TeX:UTF-8"

\chapter[绪论]{绪论}[xulun]
本章首先给出支撑研究的课题来源;其次介绍课题背景及研究的目的和意义,并分析研究当前国内外研究现状,指出目前存在的主要问题,最后介绍了本文的研究内容及组织结构。

\section{课题来源}[ketilaiyuan]

本课题来源于国家重点研发计划“XXX?????研究”(??????)及XXX项目。

\section{课题背景及研究的目的和意义}[ketibeijingjiyanjiudemudiheyiyi]



%%%%%%%%%%%%%%%%%%%%%%%%%%%%%%%%%%%%%
%%% 
%%% 
%%% 
%%% 引入CDN概念
%%% 
%%% 
%%% 
%%%%%%%%%%%%%%%%%%%%%%%%%%%%%%%%%%%%%


CDN,全称内容分发网络(Content Distribute Network),也有译作内容交付网络(Content Delivery Network)。顾名思义,CDN完成将内容从源站传递到终端用户的任务。

1991年之后的近十年间,公众互联网接入方式以拨号为主,接入带宽低且网民规模较小,互联网的瓶颈在于用户接入带宽,对提供内容的服务器和骨干传输网络的压力较小。随着互联技术的发展和网民数量增加,内容服务器和骨干网络压力随之增大,互联网瓶颈向内容服务器和骨干网络部分转移。

在CDN形成商用能力前,网络运营者们通过包括但不限于扩展技术Scale up/Scale out,镜像技术Mirror,缓存技术Cache来达到网站加速的目的。这些极大增加了网站运营成本。

1995年,麻省理工学院教授Tim Berners-Lee发起的一项技术挑战,旨在提出一种实现互联网内容的无拥塞分发方法,该学术难题最终催生出一种革新性的互联网服务,CDN。

Akamai公司通过智能化的互联网分发,结束了“World Wide Wait”(世界一起等待)的尴尬局面,并于1999年开始提供商业服务。随后三年间,全球互联网发展迎来了高潮期,2001年发布了RFC3040\cite{RFC-3040}描述了许多组件技术用于构建 CDN,完成了CDN的初始构建阶段。2012年发布了RFC6707\cite{RFC-6707}用于描述CDN互联问题。自此,CDN的发展进入了高速阶段。 


一个典型的CDN执行流程如图\ref{fig:join_cdn_classic}所示(根据CDN技术详解\cite{leibaohua-CDN-2012}及CDN技术调研\cite{Pathan-survey-2007})


\begin{figure}[h]
	\centering
	\includegraphics{cdn_abstract.pdf}
	\caption{引入CDN后的典型用户访问流程}
	\label{fig:join_cdn_classic}
\end{figure}




%%%%%%%%%%%%%%%%%%%%%%%%%%%%%%%%%%%%%
%%% 
%%%  
%%% 
%%% 这里过渡一下
%%% 
%%% 
%%% 
%%%%%%%%%%%%%%%%%%%%%%%%%%%%%%%%%%%%%

随着互联网技术的发展与普及,以及更好应对新冠疫情,混合办公模式逐渐得到部分企业与员工的认可。在线办公技术持续演进,实现形式与办公理念也在发生变革。
在线办公技术的进步离不开基础技术服务的加速跟进。云计算、互联网数据中心(Internet Data Center,IDC)、内容分发网络(Content Delivery Network,CDN)等基础技术服务的发展支撑了在线办公的发展。

以 CDN 为例,根据2022年中国互联网络信息中心(CNNIC)发布的《中国互联网络发展状况统计报告》\cite{cnnic2022} (以下简称《报告》)显示:
在企业数量方面,2020 年,取得内容分发网络牌照的增值电信业务企业数量为 44 家,2021 年前 11 个月已新增 52 家,预计未来仍会保持较快增长;
在技术方面,多家运营商提出 SD-WAN 解决方案,通过优化传输技术,解决企业邮箱、视频会议等系统的加速难题,提升用户体验。

%%%%%%%%%%%%%%%%%%%%%%%%%%%%%%%%%%%%%
%%% 
%%%  
%%% 
%%% 关注政策
%%% 
%%% 
%%% 
%%%%%%%%%%%%%%%%%%%%%%%%%%%%%%%%%%%%%

对于国家层面,
2016年我国《网络安全法》正式通过,“关键信息基础设施安全保护制度”被首次提出,
2019至2021年,《关键信息基础设施安全保护条例》连续三年纳入国家立法计划,历经多年反复锤炼,于2021年8月17日正式发布,并于2021年9月1日起施行。
可见近年来,国家对关键信息基础设施安全的重视程度。CDN作为其中一项关键基础设施,*****************************************************
对关键信息基础设施保护系列制度要素作了具体规定,涵盖总则、关键信息基础设施认定、运营者责任义务、保障和促进、法律责任等诸多方面。


%%%%%%%%%%%%%%%%%%%%%%%%%%%%%%%%%%%%%
%%% 
%%%  
%%% 
%%% 关注安全、服务、价格优势
%%% 
%%% 
%%% 
%%%%%%%%%%%%%%%%%%%%%%%%%%%%%%%%%%%%%

对于行业层面,使用CDN服务的网络内容服务商商(ICP),*************

%%%%%%%%%%%%%%%%%%%%%%%%%%%%%%%%%%%%%
%%% 
%%%  
%%% 
%%% 关注性能
%%% 
%%% 
%%% 
%%%%%%%%%%%%%%%%%%%%%%%%%%%%%%%%%%%%%

对于用户层面,CDN作为一种网络加速产品,广泛应用于图片及小文件下载、大文件下载以及音视频点播等业务场景。
《报告》显示,截至 2021 年 12 月,我国网络视频(含短视频)用户规模达 9.75 亿,较 2020年 12 月增长 4794 万,占网民整体的 94.5\%。
网络视频,尤其是热门网络视频的高并发访问或者访问突增场景下对源站性能要求非常高,且源站的带宽成本也较高。
合理选择一款,或多款CDN服务就显得很有必要。

因此,多方面都需要网络基础设施,尤其时CDN服务的服务质量现状。**************本文将给出一种方法。



%%%%%%%%%%%%%%%%%%%%%%%%%%%%%%%%%%%%%
%%% 
%%% 
%%% 
%%% 阐述为什么是面向DNS全局负载均衡
%%% 
%%% 
%%% 
%%%%%%%%%%%%%%%%%%%%%%%%%%%%%%%%%%%%% 

本文从CDN服务的基本原理出发,收集国内热点域名使用的CDN特征串,构建当前CDN使用量排名。
并根据DNS解析效率、IP时延特征对CDN的服务质量进行测绘,试图给出CDN服务质量排名,为选择CDN服务提供依据和方法支撑,最终达到终端用户获得较优体验的目的。

目前,工业界对CDN全局负载均衡实现方式可概括为下述三类方式。


\begin{enumerate}[label={(\arabic*)}]


	\item 基于DNS全局负载均衡。使用CDN服务的域名通过包括CNAME等技术\cite{Choffnes2017}将自身的解析权指定给CDN服务商提供的智能DNS解析器,由于DNS系统的简单性及其作为目录系统的普遍性\cite{Pathan-survey-2007}等优点,当终端用户向本地递归域名解析器(LDNS)或公共域名解析器(ODNS)请求时,该请求将重定向到CDN服务商进行智能解析,根据终端用户的位置、链路情况等因素分配一个或多个IP,达到负载均衡的目的。
	\item 基于任播(Anycast)IP。Anycast考虑地理(或称网络距离)最近原则,其依赖于BGP路由实现,将同一IP分布在多个AS中,为用户提供相同的服务。当终端用户向Anycast IP请求时,将优先选择BGP路径较短的IP提供服务,达到负载均衡的效果。实际使用中,其负载均衡并未使用全局信息\cite{Calder2015},不能很好应对网络状态的变化\cite{Choffnes2017},因此存在约20\%比例将用户导向次优节点。特别的,中国AS呈现出少而大的特征,AS间的BGP路由配置通常是静态的\cite{Choffnes2017},使用任播实现效果较差。
    \item 基于HTTP重定向等其他方式。HTTP重定向基于状态码3XX或manifest文件重定向\cite{Adhikari2014}等方式,这些方法额外增加了RTT时间,常用于视频分发等领域。不适合一些时延敏感服务,如搜索等。


\end{enumerate}

综上所述,本文主要面向主流\cite{Hao2018}的基于DNS全局负载均衡的CDN服务,根据其实现方法进行服务商识别,服务质量测量与评估,********。
对网络内容服务商提供一种比较依据,个人开发人员提供一种选择,国家进行CDN服务的运行管理具有重要的理论意义和实用价值。






\section{国内外研究现状}[guoneiwaiyanjiuxianzhuang]



%%%%%%%%%%%%%%%%%%%%%%%%%%%%%%%%%%%%%
%%% 
%%% 
%%% 
%%% CDN识别方案
%%% 
%%%  
%%% 
%%%%%%%%%%%%%%%%%%%%%%%%%%%%%%%%%%%%% 
\subsection{CDN厂商识别研究现状}[CDNchangshangshibieyanjiuxianzhuang]
CDN一直伴随这快速分发流量、防御DDos攻击等优点。当前也有着融合CDN产品服务,旨在使用多家厂商提供的CDN服务\cite{Zhu2021},做到相互补充,尽可能达到100\%服务覆盖率等目的。同时,与CDN有关的安全漏洞、攻击防范等问题也有较多关注。如使用CDN作为DDos攻击手段\cite{Guo2020}来瘫痪源站、利用CDN的新型审查规避技术\cite{Wei2021}为违规站点提供保护、未验证源服务器证书\cite{SHOBIRI2021}等危害。

其中,发现CDN攻击事件、漏洞检测及修复、融合CDN选择、CDN服务排名等问题,都离不开CDN厂商的识别。目前,学术界和工业界对CDN厂商的识别可概括为下述三大类方式。


\begin{enumerate}[label={(\arabic*)}]


	\item 关键字匹配。如使用CNAME关键字匹配\cite{Adhikari2014}、根据HTTP错误\cite{Huang2008,Guo2018}的判定、或利用WHOIS\cite{Huang2008}、PTR\cite{Chen2019}信息判定,也可以利用公开IP段匹配CDN\cite{Choffnes2017},更有猜测CDN边缘服务器的命名规则以构建主机名,并进行DNS探测以获得更多的节点IP\cite{Hohlfeld2018,Timm2018}的解决方案。
	\item 机器学习方法。同时,CDN识别与机器学习等方法融合也成为一种趋势。
如利用DNS信息与IP特征的统计识别\cite{Li2020}、基于图模型的CDN节点发现\cite{Ma2021}、基于深度学习的检测方法\cite{Chen2019}以及基于Naive Bayes Tree\cite{Liang2006},结合CNAME、WHOIS、HTTP响应头的统计学习方法\cite{Hou2021}。
	\item 使用互联网开源工具或公开服务接口。如CDN Finder\footnote{https://www.cdnplanet.com/tools/cdnfinder/}、CDN云观测\footnote{https://cdn.chinaz.com/}
	或findCDN\footnote{https://github.com/cisagov/findcdn}等工具。其中,CDN Finder使用页面资源列表来识别每个唯一主机名的CDN,通过获取页面资源并检查响应标头、执行DNS解析并检查完整的CNAME链获取待判定特征,根据
	其维护的Header-to-CDN和CNAME-to-CDN列表来匹配。若匹配失败,则根据其服务IP所属AS来匹配CDN服务商。findCDN则是通过HTTPS服务器响应头、CNAME记录结合IP WHOIS信息来匹配CDN服务商。
\end{enumerate}

综上所述,当前CDN服务识别*********************************************

%%%%%%%%%%%%%%%%%%%%%%%%%%%%%%%%%%%%%
%%% 
%%% 
%%% 
%%% CDN测量方案
%%% 
%%% 
%%% 
%%%%%%%%%%%%%%%%%%%%%%%%%%%%%%%%%%%%% 
\subsection{CDN服11务质量测量方面}[CDNfuwuzhiliangceliangfangmian]
说明当前主要存在的测量手段,测量内容,以及效果等。


研究\cite{Pathan-survey-2007}表明,CDN的业务目标至少包括可扩展性、安全性、可靠性、响应性和性能。
当前,学术界对CDN的研究多种多样,
有对CDN边缘服务器放置算法的研究\cite{Sahoo2016},同时,更有学者大胆提出将IP与主机名解绑\cite{Fayed-IpUnbind-2021},实现一种更灵活的CDN服务

Johnson是最早研究CDN服务质量测量\cite{Johnson-2001-cdn-measure}之一,并指出CDN的价值并非提供最佳的分发点,而是避免给出性能明显不行的。

根据以往的研究\cite{Huang2008},CDN主要存在两方面时延:DNS解析时延,也即CDN内部DNS系统向终端用户提供“最佳”CDN边缘节点地址的时间;以及内容服务器时延,及终端用户和所选CDN服务器之间的往返时间。Krishna等人提出了King\cite{King-2002}方法来测量端到端估计距离,但由于其存在缓存污染、流量开销较大问题,后续有人提出了一种轻量化测量平台\cite{Zhang-2021-Scale-platform},总之这里要给出一个测量方法。

同时,根据阿里云\footnote{https://help.aliyun.com/document\_detail/140425.html}及腾讯云\footnote{https://cloud.tencent.com/document/product/228/1198}公开的CDN性能衡量指标,解析时延和可用性也是两个重要方面。

在DNS解析过程中,需要对LDNS和ODNS进行约减,减少发包,还要对CNAME进行筛选,减少发包。主要看获取到的IP是否量级相同。



\subsection{存在的主要问题}[]

1.CDN识别方面,CNAME、ASN收集方面,需要投入大量前期工作。当前给出的基于机器学习的方法需要预先给出大量原始标注数据集,较难构造。因此,本文打算研究TSVM方法。


给出当前识别方面的问题,服务质量测量方面存在的不足。说明我自己要做一个分类器来识别,做一个方面的测量指标
\section{研究内容及论文组织结构}[yanjiuneirongjilunwenzuzhijiegou]
本小节对课题的研究内容进行阐述,并介绍文章组织结构。研究内容描述了研究对象及相关解决方案,组织结构部分对文章行文总体结构进行概括展示。
\subsection{论文研究内容}[lunwenyanjiuneirong]
\subsection{论文组织结构}[lunwenzuzhijiegou]

% Local Variables:
% TeX-master: "../thesis"
% TeX-engine: xetex
% End:
