% !Mode:: "TeX:UTF-8"

\subsection{CDN服务商识别研究现状}

CDN研究的首要步骤为CDN服务商识别。目前,学术界和工业界对CDN厂商的识别可概括为下述三种方式。
 
(1) \textbf{关键字匹配}。许多研究\cite{Huang2008,Adhikari2014,Guo2018}使用规范主机名(Canonical Name,CNAME)关键字作为基础匹配机制,对域名的CNAME记录与CDN服务商提供的特征后缀匹配。其次,根据HTTP错误日志\cite{Huang2008,Guo2018}也可以辅助识别,根据响应报文检验,判定其是否包含CDN服务商的关键字。也有利用域名或IP的WHOIS\cite{Adhikari2014}信息、指针记录\cite{Chen2019}(Pointer record,PTR)与CDN服务商关键字进行匹配。或是根据域名解析的IP地址,是否归属于CDN服务商公开的IP范围进行识别\cite{Choffnes2017}。特别的,猜测CDN边缘服务器的命名规则以构建主机名,并进行DNS探测以获得更多的节点IP\cite{Hohlfeld2018,Timm2018},也是一种解决方案。

(2) \textbf{机器学习融合方法}。CDN识别与机器学习等方法融合也成为一种趋势。国家互联网应急中心\cite{Li2020}(CNCERT)的研究表明,使用CDN的域名,其在IP和TTL的特征有明显区分度,且样本识别精度可以达到90\%以上。Ma\cite{Ma2021}提出了基于半监督图神经网络(GNN)模型的CDN节点发现,将域名与IP二部图关联进一步细化,降低误判。Chen\cite{Chen2019}提出基于长短期记忆网络(LSTM)的Fast-flux和CDN域名识别方法(  Fast-flux  and  CDN  Domains  Recognize,FCDR),将域名字符、经验信息、地理和时间相关特征作为输入,对域名进行CDN服务判定。Hou\cite{Hou2021}提出了基于朴素贝叶斯树(Naive Bayes Tree,NB-Tree)\cite{Liang2006},结合CNAME、WHOIS、HTTP响应头的统计学习方法\cite{Hou2021}。国防科技大学的闫志豪\cite{yan-2022}等人也提出了一种基于域名系统知识图谱的CDN域名识别技术,使用域名的MX、NS、CNAME、IP、地理位置等信息综合识别。 

(3) \textbf{互联网开源工具或公开服务接口}。如CDN Finder\footnote{https://www.cdnplanet.com/tools/cdnfinder/}、CDN云观测\footnote{https://cdn.chinaz.com/}
或findCDN\footnote{https://github.com/cisagov/findcdn}等工具。其中,CDN Finder使用页面资源列表来识别每个唯一主机名的CDN,通过获取页面资源并检查响应标头、执行DNS解析并检查完整的CNAME链获取待判定特征,根据
其维护的Header-to-CDN和CNAME-to-CDN列表来匹配。若匹配失败,则根据其服务IP所属AS来匹配CDN服务商。findCDN则是通过HTTPS服务器响应头、CNAME记录结合IP WHOIS信息来匹配CDN服务商。 


% Local Variables:
% TeX-master: "../section_2_研究现状"
% TeX-engine: xetex
% End: