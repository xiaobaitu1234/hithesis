% !Mode:: "TeX:UTF-8"

\section{为完成课题已具备和所需的条件和经费}
本课题组所在的研究中心可为本项目研究提供良好的支撑条件,
拟完成课题所需条件如表 \ref{table:课题所需条件表} 所示。

\begin{table}[htbp]
	\centering
	\caption{课题所需条件表}\label{table:课题所需条件表}
	\vspace{0.5em}\wuhao
	\begin{tabular}{ccc}
		\toprule
		所需条件        & 具体说明     & 备注    \\
		\midrule
		代理           & 国内分布广泛的代理节点,链路性能良好           & 待购置      \\
		探测节点          & 探测所需节点,构建分布式系统  & 待购置  \\
		开发工具         & MySQL,RabbitMQ   & 已具备 \\
		开发语言           & Python3.9  & 已具备 \\
		相关数据          & CDN识别初始特征集 &     已具备     \\


		\bottomrule
	\end{tabular}
\end{table}

目前需要经费购置代理以及探测节点,后续根据实际情况考量。

\section{预计研究过程中可能遇到的困难和问题,以及解决的措施}
本节将主要针对可能遇到的困难和问题提出相应可能的解决方法。

(1) CDN识别过程中,HTTP转换特征过多问题

\textbf{解决措施}:拟采用特征融合,更换特征提取方法等方式解决,或请教老师寻求思路。

(2) 时延特征测量结果的可靠性、完整性问题

\textbf{解决措施}:针对数据可靠性,拟采用长时间探测消除瞬时网络波动带来的影响,针对数据完整性问题,拟从冗余代理、增加协同调度机制来保障。

(3) 在放置优化策略实现过程,或其他环节中会遇到技术问题

\textbf{解决措施}:拟尝试通过查找资料,或请教老师等方式解决。


% Local Variables:
% TeX-master: "../report"
% TeX-engine: xetex
% End: