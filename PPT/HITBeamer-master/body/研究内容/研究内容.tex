\section{研究内容}
 
\subsection{可信环境建立}

\begin{frame}{待分析流量获取}
    \begin{itemize}[<+->]
        \item 操作系统
        \begin{itemize}[<+->]
            \item 环境搭建(借助QEMU\upcite{敖权2018基于QEMU的Linux应用异常通信行为分析}等\alert{虚拟化}平台)
            \item 干扰过滤(使用\alert{Whois}信息剔除非OS厂商域名所属软件/插件)
        \end{itemize}
        \item 流量提取(\alert{完整性})
        \begin{itemize}[<+->]
            \item 理论层面分析(提取\alert{原理})
            \item 宿主机流量对比校验(从流量\alert{路径}多点对比)
            \item 多种\alert{提取方式}对比校验
        \end{itemize}
        \item 数据预处理
        \begin{itemize}[<+->]
            \item 流量分组(聚集域名通信行为的协议栈流量)
            \item 服务定位(\alert{标注}流量组对应的服务,含功能、地址、归属\ldots)
        \end{itemize}
    \end{itemize}
\end{frame}

\subsection{流量内部分析(内容层面)}

\begin{frame}{流量内容敏感性判定}
    \begin{itemize}
        \item 目的:敏感流量/文件传输行为发现
        \item 方案:
            \begin{itemize}
                \item 加密流量解析
                \item 可定制化敏感流量分析模型
                    \begin{itemize}
                        \item 可配置匹配特征库
                        \item 流量敏感性分析
                        \item 流量归属判别(如隶属应用、端口\ldots)
                    \end{itemize}
            \end{itemize}
        \item 预期结果
            \begin{itemize}
                \item 预警敏感外泄
                \item 期望做到敏感流量阻断
            \end{itemize}
    \end{itemize}
\end{frame}


\subsection{流量外部分析(特征层面)}

\begin{frame}{流量特征风险性判定}
    \begin{itemize}
        \item 目的:流量攻击检测
        \item 方案
            \begin{itemize}
                \item 数据集:\href{https://www.kdd.org/kdd-cup/view/kdd-cup-1999}{KDD CUP99}
                \item 识别模型:SVM、随机森林、决策树\ldots
            \end{itemize}
        \item 预期结果:给定流量特征,推断其为攻击流量的概率
    \end{itemize}
\end{frame}
