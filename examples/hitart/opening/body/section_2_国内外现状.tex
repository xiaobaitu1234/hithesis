% !Mode:: "TeX:UTF-8"

\section{国内外研究现状及分析}

本节对国内外与CDN服务商识别技术、CDN时延特征测量技术以及CDN放置优化策略相关的研究进行介绍。

% !Mode:: "TeX:UTF-8"

\subsection{CDN服务商识别研究现状}


CDN服务商识别作为从外部研究CDN的必须环节,有着许多应用场景。如CDN服务使用量排名,统计当前域名排名情况,分析其使用CDN服务商的使用量,提供选择依据。同时,与CDN有关的安全漏洞、攻击防范等问题也有较多关注。如使用CDN作为DDos攻击手段\cite{Guo2020}来瘫痪源站、利用CDN的新型审查规避技术\cite{Wei2021}为违规站点提供保护、未验证源服务器证书\cite{SHOBIRI2021}等危害。相应的,Jalalpour\cite{Jalalpour2018}从CDN边缘服务器的安全问题入手,使用可动态构建的虚拟化安全功能链实现防护功能。实现低开销应对威胁、迅速恢复吞吐量的目的。综上所述,发现CDN攻击事件、漏洞检测及修复、融合CDN选择、CDN服务排名等问题,都离不开CDN厂商的识别。

目前,学术界和工业界对CDN厂商的识别可概括为下述三种方式。
 
(1) 关键字匹配。许多研究\cite{Huang2008,Adhikari2014,Guo2018}使用CNAME关键字作为基础匹配机制,对域名的CNAME记录与CDN服务商提供的特征后缀匹配。其次,根据HTTP错误日志\cite{Huang2008,Guo2018}也可以辅助识别,根据响应报文检验,判定其是否包含CDN服务商的关键字。也有利用域名或IP的WHOIS\cite{Adhikari2014}信息、PTR\cite{Chen2019}记录与CDN服务商关键字进行匹配。或是根据域名解析的IP地址,是否归属于CDN服务商公开的IP范围进行识别\cite{Choffnes2017}。特别的,猜测CDN边缘服务器的命名规则以构建主机名,并进行DNS探测以获得更多的节点IP\cite{Hohlfeld2018,Timm2018},也是一种解决方案。

(2) 机器学习等方法。CDN识别与机器学习等方法融合也成为一种趋势。CNCERT\cite{Li2020}的研究表明使用CDN的域名,其在IP和TTL的特征有明显区分度,且样本识别精度可以达到90\%以上。Ma\cite{Ma2021}提出了基于半监督GNN模型的CDN节点发现,将域名与IP二部图的关联进一步细化,降低误判。Chen{Chen2019}提出基于LSTM的FCDR方法,将域名字符、经验信息、地理和时间相关特征作为输入,对域名进行CDN服务判定。Hou\cite{Hou2021}提出了基于Naive Bayes Tree\cite{Liang2006},结合CNAME、WHOIS、HTTP响应头的统计学习方法\cite{Hou2021}。国防科技大学的闫志豪\cite{yan-2022}等人也提出了一种基于域名系统知识图谱的CDN域名识别技术,使用域名的MX、NS、CNAME、IP、地理位置等信息综合识别。 

(3) 使用互联网开源工具或公开服务接口。如CDN Finder\footnote{https://www.cdnplanet.com/tools/cdnfinder/}、CDN云观测\footnote{https://cdn.chinaz.com/}
或findCDN\footnote{https://github.com/cisagov/findcdn}等工具。其中,CDN Finder使用页面资源列表来识别每个唯一主机名的CDN,通过获取页面资源并检查响应标头、执行DNS解析并检查完整的CNAME链获取待判定特征,根据
其维护的Header-to-CDN和CNAME-to-CDN列表来匹配。若匹配失败,则根据其服务IP所属AS来匹配CDN服务商。findCDN则是通过HTTPS服务器响应头、CNAME记录结合IP WHOIS信息来匹配CDN服务商。 

% \begin{enumerate}[label={(\arabic*)}]

% 	\item 
% 	\item 
% 	\item 
% \end{enumerate}

从CDN识别精度而言,不同研究发现的数量不尽相同。基于匹配机制的识别方法一般精度在85\%以上,对于当前公开的CDN识别服务,如CDN Finder、CDN云观测以及findCDN等,其内在机制也是使用特征库对所选取特征进行匹配操作。集成机器学习或深度学习技术后,精度有进一步提高,如使用域名IP特征和TTL特征的方案可以达到90\%的精度,对于使用更多信息源的深度学习方案\cite{Ma2021},其样本精度可以达到98\%。相应的,高精度的代价是使用更多信息源。

从解析开销而言,Adhikari\cite{Adhikari2014}基于主被动测量,使用规范主机名(CNAME)、IP地址的WHOIS进行所有者识别,从而判定Hulu和Netflix使用的CDN服务商。Huang\cite{Huang2008}基于获取大量Windows Live搜索日志,从主机名中提取CNAME记录,并判定其是否属于特定目标CDN。以此来构建其特征识别库。Ma\cite{Ma2021}的图神经网络方法,更是集成了地理、时间等信息,在获得高精度的同时也极大增加了数据探测成本。 


% Local Variables:
% TeX-master: "../section_2_研究现状"
% TeX-engine: xetex
% End: 
% !Mode:: "TeX:UTF-8"

\subsection{CDN时延特征测量研究现状}
 
内容分发网络在建立之初就肩负着缓解源站压力,提高用户体验的使命。CDN的时延特征对于终端用户而言比较敏感。Johnson是较早研究CDN服务质量测量\cite{Johnson-2001-cdn-measure}之一,并指出CDN的价值并非提供最佳的分发点,而是避免给出性能明显较差的。Pathan认为,CDN性能\cite{Pathan-survey-2007}需要关注五个衡量指标:缓存命中率、预留带宽、代理服务器利用率、时延以及可靠性。Huang的研究指出\cite{Huang2008},CDN主要存在两方面时延:DNS解析时延,也即CDN内部DNS系统向终端用户提供“最佳”CDN边缘节点的时间;内容服务器时延,即终端用户和所选CDN服务器之间的往返时间。国内CDN服务商也针对CDN性能衡量提出通用指标,如阿里云CDN\footnote{https://help.aliyun.com/document\_detail/140425.html}和腾讯云\footnote{https://cloud.tencent.com/document/product/228/1198}认为CDN的时延性能、命中率以及丢包率较为重要。结合两者观点,可以得出如图 \ref{fig:CDN性能指标}所示CDN性能指标结构。其中缓存命中率主要受CDN配置的缓存策略影响。预留带宽指为源站预留的带宽,通常以字节为单位。代理服务器利用率指服务器忙时占比,管理员可以根据这个指标进一步计算CPU负载、服务请求数和存储I/O的使用情况等。时延,指用户感知的响应时间,根据访问流程可大致拆分为DNS时间、TCP时间和SSL时间。可靠性通常由丢包率指代,高可靠表明丢包率低,且始终对用户可用。   
  
\begin{figure}[ht]
	\centering
	\begin{tikzpicture}[grow cyclic, text width=2.7cm, align=flush center,
    level 1/.style={level distance=5cm,sibling angle=60},
    level 2/.style={level distance=3cm,sibling angle=60}]
  
    \node{CDN性能指标}
    child { node {缓存命中率}  }
    child { node {预留带宽} }
    child { node {时延} 
      child { node {SSL时间}}
      child { node {TCP时间}}
      child { node {DNS解析时间}}
    }
    child { node {代理服务器利用率} }
    child { node {可靠性}  };
  \end{tikzpicture}

	\caption{CDN性能指标}
	\label{fig:CDN性能指标}
\end{figure}
\FloatBarrier

针对网络时延测量方面的研究主要体现在测量手段和可靠性保障两方面。北京航空航天大学的张兴军\cite{2005内容分发网络性能测量方法研究与实现}等人借助可定制的网络测量基础设施CNMI构建了CDN测量平台CDNPMS。吴金福\cite{吴金福2014中国大陆}借助HTTP代理对CDN进行测量。Krishna等人提出了King\cite{King-2002}方法来测量端到端的估计时延,Huang\cite{Huang2008}在对CDN进行测绘时采取的增强King方法便由此改进而来。Zhang\cite{Zhang-2021-Scale-platform}基于DNS欺骗原理,提出了一种轻量化统计延迟测量平台DMS,使用两个DNS服务器之间的时延代表两个主机之间的端到端时延。北京邮电大学康梦晓\cite{康梦晓2014IP}利用PLE算法进行测量,并在分布参数估计部分进行改进,减少了时延量化间隔数,降低PLE算法的复杂度。胡治国\cite{胡治国2017IP}等人指出,当前测量难点主要体现在单向时延测量,由时钟频差和时钟偏差的相互作用以及非对称路径下的时延测量两方面阻碍。


% Local Variables:
% TeX-master: "../section_2_研究现状"
% TeX-engine: xetex
% End:
% !Mode:: "TeX:UTF-8"
 
\subsection{软件可疑行为检测}


% Local Variables:
% TeX-master: "../section_2_研究现状"
% TeX-engine: xetex
% End:
 

\subsection{国内外研究现状分析}
由上文可知,研究人员在CDN服务商识别、CDN时延特征测量,节点分布优化策略方面展开了研究,并提出了一些有效的理论和方法,然而仍存在下述问题。

(1) 针对CDN服务商识别,不同研究达到的识别精度与探测开销不尽相同,CDN服务识别在基于匹配规则时有良好表现,尤其是在CNAME特征匹配时,精度较高。在使用多种匹配规则进行联合识别时,识别分类的精度略有提高。在此基础上,引入机器学习和深度学习,同时增加多种信息源(如地理位置、TTL特征、IP特征等)后,其识别精度进一步提高,但已产生边际效应,探测资源再投入对精度的提高收益显著减少。

(2) 针对CDN性能测量,许多研究都关注了时延特征。其中,DNS解析时延主要依赖于用户、递归DNS服务器、CDN DNS服务器三者的链路时延,SSL时延的主要影响因素同理。因此,许多研究都从TCP时间入手,等价的,TCP时间一般情况下与IP链路时延正相关。Johnson\cite{Johnson-2001-cdn-measure}仅使用HTTP下载时间对CDN边缘服务器节点进行性能测量,且针对CDN就近解析条件考虑不足。张兴军\cite{2005内容分发网络性能测量方法研究与实现}等人提出的CDNPMS方法,依赖于大规模的测量网络。吴金福\cite{吴金福2014中国大陆}借助HTTP代理的测量方式较为符合中国大陆的互联网情况,基于HTTP响应收集边缘服务器,并记录时延分析。具体的,针对IP时延测量方面,目前常用King及其增强方法。King方法利用随机域名,容易造成缓存污染\cite{Huang2008}问题。Zhang\cite{Zhang-2021-Scale-platform}利用DNS欺骗进行时延测量,使用两个主机临近的DNS服务器间时延作为近似值。back-to-back\cite{康梦晓2014IP}方法主要侧重点为单向链路时延,由于单向时延需要考虑时钟问题,且由于往返链路并非完全对等\cite{胡治国2017IP},所以单向时延不能完全代表往返时延。

(3)针对CDN放置优化研究,许多研究主要以QoS和成本作为优化问题的目标,针对缓存路径或边缘服务器放置进行优化。其中,针对缓存的优化着重考虑了缓存层级扩展、缓存路径选择、缓存替换策略等。对于边缘服务器放置主要使用整数线性规划对图模型进行最优化处理,得到一组候选位置的最优子集。由于缓存算法依托于边缘服务器内容放置,但并非完全相关,单方面考虑优化不能保证得到全局最优结果。同时,随着虚拟化技术(如云、SDN、NFV等技术)的发展,动态部署CDN边缘服务器作为一种新兴技术,使得边缘服务器部署可动态调整,针对服务器放置和路径选择问题进行联合优化成为可能。

 
因此,为对CDN服务的服务质量进行测绘并权衡成本及服务质量,如何对CDN服务进行高效、准确的发现与识别,同时减少探测流量对实网环境的影响;如何保证对CDN时延特征的可靠、完整测量;如何对CDN的放置进行优化,统筹考虑边缘服务器放置和缓存链路选择的关系,是当前国家保障互联网稳定运行所需要解决的问题,也是本文的研究目标。 


% Local Variables:
% TeX-master: "../report"
% TeX-engine: xetex
% End: