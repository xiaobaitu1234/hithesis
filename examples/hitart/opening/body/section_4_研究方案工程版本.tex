% !Mode:: "TeX:UTF-8"

\section{研究方案及进度安排,预期达到的目标和已取得的研究成果}
本节对研究方案及进度安排,预期达到的目标和已取得的研究成果进行介绍。
\subsection{研究方案}

为实现对典型系统与软件可疑通信行为分析,
首先需要采集多设备多通信协议流量,该数据需要设计实现基于动态分布式的设备多通信协议实时流量采集系统得到;
其次,需要设计可信性验证,并基于多维特征协同计算构建软件可信行为特征库;
最后,需要设计基于相异行为特征的软件可疑行为检测与分析方案,对可信流量输入扩充,对可疑流量分析预警。


\subsubsection{基于动态分布式的设备多通信协议实时流量采集}

为实现多设备的多通信协议流量高效采集,本研究设计基于动态分布式的设备多通信协议实时流量采集系统,包括任务管理模块、采集探针模块、结果预处理模块以及系统健康度巡查模块。系统的架构示意图如图 \ref{fig:基于动态分布式的设备多通信协议实时流量采集}所示:
\FloatBarrier
 
\begin{figure}[ht]
  \centering
  \includegraphics[width = 0.9\textwidth]{基于动态分布式的设备多通信协议实时流量采集.pdf}
  \caption{基于动态分布式的设备多通信协议实时流量采集功能模块示意图}
  \label{fig:基于动态分布式的设备多通信协议实时流量采集}
\end{figure} 

\FloatBarrier

(1) \textbf{任务管理模块}\quad
该模块通过消息队列向分布式集群下发定时/实时探针控制任务,支持多通信协议流量采集任务的动态控制,以及探针时钟同步、任务同步功能。基于订阅发布模式实现集群动态扩容,支持系统运行时,探针动态加入或退出采集系统,保证系统全时段无间断运行。确保数据采集完整性,任务执行高效性。

(2) \textbf{采集探针模块}\quad
该模块负责解释执行任务管理模块下发的控制任务,实时监测并采集宿主终端的通信流量,主要包括DNS流量、IP流量、HTTP流量及其他流量,并记录采集运行日志。为结果预处理模块提供终端原始通信协议流量数据,为系统健康度巡查提供数据支持。

(3) \textbf{结果预处理模块}\quad
该模块负责收集分布式终端设备多通信协议流量采集结果。在可控环境下,使用差分机制提取特定软件产生的协议流量,并根据多组重复实验数据交叉验证,为可信行为特征库构建提供高可信流量。在实网环境下,基于重复实验数据交叉验证,为可疑行为检测分析提供灰色流量。

(4) \textbf{系统健康度巡查模块}\quad
该模块负责分布式终端日志收集、探针采集进度汇总以及各探针运行状态巡查工作。
探针采集进度汇总用于观测流量采集任务执行情况,指导后续采集任务方案。
探针运行状态巡查
从系统资源角度获取CPU使用率、内存占用等,
从网络状态角度获取IP地址、设备MAC地址、带宽占用情况等。
根据各探针巡查状态综合分析,对系统采集能力及负载情况宏观把控,对探针异常情况及时发现和预排查。
日志收集工作主要面向探针异常处理,对运行状态信息预查无法定位的异常情况,将采取日志分析手段排查探针异常原因,有针对性地提高系统稳定性。

\subsubsection{基于多维特征协同计算的软件可信行为特征库构建}

为了对软件可信行为特征进行全面、透明和可靠地画像,本研究设计基于多维特征协同计算的软件可信行为特征库构建方案,包括待验证流量输入、可信性验证以及软件可信行为特征库三部分。研究模块结构示意图如图 \ref{fig:基于多维特征协同计算的软件可信行为特征库构建}所示:

\FloatBarrier
 
\begin{figure}[ht]
  \centering
  \includegraphics[width = 0.9\textwidth]{基于多维特征协同计算的软件可信行为特征库构建.pdf}
  \caption{基于多维特征协同计算的软件可信行为特征库构建功能模块示意图}
  \label{fig:基于多维特征协同计算的软件可信行为特征库构建}
\end{figure} 

\FloatBarrier


(1) \textbf{待验证流量输入}\quad
该部分负责与通信流量获取以及可疑行为检测分析对接,分别接收来自可控环境下的高可信流量,如演练环境下长期采集的可靠数据高概率为可信流量,以及来自实网环境下检测分析得出的可信扩充流量,如灰色流量经检测得出的可信部分。为后续可信性验证提供原始数据。

(2) \textbf{可信性验证}\quad
该部分基于时间窗口及网络空间关联协同验证机制判别输入流量的可信性,时间窗口验证用于校验高概率不随时间变动的稳定特征,网络空间关联验证则校验较高时间随动率的不稳定特征。具体的,时间窗口验证可通过IP ASN验证、WHOIS资源验证、通信协议功能验证以及证书资源验证组成。网络空间关联验证可使用地址资源验证、协议内容验证、搜索引擎验证以及IP归属验证考量。通过可信性验证的通信行为特征将加入到软件可信行为特征库中。

(3) \textbf{软件可信行为特征库}\quad
该部分负责存储并管理软件可信行为,根据协议的多维特征协同构建可信行为网格,形成软件可信行为特征库。具体的,从目标维度可构建协议目标特征库,由内容维度可构建协议内容特征库。各维度特征库联结,实现对软件多协议流量目标定位和协议内容把控,为可疑行为检测分析提供有力支撑。 
 
\subsubsection{基于相异行为特征的软件可疑行为检测与分析}

为了对实网环境中软件产生的可疑行为进行准确、及时发现与分析,
本文设计基于相异行为特征的软件可疑行为检测与分析方案,包括匹配机制、可疑分类机制以及可疑行为分析三部分,实现对可疑行为检测,可疑行为成因分析的目的,并提供预警机制。研究模块结构示意图如图 \ref{fig:基于相异行为特征的软件可疑行为检测与分析}所示:

\FloatBarrier
 
\begin{figure}[ht]
  \centering
  \includegraphics[width = 0.9\textwidth]{基于相异行为特征的软件可疑行为检测与分析.pdf}
  \caption{基于相异行为特征的软件可疑行为检测与分析功能模块示意图}
  \label{fig:基于相异行为特征的软件可疑行为检测与分析}
\end{figure} 

\FloatBarrier

(1) \textbf{匹配机制}\quad

该部分负责与通信流量获取以及软件可信行为特征库对接,接收来自通信流量获取系统采集的灰色流量,结合软件可信行为特征库中的可信特征,根据相异目标匹配以及相异内容匹配分离灰色流量中可信流量与可疑流量。其中,相异目标匹配借助目标特征库对流量目标进行差异性对比,如目标地址匹配及预期结果匹配等,相异内容匹配使用内容特征库对流量内容进行对比,如协议内容匹配及隐私内容识别等。判别结果中,可信流量经可信性验证后将扩充到可信行为特征库中,可疑流量或成匹配失败流量将进入可疑分类进一步判别。

(2) \textbf{可疑分类机制}\quad
可疑分类机制承接匹配机制未能判别的流量,通过隐私数据保护以及流量特征对比两种方式判定流量是否存疑。
其中,隐私数据保护倾向数据内容侧,基于广告匹配、涉密匹配等手段发现隐私泄露情况。具体的,通过识别流量中与广告相关域名及IP流转情况,分析软件是否利用隐私数据构造推送;通过使用涉密匹配,识别流量中是否存在涉密文件传输情况,分析软件是否存在窃取涉密文件行为。
流量特征对比倾向流量行为侧,基于Top N匹配、端口匹配、地址异常、解析异常方式判定流量存疑情况。
其中,通过识别Top N可用于DDoS攻击的筛查,具体表现为对一个或多个目标的IP或者端口发送超出正常阈值的连接请求。
使用端口匹配可疑帮助筛查部分病毒,如勒索病毒、硬盘杀手等。此类病毒通常会使用固定协议对目标主机的固定端口发送固定字节数报文。
地址异常,主要针对流量的源IP地址或目的IP地址进行检测,着重关注回环地址、广播地址以及源IP与目的IP相同情况,判别是否可能为攻击流量。
解析异常用于筛查DNS异常解析情况,观察其是否使用非本机配置DNS服务器,并根据解析频率,解析失败率联合分析可疑情况。
 
(3) \textbf{可疑行为分析}\quad

可疑行为分析负责对可疑分类机制得到的可疑情况进行流量功能判定,将可疑流量划分为攻击流量、涉密流量、隐私流量及其他流量。其中,涉密流量与涉密匹配机制强关联,攻击流量源于流量特征对比,隐私流量参考广告匹配机制。经行为分析后的流量将根据其功能判定为可信行为和风险行为,可信流量用于扩容可信行为特征库,风险流量将提供预警机制。

\subsection{预期达到的目标和取得的研究成果}
本节对预期达到的目标和取得的研究成果进行介绍。
\subsubsection{预期达到的目标}

通过前期方案准备,本次拟达到的目标如下:

(1) 设计实现基于动态分布式的设备多通信协议实时流量采集系统,实现多设备的多通信协议流量高效动态采集功能。

(2) 设计基于多维特征协同计算的软件可信行为特征库构建方案,完成多协议多特征维度可信行为的网格化存储。

(3) 设计基于相异行为特征的软件可疑行为检测与分析,完成对实网环境中软件产生的可疑行为及时发现与分析预警。

\subsubsection{已取得的研究成果}

	\noindent\textbf{(一)已发表论文}
	\begin{enumerate}
	\item Li C, Cheng Y, \textbf{Men H}, et al. Performance Analysis of Root Anycast Nodes Based on Active Measurement[J]. Electronics, 2022, 11(8): 1194. \quad (EI检索)
\end{enumerate}
\noindent\textbf{(二)申请及已获得的专利}
\begin{enumerate}
\item 张兆心,李超,程亚楠,郭长勇,杜跃进,\textbf{门浩}. 一种获取网络数据时网络阻塞造成的噪声数据消除方法: 中国,CN202110121032.7[P]. 2022-04-15. (已授权)
\item 张兆心,\textbf{门浩},程亚楠,梁浩宇,郭长勇,李超,赵东. 一种多设备网页中内嵌广告获取以及恶意性识别的方法: 中国,202210606165.8[P]. 2022-08-09. (已受理)
\item 张兆心,李超,程亚楠,陆柯羽,\textbf{门浩}. 一种基于多源信息定位域名根镜像节点地理位置的方法: 中国,CN202110856090.4[P]. 2021-11-02. (已受理)
\item 张兆心,李超,柴婷婷,程亚楠,陆柯羽,郭长勇,杜跃进,\textbf{门浩}. 基于网络服务商国别标注的域名国家可控性评估方法:中国, CN202110258091.9[P]. 2021-06-01.(已受理)
\end{enumerate}

 
\subsection{进度安排}
论文进度安排如表 \ref{table:进度安排}所示:


\begin{table}[htbp]
	\centering
	\caption{进度安排}\label{table:进度安排}
	\vspace{0.5em}\wuhao
	\begin{tabular}{ccc}
		\toprule
		起始日期        & 截止日期     & 进度安排    \\
		\midrule
		2022年05月           & 2022年07月           & 阅读文献、调研、数据验证      \\
		2022年07月          & 2022年08月      & 开题报告撰写,研究流程设计  \\
		2022年08月         & 2022年11月        & 构建动态分布式通信流量采集系统,建立可信行为特征库  \\
		2022年11月         & 2022年02月     & 建立可疑分类检测预警机制,提供灰色流量分类功能  \\
		2022年02月           & 2023年03月   & 验证预警分类功能正确性,优化系统流程 \\
		2023年03月          & 2023年05月     &      撰写毕业论文     \\


		\bottomrule
	\end{tabular}
\end{table}


\FloatBarrier




% Local Variables:
% TeX-master: "../report"
% TeX-engine: xetex
% End: