% !Mode:: "TeX:UTF-8"

\section{研究方案及进度安排,预期达到的目标和已取得的研究成果}
本节对研究方案及进度安排,预期达到的目标和已取得的研究成果进行介绍。
\subsection{研究方案}
本研究的主要内容从CDN服务商识别技术、。。。。。。。。。。。。。。。三个方面展开,用以实现。。。。。。。。。。。。。的。。。。。

\subsubsection{基于探测空间约减的CDN服务商识别技术}
本研究方案主要用于。。。。。。。从而实现。。。。。。


(1)\textbf{基于FDNS聚类的探测空间约减方法研究}  

本文研究识别数据获取过程中,发包量主要集中在DNS解析部分,且主要考虑其覆盖面和解析目标选取合理性。针对覆盖面问题,本文考虑使用基于FDNS聚类的探测方法,对于解析目标,本文将选取重点域名排名,具体介绍如下。

1)CDN服务在进行智能DNS解析,会根据解析请求的发送方特征,返回相应较优的A记录,使用户得到良好的时延体验。因此,在模拟多地域探测时(即使用较少探针或单探针解析,获取与多个不同地域探针解析覆盖效果相同的结果集)。需要借助模拟解析地区的区域递归DNS服务器(LDNS),并舍弃EDNS扩展选项,因为EDNS可能会携带请求方IP地址,使得CDN根据智能DNS返回重复结果。针对这种使用LDNS作为请求目标服务器的方法,本文称之为基于LDNS聚类的探测方法。Huang\cite{Huang2008}使用基于LDNS聚类的探测方法,利用公开LDNS作为解析器,获取多个区域的解析结果。但由于许多LDNS被配置非公开,即只为其内部用户进行DNS解析,因此覆盖面并不完整。哈尔滨工业大学陆柯羽研究表明,转发DNS(FDNS)占据开放DNS比重达40\%左右DNS\cite{lukeyu-2021-DNS},远高于递归DNS占比。且转发DNS采用将解析请求转达递归等有解析能力的DNS服务器进行解析,即FDNS必然对应一个或多个递归DNS。因此,本文拟采用FDNS作为解析的请求目标,故称为基于FDNS聚类的探测方法。
 

2)重点域名数据集选取,由于Alexa排名目前处于停止更新状态,本文考虑当前较为出色的域名排行:Traco\cite{Pochat2018}排名以及SecRank\cite{Xie2022}排名。其中,Traco融合了多种国际排名,如Umbrella、Majestic以及Farsight,具有较好的研究价值。但由于其提供域名多为主域名,不适合用来做CNAME检测及HTTP获取。因此,本文选择具有包含多种标签长度域名的SecRank排名作为研究对象。进一步的,针对排名域名选择,本文考虑使用www域名以及SLD域名作为研究对象,因为它们大概率是站点。从减少探测目标的角度来约减DNS发包量,从而降低探测对实网环境的污染。 


(2) \textbf{面向HTTP特征的CDN识别特征集扩充方法研究}

本研究使用域名的HTTP特征进行是否使用CDN服务的判定。并提取使用CDN服务的HTTP特征对应的CNAME记录,并根据CNAME记录的特征进行人工判定,将判定为CDN服务商的CNAME特征串扩充到CDN特征库中。

1) 数据集构建

通常,分类器需要提供一定规模的正例和反例进行训练,基于DNS解析得到的域名CNAME、以及HTTP Head命令解析得到的域名HTTP信息,本文给出如图\ref{fig:解析数据预分类}所示的分类过程,保守得将数据划分为使用CDN服务、未使用CDN服务、待分类数据、无效数据四类。该分类依据算法需要,过滤掉信息不完整内容,即将不含HTTP信息或不含CNAME信息得数据作为无效数据。随后,将明确匹配成功的数据作为使用CDN服务数据,将HTTP信息不含跳转特征,并且多地解析结果相同的数据,视为未使用CDN服务数据。此后,剩余数据为待分类数据,CDN特征库将根据待分类数据的识别结果进行选择性扩充。
 
\begin{figure}[h]
	\centering
	\includegraphics{解析数据预分类.pdf}
	\caption{解析数据预分类}
	\label{fig:解析数据预分类}
\end{figure}
\FloatBarrier

2) HTTP分类器构建
该部分利用前述使用CDN服务和未使用CDN服务的HTTP信息进行特征提取,主要考虑使用TF-IDF方法对关键字进行分类效益排序,提取主要的特征作为编码依据,将HTTP进行特征编码。后续也将尝试PCA、LDA等主成分分析方法进行特征提取,需要根据实验结果进行探讨。针对分类器构建问题,当前需求明确为二分类问题,可以考虑使用支持向量机(SVM)来进行解决。同时,由于当前数据的标签较为保守,待分类数据中可能包含较多未明确标签的数据,因此,本研究拟采用TSVM\cite{Joachims1999}(一种经典半监督 SVM)方法来对待分类数据进行判定。


4) CDN特征库扩充

作为辅助识别特征,HTTP分类得主要目的为增强主特征的识别精度,即学习新的CNAME特征串,扩充到CDN特征库中。针对待分类数据进行分类判定后,将主要针对分类器判定为使用CDN服务的域名数据,针对其CNAME记录进行判定。该阶段中,可以使用WHOIS信息或搜索引擎对当前CNAME的归属进行判定。若判定其隶属于某CDN服务商,则将其扩充到CDN特征库中。



\subsection{预期达到的目标和已取得的研究成果}
\subsection{进度安排}
本次进度安排如表\ref{table:jinduanpai}所示:


\begin{table}[htbp]
	\centering
	\caption{进度安排}\label{table:jinduanpai}
	\vspace{0.5em}\wuhao
	\begin{tabularx}{1\textwidth}{ccc}
		\toprule
		起始日期        & 截止日期     & 进度安排    \\
		\midrule
		2022年05月           & 2022年07月           & 阅读文献、调研、数据验证      \\
		2022年07月          & 2022年08月  & 开题报告撰写,算法构建  \\
		2022年08月         & 2022年12月   & 算法实现  \\
		2022年12月           & 2023年03月  & 算法验证与优化 \\
		2023年03月          & 2023年05月 &      撰写毕业论文     \\


		\bottomrule
	\end{tabularx}
\end{table}


\FloatBarrier
% Local Variables:
% TeX-master: "../report"
% TeX-engine: xetex
% End: