% !Mode:: "TeX:UTF-8"

\section{国内外研究现状及分析}

 本节对国内外与CDN服务商识别、CDN服务质量测绘以及CDN放置优化策略相关的研究进行介绍。

% !Mode:: "TeX:UTF-8"

\subsection{CDN服务商识别研究现状}

CDN研究的首要步骤为CDN服务商识别。目前,学术界和工业界对CDN厂商的识别可概括为下述三种方式。
 
(1) \textbf{关键字匹配}。许多研究\cite{Huang2008,Adhikari2014,Guo2018}使用规范主机名(Canonical Name,CNAME)关键字作为基础匹配机制,对域名的CNAME记录与CDN服务商提供的特征后缀匹配。其次,根据HTTP错误日志\cite{Huang2008,Guo2018}也可以辅助识别,根据响应报文检验,判定其是否包含CDN服务商的关键字。也有利用域名或IP的WHOIS\cite{Adhikari2014}信息、指针记录\cite{Chen2019}(Pointer record,PTR)与CDN服务商关键字进行匹配。或是根据域名解析的IP地址,是否归属于CDN服务商公开的IP范围进行识别\cite{Choffnes2017}。特别的,猜测CDN边缘服务器的命名规则以构建主机名,并进行DNS探测以获得更多的节点IP\cite{Hohlfeld2018,Timm2018},也是一种解决方案。

(2) \textbf{机器学习融合方法}。CDN识别与机器学习等方法融合也成为一种趋势。国家互联网应急中心\cite{Li2020}(CNCERT)的研究表明,使用CDN的域名,其在IP和TTL的特征有明显区分度,且样本识别精度可以达到90\%以上。Ma\cite{Ma2021}提出了基于半监督图神经网络(GNN)模型的CDN节点发现,将域名与IP二部图关联进一步细化,降低误判。Chen\cite{Chen2019}提出基于长短期记忆网络(LSTM)的Fast-flux和CDN域名识别方法(  Fast-flux  and  CDN  Domains  Recognize,FCDR),将域名字符、经验信息、地理和时间相关特征作为输入,对域名进行CDN服务判定。Hou\cite{Hou2021}提出了基于朴素贝叶斯树(Naive Bayes Tree,NB-Tree)\cite{Liang2006},结合CNAME、WHOIS、HTTP响应头的统计学习方法\cite{Hou2021}。国防科技大学的闫志豪\cite{yan-2022}等人也提出了一种基于域名系统知识图谱的CDN域名识别技术,使用域名的MX、NS、CNAME、IP、地理位置等信息综合识别。 

(3) \textbf{互联网开源工具或公开服务接口}。如CDN Finder\footnote{https://www.cdnplanet.com/tools/cdnfinder/}、CDN云观测\footnote{https://cdn.chinaz.com/}
或findCDN\footnote{https://github.com/cisagov/findcdn}等工具。其中,CDN Finder使用页面资源列表来识别每个唯一主机名的CDN,通过获取页面资源并检查响应标头、执行DNS解析并检查完整的CNAME链获取待判定特征,根据
其维护的Header-to-CDN和CNAME-to-CDN列表来匹配。若匹配失败,则根据其服务IP所属AS来匹配CDN服务商。findCDN则是通过HTTPS服务器响应头、CNAME记录结合IP WHOIS信息来匹配CDN服务商。 


% Local Variables:
% TeX-master: "../section_2_研究现状"
% TeX-engine: xetex
% End: 
% !Mode:: "TeX:UTF-8"

\subsection{CDN服务质量测绘研究现状}
 
内容分发网络在建立之初就肩负着缓解源站压力,提高用户体验的使命。CDN的时延特征对于终端用户而言比较敏感。Johnson是较早研究CDN服务质量测量\cite{Johnson-2001-cdn-measure}之一,并指出CDN的价值并非提供最佳的分发点,而是避免给出性能明显较差的。Pathan认为,CDN性能\cite{Pathan-survey-2007}需要关注五个衡量指标:缓存命中率、预留带宽、代理服务器利用率、时延以及可靠性。Huang的研究指出\cite{Huang2008},CDN主要存在两方面时延:DNS解析时延,也即CDN内部DNS系统向终端用户提供“最佳”CDN边缘节点的时间;内容服务器时延,即终端用户和所选CDN服务器之间的往返时间。国内CDN服务商也针对CDN性能衡量提出通用指标,如阿里云CDN\footnote{https://help.aliyun.com/document\_detail/140425.html}和腾讯云\footnote{https://cloud.tencent.com/document/product/228/1198}认为CDN的时延性能、命中率以及丢包率较为重要。

网络时延测量方面的研究主要有%方面的研究主要体现在测量方式%和可靠性保障两方面。
北京航空航天大学的张兴军\cite{2005内容分发网络性能测量方法研究与实现}等人借助可定制的网络测量基础设施构建了CDN测量平台。吴金福\cite{吴金福2014中国大陆}借助HTTP代理对CDN进行测量。Krishna等人提出了King\cite{King-2002}方法来测量端到端的估计时延,Huang\cite{Huang2008}在对CDN进行测绘时采取的增强King方法便由此改进而来。Zhang\cite{Zhang-2021-Scale-platform}基于DNS欺骗原理,提出了一种轻量化统计延迟测量平台,使用两个DNS服务器之间的时延代表两个主机之间的端到端时延。%北京邮电大学康梦晓\cite{康梦晓2014IP}利用PLE算法进行测量,并在分布参数估计部分进行改进,减少了时延量化间隔数,降低PLE算法的复杂度。胡治国\cite{胡治国2017IP}等人指出,当前测量难点主要体现在单向时延测量,由时钟频差和时钟偏差的相互作用以及非对称路径下的时延测量两方面阻碍。


% Local Variables:
% TeX-master: "../section_2_研究现状"
% TeX-engine: xetex
% End:
% !Mode:: "TeX:UTF-8"
 
\subsection{CDN放置优化策略研究现状}

CDN放置策略可以从节点放置,节点内容缓存\cite{Sahoo2016}两大方面进行研究。前者是确定新位置来放置边缘服务器,内容缓存方面,则是考虑一组内容对象与已部署边缘服务器的关系,称为缓存/复制策略,期望达到服务质量约束(QoS,Quality of Service)下,使成本最小\cite{salahuddin2017survey,Pathan-survey-2007,Tang-2018}。
 
缓存方面,
清华大学的葛志诚\cite{葛志诚2018一种移动内容分发网络的分层协同缓存机制}等人由移动CDN缓存入手,针对资源缓存策略以及资源请求路径,研究多层缓存协调配合问题研究,基于贪心思想提出了一种启发式分层协作缓存策略。
Fu\cite{fu2018}则针对任播CDN入手,通过使用SDN结合NFV(网络功能虚拟化)控制细粒度流量重定向,使依赖BGP路由的任播CDN可根据实际情况重定向,通过控制路径来提高CDN的服务质量。
Liu\cite{liu2018}重点关注移动CDN基站的协作问题,使用随机优化模型对网络稳定约束下的长时间平均传输成本进行优化。有效决定了内容放置和请求重定向问题,在拥塞避免和传输成本方面拥有高性能。此外,也有少数研究从结构方面入手,进行CDN服务的优化,如
Fayed使用可编程套接字sk\_lookup,使IP与主机资源解绑\cite{Fayed-IpUnbind-2021},实现一种更灵活的CDN服务,节约IP资源。
Li\cite{Li-2019-AI-attention-CDN}通过构建一种通用AI注意力网络对CDN服务的性能进行预测。
   
节点放置方面, 
Huang\cite{Huang2008}在对CDN进行测绘时,发现CDN分布的两种设计理念,一种是深入ISP的设计,其更接近终端用户,在时延和吞吐量方便有较大优化。但集群较为分散,管理困难。另一种是ISP入户设计,即只在几个关键点建立大型内容分发中心,并使用高速链路互联,与第一种相比,在维护和管理费用方面有所降低,但时延性能有一定损失。
Zhu\cite{zhu-2021}指出CDN设置逐渐边缘化,靠近终端用户。针对优酷部署的路由宝入户等现象进行分析,发现其智能路由器可配置多达1TB的存储空间,起到辅助CDN进行内容分发的功能。并研究一种边缘网络视频内容分发的新架构,使智能路由器拥有轻量级的内容分发功能,达到CDN下沉的目的。
Sahoo\cite{Sahoo2016}指出,传统CDN的边缘服务器放置是NP-Hard问题。其问题定义如下:给定一组服务器侯选位置,一组用户位置,边缘服务器放置涉及从候选集中找到边缘服务器的最佳数量和位置,使得每个最终用户分配到其中一个服务器,且CDN服务商的成本最小化。同理,基于云的CDN使用基于租用的虚拟化资源建立边缘服务器,需要结合云提供商考虑,基于NFV的CDN中,服务器放置需要VNF(虚拟化网络功能)实现的CDN节点,称为VNF放置问题。基于虚拟化程度分类,传统CDN放置依赖于离线算法实现,基于云和VNF的放置则支持在线算法。
Benkacem\cite{benkacem2018optimal}等人介绍了一种CDN as a Service平台,支持跨云平台对基于VNF的CDN进行动态部署和管理。相应的,使用两个整数线性规划,分别用于降低成本和提高服务质量。并使用博弈论提供权衡方案。
Tang\cite{Tang-2018}等人在部署优化方面,额外考虑了动态用户流量,并将流量分发成本降维至整数线性规划(ILP)问题。随后,通过贪心算法来解决ILP问题的可扩展形式。


% Local Variables:
% TeX-master: "../section_2_研究现状"
% TeX-engine: xetex
% End:
 

\subsection{国内外研究现状分析}
由上文可知,研究人员在CDN服务商识别、CDN服务质量测绘,CDN放置优化策略方面展开了研究,并提出了一些有效的理论和方法,然而仍存在下述问题。

(1) 针对CDN服务商识别,不同研究达到的识别精度与探测开销存在矛盾。CDN服务识别在基于匹配规则时有良好表现,使用CNAME特征时,精度通常在85\%以上。在使用多种匹配规则进行联合识别时,识别精度略有提高。在此基础上,引入机器学习和深度学习,同时增加多种信息源,如地理位置、生存时间值(Time To Live,TTL)特征、IP特征,其识别精度进一步提高至90\%以上,但已产生边际效应,信息源再扩充对精度提高收益显著减少。从探测开销而言,基于主被动测量DNS记录的方法需要发送大量DNS报文,对实网环境存在污染情况。基于Windows Live搜索日志,从主机名中提取CNAME记录方式需要大量用户提供信息,使用户隐私方面存在风险。WHOIS信息由于服务商限制,获取时间开销较大,集成地理、时间等信息的识别方法,在获得高精度的同时也极大增加了数据探测成本。 



(2) 当前研究对CDN服务质量测量存在误差。CDN服务质量性能测量常用指标如图 \ref{fig:CDN服务质量性能指标}所示。其中缓存命中率主要受CDN配置的缓存策略影响。预留带宽指为源站预留的带宽,通常以字节为单位。代理服务器利用率指服务器忙时占比,管理员可以根据这个指标进一步计算CPU负载、服务请求数和存储I/O的使用情况等。时延,指用户感知的响应时间,根据访问流程可大致拆分为DNS时间、TCP时间和SSL时间。可靠性通常由丢包率指代,高可靠表明丢包率低,且始终对用户可用。
  
\begin{figure}[ht]
	\centering
	\begin{tikzpicture}[grow cyclic, text width=2.7cm, align=flush center,
    level 1/.style={level distance=3cm,sibling angle=60},
    level 2/.style={level distance=2.5cm,sibling angle=60}]
  
    \node{CDN性能指标}
    child { node {缓存命中率}  }
    child { node {预留带宽} }
    child { node {时延} 
      child { node {SSL时间}}
      child { node {TCP时间}}
      child { node {DNS解析时间}}
    }
    child { node {代理服务器利用率} }
    child { node {可靠性}  };
  \end{tikzpicture}

	\caption{CDN服务质量性能指标}
	\label{fig:CDN服务质量性能指标}
\end{figure}
\FloatBarrier

对于CDN外部测量而言,许多研究关注时延特征。其中,DNS解析时延主要依赖于用户、递归DNS服务器、CDN DNS服务器三者的链路时延,SSL时延的主要影响因素同理。因此,研究多由TCP时间入手,等价的,TCP时间通常与IP链路往返时延(Round Trip Time,RTT)正相关。使用HTTP下载时间对CDN边缘服务器节点进行性能测量,对CDN就近解析条件考虑不足。使用测量网络进行探测,依赖大规模测量平台支持。借助HTTP代理的测量方式较为符合中国大陆的互联网情况,但其重点面向基于HTTP跳转的CDN请求路由实现方式,对DNS解析路由涉及较少。
具体的,针对RTT测量方面,目前常用King及其增强方法。King方法利用随机域名,容易造成缓存污染问题。基于DNS欺骗进行时延测量的方式,需要搜索两组主机临近的DNS服务器,测量其时延作为近似值,对于CDN低时延的本质要求而言,测量结果误差过大。

(3)针对CDN放置优化策略研究,许多研究主要以QoS和成本作为优化问题的目标,针对缓存内容或边缘服务器放置进行单方面优化。其中,针对缓存的优化着重考虑了缓存层级扩展、缓存路径选择、缓存替换策略等。对于边缘服务器放置主要使用整数线性规划对图模型进行最优化处理,得到一组候选位置的最优子集。由于缓存算法依托于边缘服务器内容放置,但并非完全相关,单方面考虑优化不能保证得到全局最优结果。同时,随着虚拟化技术(如云、SDN、NFV等技术)的发展,动态部署CDN边缘服务器作为一种新兴技术,使得边缘服务器部署可动态调整,针对服务器放置和路径选择问题进行联合优化成为可能。

 
因此,为对CDN服务的QoS进行测绘并权衡成本及服务质量,
如何对CDN服务商进行高效、准确的发现与识别,同时减少探测流量对实网环境的影响;
如何保证对CDN时延特征的可靠、完整测量;
如何对CDN放置策略进行优化,统筹考虑边缘服务器放置和缓存内容的关系,
是当前国家保障互联网稳定运行所需要解决的问题,也是本文的研究目标。
为此,本文致力于研究使用较少信息源(CNAME特征、HTTP特征),对CDN服务商进行高效识别和发现;并设计实现一种基于分布式代理技术的CDN时延特征测量平台,用于测量CDN边缘服务器的时延特征,在时间、空间分布中的服务质量情况,并给出QoS分析方法,评估CDN的部署效果;针对CDN部署情况的不足之处,研究一种权衡CDN部署成本和QoS的多目标优化方案。


% Local Variables:
% TeX-master: "../report"
% TeX-engine: xetex
% End: