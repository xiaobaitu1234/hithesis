% !Mode:: "TeX:UTF-8"
%%%%%%%%%%%%%%%%%%%%%%%%%%%%%%%%%%%%%%%%%%%%%%%%%%%%%%%%%%%%%%%%%%%%%%%%%%%%%%%%
%          ,
%      /\^/`\
%     | \/   |                CONGRATULATIONS!
%     | |    |             SPRING IS IN THE AIR!
%     \ \    /                                                _ _
%      '\\//'                                               _{ ' }_
%        ||                     hithesis v3                { `.!.` }
%        ||                                                ',_/Y\_,'
%        ||  ,                   dustincys                   {_,_}
%    |\  ||  |\          Email: yanshuoc@gmail.com             |
%    | | ||  | |            https://yanshuo.name             (\|  /)
%    | | || / /                                               \| //
%    \ \||/ /       https://github.com/dustincys/hithesis      |//
%      `\\//`   \\   \./    \\ /     //    \\./   \\   //   \\ |/ /
%     ^^^^^^^^^^^^^^^^^^^^^^^^^^^^^^^^^^^^^^^^^^^^^^^^^^^^^^^^^^^^^^
%%%%%%%%%%%%%%%%%%%%%%%%%%%%%%%%%%%%%%%%%%%%%%%%%%%%%%%%%%%%%%%%%%%%%%%%%%%%%%%%
\documentclass[fontset=fandol,type=master,campus=harbin, tocblank=false]{hithesisbook} 
% 此处选项中不要有空格
%%%%%%%%%%%%%%%%%%%%%%%%%%%%%%%%%%%%%%%%%%%%%%%%%%%%%%%%%%%%%%%%%%%%%%%%%%%%%%%%
% 必填选项
% type=doctor|master|bachelor|postdoc
%%%%%%%%%%%%%%%%%%%%%%%%%%%%%%%%%%%%%%%%%%%%%%%%%%%%%%%%%%%%%%%%%%%%%%%%%%%%%%%%
% 选填选项(选填选项的缺省值已经尽可能满足了大多数需求,除非明确知道自己有什么
% 需求)
% campus=shenzhen|weihai|harbin
%   含义:校区选项,默认harbin
% glue=true|false
%   含义:由于我工规范中要求字体行距在一个闭区间内,这个选项为true表示tex自
%   动选择,为false表示区间内一个最接近版心要求行数的要求的默认值,缺省值为
%   false。
% tocfour=true|false
%   含义:是否添加第四级目录,只对本科文科个别要求四级目录有效,缺省值为
%   false
% fontset=windows|mac|ubuntu|fandol|adobe
%   含义:设置字体,若不指定会自动识别系统,然后设置字体。fandol是开源字体,自行
%   下载安装后设置使用。windows是中易字库,窝工默认常用字体,绝对没毛病。mac和
%   ubuntu 默认分别是华文和思源字库,理论上用什么字库都行。后两种字库的安装方法
%   到谷歌上百度一下什么都有了。Linux非ubuntu发行版、非x86架构机器等如何运行可到
%   github issue上讨论。
% tocblank=true|false
%   含义:目录中第一章之前,是否加一行空白。缺省值为true。
% chapterhang=true|false
%   含义:目录的章标题是否悬挂居中,规范中要求章标题少于15字,所以这个选项
%   有无没什么用,除了特殊需求。缺省值为true。
% fulltime=true|false
%   含义:是否全日制,缺省值为true。非全日制如同等学力等,要在cover中设置类
%   型,封面中不同格式
% subtitle=true|false
%   含义:论文题目是否含有副标题,缺省值为false,如果有要在cover中设置副标
%   题内容,封面中显示。
% newgeometry=one|two|no
%   含义:规范中的自相矛盾之处,版芯是否包含页眉页脚,旧方法是按照包含页眉
%   页脚来设置。该选项是多选选项,如果设置为no,则版新为旧模板的版芯设置方法,
%   如果设置该选项one或two,分别对应两种页眉页码对应版芯线的相对位置。第一种
%   是严格按照规范要求,难看。第二种微调了页眉页码位置,好一点。默认two。
% debug=true|false
%   含义:是否显示版芯框和行号,用来调试。默认否。
% openright=true|false
%   含义:博士论文是否要求章节首页必须在奇数页,此选项不在规范要求中,按个
%   人喜好自行决定。 默认否。注意,窝工的默认情况是打印版博士论文要求右翻页
%   ,电子版要求非右翻页且无空白页。如果想DIY(或身不由己DIY)在什么地方右
%   翻页,将这个选项设置为false,然后在目标位置添加`\cleardoublepage`命令即
%   可。
% library=true|false
%   含义:是否为提交到图书馆的电子版。默认否。注意:如果设置成true,那么
%   openright选项将被强制转换为false。
% capcenterlast=true|false
%   含义:图题、表题最后一行是否居中对齐(我工规范要求居中,但不要求居中对
%   齐),此选项不在规范要求中,按个人喜好自行决定。默认否。
% subcapcenterlast=true|false
%   含义:子图图题最后一行是否居中对齐(我工规范要求居中,但不要求居中对齐
%   ),此选项不在规范要求中,按个人喜好自行决定。默认否。
% absupper=true|false
%   含义:中文目录中的英文摘要在中文目录中的大小写样式歧义,在规范中要求首
%   字母大写,在work样例中是全大写。该选项控制是否全大写。默认否。
% bsmainpagenumberline=true|false
%   含义:由于本科生论文官方模板的页码和页眉格式混乱,提供这个选项自定义设
%   置是否在正文中显示页码横线,默认显示。
% bsfrontpagenumberline=true|false
%   含义:由于本科生论文官方模板的页码和页眉格式混乱,提供这个选项自定义设
%   置是否在前文中显示页码横线,默认显示。
% bsheadrule=true|false
%   含义:由于本科生论文官方模板的页码和页眉格式混乱,提供这个选项自定义设
%   置是否显示页眉横线,默认显示。
% splitbibitem=true|false
%   含义:参考文献每一个条目内能不能断页,应广大刀客要求添加。默认否。
% newtxmath=true|false
%   含义:数学字体是否使用新罗马。默认是。
% chapterbold=true|false
%   含义:本科生章标题在目录和正文中是否加粗
% engtoc=true|false
%   含义:非博士生需要添加英文目录的,手动添加,如果是博士,此开关无效
% zijv=word|regu
%   含义:字距设置为规范规定33个字还是word中34个字。默认regu。
%%%%%%%%%%%%%%%%%%%%%%%%%%%%%%%%%%%%%%%%%%%%%%%%%%%%%%%%%%%%%%%%%%%%%%%%%%%%%%%%
\usepackage{hithesis}

\graphicspath{{figures/}}

\begin{document} 
\frontmatter
% !Mode:: "TeX:UTF-8"

\hitsetup{
  %******************************
  % 注意:
  %   1. 配置里面不要出现空行
  %   2. 不需要的配置信息可以删除
  %******************************
  %
  %=====
  % 秘级
  %=====
  statesecrets={公开},
  natclassifiedindex={TM301.2},
  intclassifiedindex={62-5},
  %
  %=========
  % 中文信息
  %=========
  ctitleone={面向DNS全局负载均衡的CDN服务质量测绘},%本科生封面使用
  ctitletwo={面向DNS全局负载均衡的CDN服务质量测绘},%本科生封面使用
  ctitlecover={面向DNS全局负载均衡的CDN服务质量测绘},%放在封面中使用,自由断行
  ctitle={面向DNS全局负载均衡的CDN服务质量测绘},%放在原创性声明中使用
%  csubtitle={一条副标题}, %一般情况没有,可以注释掉
  cxueke={工学},
  csubject={计算机科学与技术},
  caffil={计算机科学与技术学院},
  cauthor={门浩},
  csupervisor={张兆心教授},
%  cassosupervisor={某某某教授}, % 副指导老师
%  ccosupervisor={某某某教授}, % 联合指导老师
  % 日期自动使用当前时间,若需指定按如下方式修改:
  cdate={2023年6月},
  cstudentid={9527},
  cstudenttype={学术学位论文}, %非全日制教育申请学位者
  cnumber={no9527}, %编号
  cpositionname={哈铁西站}, %博士后站名称
  cfinishdate={20XX年X月---20XX年X月}, %到站日期
  csubmitdate={20XX年X月}, %出站日期
  cstartdate={3050年9月10日}, %到站日期
  cenddate={3090年10月10日}, %出站日期
  %(同等学力人员)、(工程硕士)、(工商管理硕士)、
  %(高级管理人员工商管理硕士)、(公共管理硕士)、(中职教师)、(高校教师)等
  %
  %
  %=========
  % 英文信息
  %=========
  etitle={Research on CDN Service Quality Mapping for GSLB of DNS}, % Global Server Load Balance
  esubtitle={This is the sub title},
  exueke={Engineering},
  esubject={Computer Science and Technology},
  eaffil={\emultiline[t]{School of Computer Science and \\Technology}},
  eauthor={Men Hao},
  esupervisor={Prof. Zhang Zhaoxin},
%  eassosupervisor={XXX},
  % 日期自动生成,若需指定按如下方式修改:
  edate={June, 2023},
  estudenttype={Master of Art},
  %
  % 关键词用“英文逗号”分割
  ckeywords={\TeX, \LaTeX, CJK, 门浩},
  ekeywords={\TeX, \LaTeX, CJK, menhao},
}

\begin{cabstract}

摘要暂时没有确定
\end{cabstract}

\begin{eabstract}
   zhaiyao zanshi weiyou queding.
\end{eabstract}
 % 封面
\makecover 
%\begin{denotation}
\begin{table}[h]%此处最好是h
\caption{国际单位制中具有专门名称的导出单位}
\vspace{0.5em}\centering\wuhao
\begin{tabular}{ccccc}
\toprule
量的名称&单位名称&单位符号&其它表示实例\\
\midrule
频率&赫[兹]&Hz&s-1\\
\bottomrule
\end{tabular}
\end{table}
\end{denotation}
%物理量名称表,符合规范为主,有要求添加
\tableofcontents %目录
\mainmatter 
% !Mode:: "TeX:UTF-8"

\chapter[绪论]{绪论}[xulun]
本章首先给出支撑研究的课题来源;其次介绍课题背景及研究的目的和意义,并分析研究当前国内外研究现状,指出目前存在的主要问题,最后介绍了本文的研究内容及组织结构。

\section{课题来源}[ketilaiyuan]

本课题来源于国家重点研发计划“XXX?????研究”(??????)及XXX项目。

\section{课题背景及研究的目的和意义}[ketibeijingjiyanjiudemudiheyiyi]



%%%%%%%%%%%%%%%%%%%%%%%%%%%%%%%%%%%%%
%%% 
%%% 
%%% 
%%% 引入CDN概念
%%% 
%%% 
%%% 
%%%%%%%%%%%%%%%%%%%%%%%%%%%%%%%%%%%%%


CDN,全称内容分发网络(Content Distribute Network),也有译作内容交付网络(Content Delivery Network)。顾名思义,CDN完成将内容从源站传递到终端用户的任务。

1991年之后的近十年间,公众互联网接入方式以拨号为主,接入带宽低且网民规模较小,互联网的瓶颈在于用户接入带宽,对提供内容的服务器和骨干传输网络的压力较小。随着互联技术的发展和网民数量增加,内容服务器和骨干网络压力随之增大,互联网瓶颈向内容服务器和骨干网络部分转移。

在CDN形成商用能力前,网络运营者们通过包括但不限于扩展技术Scale up/Scale out,镜像技术Mirror,缓存技术Cache来达到网站加速的目的。这些极大增加了网站运营成本。

1995年,麻省理工学院教授Tim Berners-Lee发起的一项技术挑战,旨在提出一种实现互联网内容的无拥塞分发方法,该学术难题最终催生出一种革新性的互联网服务,CDN。

Akamai公司通过智能化的互联网分发,结束了“World Wide Wait”(世界一起等待)的尴尬局面,并于1999年开始提供商业服务。随后三年间,全球互联网发展迎来了高潮期,2001年发布了RFC3040\cite{RFC-3040}描述了许多组件技术用于构建 CDN,完成了CDN的初始构建阶段。2012年发布了RFC6707\cite{RFC-6707}用于描述CDN互联问题。自此,CDN的发展进入了高速阶段。 


一个典型的CDN执行流程如图\ref{fig:join_cdn_classic}所示(根据CDN技术详解\cite{leibaohua-CDN-2012}及CDN技术调研\cite{Pathan-survey-2007})


\begin{figure}[h]
	\centering
	\includegraphics{cdn_abstract.pdf}
	\caption{引入CDN后的典型用户访问流程}
	\label{fig:join_cdn_classic}
\end{figure}




%%%%%%%%%%%%%%%%%%%%%%%%%%%%%%%%%%%%%
%%% 
%%%  
%%% 
%%% 这里过渡一下
%%% 
%%% 
%%% 
%%%%%%%%%%%%%%%%%%%%%%%%%%%%%%%%%%%%%

随着互联网技术的发展与普及,以及更好应对新冠疫情,混合办公模式逐渐得到部分企业与员工的认可。在线办公技术持续演进,实现形式与办公理念也在发生变革。
在线办公技术的进步离不开基础技术服务的加速跟进。云计算、互联网数据中心(Internet Data Center,IDC)、内容分发网络(Content Delivery Network,CDN)等基础技术服务的发展支撑了在线办公的发展。

以 CDN 为例,根据2022年中国互联网络信息中心(CNNIC)发布的《中国互联网络发展状况统计报告》\cite{cnnic2022} (以下简称《报告》)显示:
在企业数量方面,2020 年,取得内容分发网络牌照的增值电信业务企业数量为 44 家,2021 年前 11 个月已新增 52 家,预计未来仍会保持较快增长;
在技术方面,多家运营商提出 SD-WAN 解决方案,通过优化传输技术,解决企业邮箱、视频会议等系统的加速难题,提升用户体验。

%%%%%%%%%%%%%%%%%%%%%%%%%%%%%%%%%%%%%
%%% 
%%%  
%%% 
%%% 关注政策
%%% 
%%% 
%%% 
%%%%%%%%%%%%%%%%%%%%%%%%%%%%%%%%%%%%%

对于国家层面,
2016年我国《网络安全法》正式通过,“关键信息基础设施安全保护制度”被首次提出,
2019至2021年,《关键信息基础设施安全保护条例》连续三年纳入国家立法计划,历经多年反复锤炼,于2021年8月17日正式发布,并于2021年9月1日起施行。
可见近年来,国家对关键信息基础设施安全的重视程度。CDN作为其中一项关键基础设施,*****************************************************
对关键信息基础设施保护系列制度要素作了具体规定,涵盖总则、关键信息基础设施认定、运营者责任义务、保障和促进、法律责任等诸多方面。


%%%%%%%%%%%%%%%%%%%%%%%%%%%%%%%%%%%%%
%%% 
%%%  
%%% 
%%% 关注安全、服务、价格优势
%%% 
%%% 
%%% 
%%%%%%%%%%%%%%%%%%%%%%%%%%%%%%%%%%%%%

对于行业层面,使用CDN服务的网络内容服务商商(ICP),*************

%%%%%%%%%%%%%%%%%%%%%%%%%%%%%%%%%%%%%
%%% 
%%%  
%%% 
%%% 关注性能
%%% 
%%% 
%%% 
%%%%%%%%%%%%%%%%%%%%%%%%%%%%%%%%%%%%%

对于用户层面,CDN作为一种网络加速产品,广泛应用于图片及小文件下载、大文件下载以及音视频点播等业务场景。
《报告》显示,截至 2021 年 12 月,我国网络视频(含短视频)用户规模达 9.75 亿,较 2020年 12 月增长 4794 万,占网民整体的 94.5\%。
网络视频,尤其是热门网络视频的高并发访问或者访问突增场景下对源站性能要求非常高,且源站的带宽成本也较高。
合理选择一款,或多款CDN服务就显得很有必要。

因此,多方面都需要网络基础设施,尤其时CDN服务的服务质量现状。**************本文将给出一种方法。



%%%%%%%%%%%%%%%%%%%%%%%%%%%%%%%%%%%%%
%%% 
%%% 
%%% 
%%% 阐述为什么是面向DNS全局负载均衡
%%% 
%%% 
%%% 
%%%%%%%%%%%%%%%%%%%%%%%%%%%%%%%%%%%%% 

本文从CDN服务的基本原理出发,收集国内热点域名使用的CDN特征串,构建当前CDN使用量排名。
并根据DNS解析效率、IP时延特征对CDN的服务质量进行测绘,试图给出CDN服务质量排名,为选择CDN服务提供依据和方法支撑,最终达到终端用户获得较优体验的目的。

目前,工业界对CDN全局负载均衡实现方式可概括为下述三类方式。


\begin{enumerate}[label={(\arabic*)}]


	\item 基于DNS全局负载均衡。使用CDN服务的域名通过包括CNAME等技术\cite{Choffnes2017}将自身的解析权指定给CDN服务商提供的智能DNS解析器,由于DNS系统的简单性及其作为目录系统的普遍性\cite{Pathan-survey-2007}等优点,当终端用户向本地递归域名解析器(LDNS)或公共域名解析器(ODNS)请求时,该请求将重定向到CDN服务商进行智能解析,根据终端用户的位置、链路情况等因素分配一个或多个IP,达到负载均衡的目的。
	\item 基于任播(Anycast)IP。Anycast考虑地理(或称网络距离)最近原则,其依赖于BGP路由实现,将同一IP分布在多个AS中,为用户提供相同的服务。当终端用户向Anycast IP请求时,将优先选择BGP路径较短的IP提供服务,达到负载均衡的效果。实际使用中,其负载均衡并未使用全局信息\cite{Calder2015},不能很好应对网络状态的变化\cite{Choffnes2017},因此存在约20\%比例将用户导向次优节点。特别的,中国AS呈现出少而大的特征,AS间的BGP路由配置通常是静态的\cite{Choffnes2017},使用任播实现效果较差。
    \item 基于HTTP重定向等其他方式。HTTP重定向基于状态码3XX或manifest文件重定向\cite{Adhikari2014}等方式,这些方法额外增加了RTT时间,常用于视频分发等领域。不适合一些时延敏感服务,如搜索等。


\end{enumerate}

综上所述,本文主要面向主流\cite{Hao2018}的基于DNS全局负载均衡的CDN服务,根据其实现方法进行服务商识别,服务质量测量与评估,********。
对网络内容服务商提供一种比较依据,个人开发人员提供一种选择,国家进行CDN服务的运行管理具有重要的理论意义和实用价值。






\section{国内外研究现状}[guoneiwaiyanjiuxianzhuang]



%%%%%%%%%%%%%%%%%%%%%%%%%%%%%%%%%%%%%
%%% 
%%% 
%%% 
%%% CDN识别方案
%%% 
%%%  
%%% 
%%%%%%%%%%%%%%%%%%%%%%%%%%%%%%%%%%%%% 
\subsection{CDN厂商识别研究现状}[CDNchangshangshibieyanjiuxianzhuang]
CDN一直伴随这快速分发流量、防御DDos攻击等优点。当前也有着融合CDN产品服务,旨在使用多家厂商提供的CDN服务\cite{Zhu2021},做到相互补充,尽可能达到100\%服务覆盖率等目的。同时,与CDN有关的安全漏洞、攻击防范等问题也有较多关注。如使用CDN作为DDos攻击手段\cite{Guo2020}来瘫痪源站、利用CDN的新型审查规避技术\cite{Wei2021}为违规站点提供保护、未验证源服务器证书\cite{SHOBIRI2021}等危害。

其中,发现CDN攻击事件、漏洞检测及修复、融合CDN选择、CDN服务排名等问题,都离不开CDN厂商的识别。目前,学术界和工业界对CDN厂商的识别可概括为下述三大类方式。


\begin{enumerate}[label={(\arabic*)}]


	\item 关键字匹配。如使用CNAME关键字匹配\cite{Adhikari2014}、根据HTTP错误\cite{Huang2008,Guo2018}的判定、或利用WHOIS\cite{Huang2008}、PTR\cite{Chen2019}信息判定,也可以利用公开IP段匹配CDN\cite{Choffnes2017},更有猜测CDN边缘服务器的命名规则以构建主机名,并进行DNS探测以获得更多的节点IP\cite{Hohlfeld2018,Timm2018}的解决方案。
	\item 机器学习方法。同时,CDN识别与机器学习等方法融合也成为一种趋势。
如利用DNS信息与IP特征的统计识别\cite{Li2020}、基于图模型的CDN节点发现\cite{Ma2021}、基于深度学习的检测方法\cite{Chen2019}以及基于Naive Bayes Tree\cite{Liang2006},结合CNAME、WHOIS、HTTP响应头的统计学习方法\cite{Hou2021}。
	\item 使用互联网开源工具或公开服务接口。如CDN Finder\footnote{https://www.cdnplanet.com/tools/cdnfinder/}、CDN云观测\footnote{https://cdn.chinaz.com/}
	或findCDN\footnote{https://github.com/cisagov/findcdn}等工具。其中,CDN Finder使用页面资源列表来识别每个唯一主机名的CDN,通过获取页面资源并检查响应标头、执行DNS解析并检查完整的CNAME链获取待判定特征,根据
	其维护的Header-to-CDN和CNAME-to-CDN列表来匹配。若匹配失败,则根据其服务IP所属AS来匹配CDN服务商。findCDN则是通过HTTPS服务器响应头、CNAME记录结合IP WHOIS信息来匹配CDN服务商。
\end{enumerate}

综上所述,当前CDN服务识别*********************************************

%%%%%%%%%%%%%%%%%%%%%%%%%%%%%%%%%%%%%
%%% 
%%% 
%%% 
%%% CDN测量方案
%%% 
%%% 
%%% 
%%%%%%%%%%%%%%%%%%%%%%%%%%%%%%%%%%%%% 
\subsection{CDN服11务质量测量方面}[CDNfuwuzhiliangceliangfangmian]
说明当前主要存在的测量手段,测量内容,以及效果等。


研究\cite{Pathan-survey-2007}表明,CDN的业务目标至少包括可扩展性、安全性、可靠性、响应性和性能。
当前,学术界对CDN的研究多种多样,
有对CDN边缘服务器放置算法的研究\cite{Sahoo2016},同时,更有学者大胆提出将IP与主机名解绑\cite{Fayed-IpUnbind-2021},实现一种更灵活的CDN服务

Johnson是最早研究CDN服务质量测量\cite{Johnson-2001-cdn-measure}之一,并指出CDN的价值并非提供最佳的分发点,而是避免给出性能明显不行的。

根据以往的研究\cite{Huang2008},CDN主要存在两方面时延:DNS解析时延,也即CDN内部DNS系统向终端用户提供“最佳”CDN边缘节点地址的时间;以及内容服务器时延,及终端用户和所选CDN服务器之间的往返时间。Krishna等人提出了King\cite{King-2002}方法来测量端到端估计距离,但由于其存在缓存污染、流量开销较大问题,后续有人提出了一种轻量化测量平台\cite{Zhang-2021-Scale-platform},总之这里要给出一个测量方法。

同时,根据阿里云\footnote{https://help.aliyun.com/document\_detail/140425.html}及腾讯云\footnote{https://cloud.tencent.com/document/product/228/1198}公开的CDN性能衡量指标,解析时延和可用性也是两个重要方面。

在DNS解析过程中,需要对LDNS和ODNS进行约减,减少发包,还要对CNAME进行筛选,减少发包。主要看获取到的IP是否量级相同。



\subsection{存在的主要问题}[]

1.CDN识别方面,CNAME、ASN收集方面,需要投入大量前期工作。当前给出的基于机器学习的方法需要预先给出大量原始标注数据集,较难构造。因此,本文打算研究TSVM方法。


给出当前识别方面的问题,服务质量测量方面存在的不足。说明我自己要做一个分类器来识别,做一个方面的测量指标
\section{研究内容及论文组织结构}[yanjiuneirongjilunwenzuzhijiegou]
本小节对课题的研究内容进行阐述,并介绍文章组织结构。研究内容描述了研究对象及相关解决方案,组织结构部分对文章行文总体结构进行概括展示。
\subsection{论文研究内容}[lunwenyanjiuneirong]
\subsection{论文组织结构}[lunwenzuzhijiegou]

% Local Variables:
% TeX-master: "../thesis"
% TeX-engine: xetex
% End:

% !Mode:: "TeX:UTF-8"

\chapter{基于多阶段学习的CDN服务域名识别方法}[jiyuTSVMdeCDNfuwuyumingshibiefangfa]




\section{引言}[yinyan]


\section{相关工作}[xiangguangongzuo]



\section{基于半监督学习的数据集构建方法}[pulearn]

\subsection{域名数据集选择方法}[]
当前较为出色的域名排行有Traco\cite{Pochat2018}排名以及SecRank\cite{Xie2022}排名。其中,Traco融合了多种国际排名,如Alexa、Umbrella、Majestic以及Farsight,具有较好的研究价值。但由于其提供域名多为SLD域名,不适合用来做CNAME检测及HTTP获取。因此,本文选择具有包含多种标签长度域名的SecRank排名作为研究对象。

这里我们考虑使用CNAME的二级域名(SLD)的WHOIS信息进行识别,判定是否为公司CDN入口域名。

于是方法就变成了CNAME特征串判定+CNAME的SLD的WHOIS判定补充+HTTP响应头判定。
\subsection{特征选取}[]

先前研究表明,中国大陆主要以以DNS全局负载均衡系统作为CDN IP选择依据,在这里,我们选取SecRank数据集的www域名以及SLD域名作为研究对象,因为他们大概率是网站,共计筛选出20W个。其次,我们选择来自不同地区的***个LDNS对域名集合进行DNS解析,并对包含解析结果的域名进行HTTP Head命令解析,获取其响应头信息字典。根据上述条件进行CDN代理标注。标注规则如下

\begin{description}
	\item[CDN代理域名] 域名CNAME记录包含CDN服务商关键字
	\item[非CDN代理域名] 域名多地解析结果相同,不含CNAME记录,且HTTP响应过程中不存在类似302、307等跳转状态码。
\end{description}

根据上述条件对样本筛选后,我们保守得判定出***个CDN域名,以及***个非CDN域名。

根据当前数据集,我们对HTTP响应头进行特征提取。

提取后排序如下:

因此,我们选取前10个特征作为特征提取依据。

\subsection{基于PU learn的小样本标注方法}[]

在对特征提取过程中,我们选取过程较为保守,即仅根据一些热点厂商的CNAME特征进行CDN代理域名判定,对使用一些CNAME特征白名单不包含厂商特征的域名,没有完整加入到正例集,且含CNAME记录的域名也存在部分未使用CDN服务。因此,这里采用PU learn方法对数据集中未标注的数据进行判定。



\subsection{基于TSVM的数据集扩充方法}[]
由于本文研究内容面向使用DNS作为全局负载均衡的CDN,其数据可从含CNAME记录的域名中产生。
NB-tree方法\cite{Hou2021}指出,使用CNAME、WHOIS以及HTTP响应头可以较好识别CDN服务商。然而,现网环境下,域名与IP的映射关系存在一定误差\cite{Ma2021}。其方法过于依赖IP的相关信息,IP变量的引入势必会对识别精度产生影响。

本文在NB-tree方法\cite{Hou2021}基础上进行改进,舍弃IP的相关信息。选取识别效果较好的CNAME及其二级域名(SLD)的WHOIS信息,结合HTTP响应头的特征分类进行识别器构造。首先,当前CDN识别数据集在构建时,大多采用监督学习,对标注后的数据进行训练。在实际场景中,根据符合厂商特征的CNAME特征串对HTTP响应头标注为正例准确率较高,但对未明确识别的数据进行类别划分较为困难。因此,本文使用PU learning\cite{Bekker2020}方法(一种从正例和未标注数据集学习的二元分类方法)进行小样本数据集标注工作。得到一批存在正例和负例的样本。随后,使用TSVM\cite{Joachims1999}(一种经典半监督 SVM)方法来构造大样本集,增加分类器的准确率。


\section{基于半监督集成学习的CDN节点识别方法}[pulearn]




算法在使用之前,要测一下数据集精度!!!!!!!,数据集可用之后,还要拿别人的算法对比。


使用决策树结合机器学习方法评估 CDN 对站点返回 HTTP报文的 headers 字典特征。此处可以结合多种方法对比,如SVM,随机森林,朴素贝叶斯等。(其中朴素贝叶斯用于跟侯锐杰的方法对比)

\section{实验结果}[shiyanujieguo]



\subsection{数据集构建精度结果}[]
在这里验证数据集构建方法的精度,如果达到什么条件???可以采纳???



\subsection{CNAME关键字发现方法结果}[]
拿出来看看找到多少家的特征串,说明一下,找找官方公布的数据对比一下,说一下我发现了百分之多少的厂商串



\subsection{CDN域名识别结果}[]
这里给出一个F1分数什么的作为结果验证。

效果评估部分: 在这里,可以先考虑不使用 CNAME,作为极端情况的特征串发现方法。

应用部分:给出一个使用 SecRank 数据集\cite{Xie2022}情况下,解析出CDN 服务的情况,在这里给出《互联网基础设施与软件安全》类似的 CDN 报告图片。(给出厂商数量,厂商使用量,排名等)



\section{本章小结}[benzhangxiaojie]




本章旨在提取CDN服务域名特征,给出一种CDN入口地址特征串发现方法。为后续测量分析部分提供厂商分类支撑。



% Local Variables:
% TeX-master: "../main"
% TeX-engine: xetex
% End:

% !Mode:: "TeX:UTF-8"

\chapter{基于***的D???NS解析效率测量}[Example]


\section{引言}[yinyan]
\section{相关工作}[xiangguangongzuo]

\section{方法}[fangfa]

\section{实验结果}[shiyanujieguo]
\section{本章小结}[benzhangxiaojie]


1.公共DNS选取策略(很多论文都是用LDNS,用运营商提供的递归来判定,由于大约有13\%的用户使用公共DNS\cite{Callejo-2019-Measuring},因此引入部分公共DNS进行测量) 
选取TOP域名测一遍,观察准确率,成功率等(出一个算法)
(研究表明,在开启EDNS解析时,会附带终端用户的部分IP信息。此使,DNS全局负载均衡可以根据这些信息来返回对当前用户更优的选择。由ECS推出的Akamai映射方案为公共DNS服务的用户提供了显著的性能优势\cite{Chen2015}。相反,当终端用户未启用EDNS时,DNS负载均衡时会根据LDNS的地理位置进行智能解析\cite{Hao2018})
同时,根据用户举例LDNS的距离不同,最终结果的效果也不尽人意\cite{pan-2003-dns},目前也有研究提供一种用户选择最近开放DNS服务\cite{Zhang-2021-Scale-platform}方法。因此,在本次测量过程中,以LDNS作为聚类中心分配用户请求,理想化假定用户请求与自己距离最近的LDNS。同时为减少探测点网络性能的差距,我们使用请求时间减去探测点到DNS解析器的时间,减少探测点网络情况对实验结果的影响(考虑是否引用专利、论文~!!!!)。
 
解释方面:CDN入口CNAME域名的TTL比较长,在使用时可以忽略这部分时延。可以找一些CDN域名,向权威CNAME,以及向CDN解析CNAME请求对比一下TTL。
这个需要自己搭建递归做一下实验吧!!! 研究\cite{Moura2019}建议对DNS负载均衡使用较短的TTL而其他的建议长一些,且根据Li\cite{Li2020}的基于机器学习识别CDN方案,IP和TTL相关特征对域名CDN识别较为有效。   

2.基于约减策略,观察CDN厂商是否根据使用频率来修改TTL值?


<为什么要测量DNS呢,因为看CDN厂商对LDNS的支持程度,一定程度上对用户有一定好处,
我们也可以绘制一个延迟地图,给用户作为选择>
我们选择secrank的合适域名,对全国常用公共递归进行判定,对每一个递归拿到一个平均值,以及解析成功率
根据这些对每一个递归进行评分,最终显示到地图上。同时给出一个总分,供用户选择CDN时有一定区域偏向,或选择多款CDN来弥补偏向


\chapter{基于***的IP通信时延测量方法}[IPtongxinshiyanceliang]
\section{引言}[yinyan]
\section{相关工作}[xiangguangongzuo]
\section{实验结果}[shiyanujieguo]
\section{本章小结}[benzhangxiaojie]

是否探测节点数量????????

1.使用全球ping等工具抓数据
2.对数据进行清晰(聚类分析等)
3.对数据进行分析(想一个算法)




\chapter{基于***的数据分析方法}[shujufenxi]

或者是分时段、分地区测量(上下文,用于服务第三章、第四章)

1.想一个方法来构造评估策略。

层次分析?有客观部分,加一个算法


2.研究一下CDN测量结果评价标准,有没有相应的国标等。重点关注测量方法,测量效率,评估方面。







% Local Variables:
% TeX-master: "../main"
% TeX-engine: xetex
% End:

% !Mode:: "TeX:UTF-8"



\chapter{基于***的部署优化方法}[shujufenxi]


博弈论???






% Local Variables:
% TeX-master: "../main"
% TeX-engine: xetex
% End:

% !Mode:: "TeX:UTF-8"

\chapter{系统构建}[xitonggoujian]



% Local Variables:
% TeX-master: "../main"
% TeX-engine: xetex
% End:


\backmatter
% !Mode:: "TeX:UTF-8" 
\begin{conclusions}

学位论文的结论作为论文正文的最后一章单独排写,但不加章标题序号。

结论应是作者在学位论文研究过程中所取得的创新性成果的概要总结,不能与摘要混为一谈。博士学位论文结论应包括论文的主要结果、创新点、展望三部分,在结论中应概括论文的核心观点,明确、客观地指出本研究内容的创新性成果(含新见解、新观点、方法创新、技术创新、理论创新),并指出今后进一步在本研究方向进行研究工作的展望与设想。对所取得的创新性成果应注意从定性和定量两方面给出科学、准确的评价,分(1)、(2)、(3)…条列出,宜用“提出了”、“建立了”等词叙述。

\end{conclusions}
   % 结论
\bibliographystyle{gbt7714-numerical}
%\bibliographystyle{hithesis} %理论上2020最新要求文献样式GB/T 7714—2015,但若院系要求文献英文作者不全大写,可改用hithesis文献样式
%%%%%%%%%%%%%%%%%%%%%%%%%%%%%%%%%%%%%%%%%%%%%%%%%%%%%%%%%%%%%%%%%%%%%%%%%%%%%%%%
%-- 注意:以下本硕博、博后书序不一致 --%
%%%%%%%%%%%%%%%%%%%%%%%%%%%%%%%%%%%%%%%%%%%%%%%%%%%%%%%%%%%%%%%%%%%%%%%%%%%%%%%%
% 本科书序(哈尔滨、深圳校区)
%%%%%%%%%%%%%%%%%%%%%%%%%%%%%%%%%%%%%%%%%%%%%%%%%%%%%%%%%%%%%%%%%%%%%%%%%%%%%%%%
%\bibliography{reference} % 参考文献
%\authorization %授权
% \authorization[scan.pdf] %添加扫描页的命令,与上互斥
%% !Mode:: "TeX:UTF-8"
\begin{acknowledgements}
衷心感谢导师~XXX~教授对本人的精心指导。他的言传身教将使我终生受益。

……

感谢哈工大\LaTeX\ 论文模板\hithesis\ !

\end{acknowledgements}
 %致谢
%\begin{appendix}%附录
%\chapter{外文资料原文}
\label{cha:engorg}

\title{The title of the English paper}

\textbf{Abstract:} As one of the most widely used techniques in operations
research, \emph{ mathematical programming} is defined as a means of maximizing a
quantity known as \emph{bjective function}, subject to a set of constraints
represented by equations and inequalities. Some known subtopics of mathematical
programming are linear programming, nonlinear programming, multiobjective
programming, goal programming, dynamic programming, and multilevel
programming$^{[1]}$.

It is impossible to cover in a single chapter every concept of mathematical
programming. This chapter introduces only the basic concepts and techniques of
mathematical programming such that readers gain an understanding of them
throughout the book$^{[2,3]}$.


\section{Single-Objective Programming}
The general form of single-objective programming (SOP) is written
as follows,
\begin{equation}\tag*{(123)} % 如果附录中的公式不想让它出现在公式索引中,那就请
                             % 用 \tag*{xxxx}
\left\{\begin{array}{l}
\max \,\,f(x)\\[0.1 cm]
\mbox{subject to:} \\ [0.1 cm]
\qquad g_j(x)\le 0,\quad j=1,2,\cdots,p
\end{array}\right.
\end{equation}
which maximizes a real-valued function $f$ of
$x=(x_1,x_2,\cdots,x_n)$ subject to a set of constraints.

\newtheorem{mpdef}{Definition}[chapter]
\begin{mpdef}
In SOP, we call $x$ a decision vector, and
$x_1,x_2,\cdots,x_n$ decision variables. The function
$f$ is called the objective function. The set
\begin{equation}\tag*{(456)} % 这里同理,其它不再一一指定。
S=\left\{x\in\Re^n\bigm|g_j(x)\le 0,\,j=1,2,\cdots,p\right\}
\end{equation}
is called the feasible set. An element $x$ in $S$ is called a
feasible solution.
\end{mpdef}

\newtheorem{mpdefop}[mpdef]{Definition}
\begin{mpdefop}
A feasible solution $x^*$ is called the optimal
solution of SOP if and only if
\begin{equation}
f(x^*)\ge f(x)
\end{equation}
for any feasible solution $x$.
\end{mpdefop}

One of the outstanding contributions to mathematical programming was known as
the Kuhn-Tucker conditions\ref{eq:ktc}. In order to introduce them, let us give
some definitions. An inequality constraint $g_j(x)\le 0$ is said to be active at
a point $x^*$ if $g_j(x^*)=0$. A point $x^*$ satisfying $g_j(x^*)\le 0$ is said
to be regular if the gradient vectors $\nabla g_j(x)$ of all active constraints
are linearly independent.

Let $x^*$ be a regular point of the constraints of SOP and assume that all the
functions $f(x)$ and $g_j(x),j=1,2,\cdots,p$ are differentiable. If $x^*$ is a
local optimal solution, then there exist Lagrange multipliers
$\lambda_j,j=1,2,\cdots,p$ such that the following Kuhn-Tucker conditions hold,
\begin{equation}
\label{eq:ktc}
\left\{\begin{array}{l}
    \nabla f(x^*)-\sum\limits_{j=1}^p\lambda_j\nabla g_j(x^*)=0\\[0.3cm]
    \lambda_jg_j(x^*)=0,\quad j=1,2,\cdots,p\\[0.2cm]
    \lambda_j\ge 0,\quad j=1,2,\cdots,p.
\end{array}\right.
\end{equation}
If all the functions $f(x)$ and $g_j(x),j=1,2,\cdots,p$ are convex and
differentiable, and the point $x^*$ satisfies the Kuhn-Tucker conditions
(\ref{eq:ktc}), then it has been proved that the point $x^*$ is a global optimal
solution of SOP.

\subsection{Linear Programming}
\label{sec:lp}

If the functions $f(x),g_j(x),j=1,2,\cdots,p$ are all linear, then SOP is called
a {\em linear programming}.

The feasible set of linear is always convex. A point $x$ is called an extreme
point of convex set $S$ if $x\in S$ and $x$ cannot be expressed as a convex
combination of two points in $S$. It has been shown that the optimal solution to
linear programming corresponds to an extreme point of its feasible set provided
that the feasible set $S$ is bounded. This fact is the basis of the {\em simplex
  algorithm} which was developed by Dantzig as a very efficient method for
solving linear programming.
\begin{table}[ht]
\centering
  \centering
  \caption*{Table~1\hskip1em This is an example for manually numbered table, which
    would not appear in the list of tables}
  \label{tab:badtabular2}
  \begin{tabular}[c]{|m{1.5cm}|c|c|c|c|c|c|}\hline
    \multicolumn{2}{|c|}{Network Topology} & \# of nodes &
    \multicolumn{3}{c|}{\# of clients} & Server \\\hline
    GT-ITM & Waxman Transit-Stub & 600 &
    \multirow{2}{2em}{2\%}&
    \multirow{2}{2em}{10\%}&
    \multirow{2}{2em}{50\%}&
    \multirow{2}{1.2in}{Max. Connectivity}\\\cline{1-3}
    \multicolumn{2}{|c|}{Inet-2.1} & 6000 & & & &\\\hline
    & \multicolumn{2}{c|}{ABCDEF} &\multicolumn{4}{c|}{} \\\hline
\end{tabular}
\end{table}

Roughly speaking, the simplex algorithm examines only the extreme points of the
feasible set, rather than all feasible points. At first, the simplex algorithm
selects an extreme point as the initial point. The successive extreme point is
selected so as to improve the objective function value. The procedure is
repeated until no improvement in objective function value can be made. The last
extreme point is the optimal solution.

\subsection{Nonlinear Programming}

If at least one of the functions $f(x),g_j(x),j=1,2,\cdots,p$ is nonlinear, then
SOP is called a {\em nonlinear programming}.

A large number of classical optimization methods have been developed to treat
special-structural nonlinear programming based on the mathematical theory
concerned with analyzing the structure of problems.

Now we consider a nonlinear programming which is confronted solely with
maximizing a real-valued function with domain $\Re^n$.  Whether derivatives are
available or not, the usual strategy is first to select a point in $\Re^n$ which
is thought to be the most likely place where the maximum exists. If there is no
information available on which to base such a selection, a point is chosen at
random. From this first point an attempt is made to construct a sequence of
points, each of which yields an improved objective function value over its
predecessor. The next point to be added to the sequence is chosen by analyzing
the behavior of the function at the previous points. This construction continues
until some termination criterion is met. Methods based upon this strategy are
called {\em ascent methods}, which can be classified as {\em direct methods},
{\em gradient methods}, and {\em Hessian methods} according to the information
about the behavior of objective function $f$. Direct methods require only that
the function can be evaluated at each point. Gradient methods require the
evaluation of first derivatives of $f$. Hessian methods require the evaluation
of second derivatives. In fact, there is no superior method for all
problems. The efficiency of a method is very much dependent upon the objective
function.

\subsection{Integer Programming}

{\em Integer programming} is a special mathematical programming in which all of
the variables are assumed to be only integer values. When there are not only
integer variables but also conventional continuous variables, we call it {\em
  mixed integer programming}. If all the variables are assumed either 0 or 1,
then the problem is termed a {\em zero-one programming}. Although integer
programming can be solved by an {\em exhaustive enumeration} theoretically, it
is impractical to solve realistically sized integer programming problems. The
most successful algorithm so far found to solve integer programming is called
the {\em branch-and-bound enumeration} developed by Balas (1965) and Dakin
(1965). The other technique to integer programming is the {\em cutting plane
  method} developed by Gomory (1959).

\hfill\textit{Uncertain Programming\/}\quad(\textsl{BaoDing Liu, 2006.2})

\section*{References}
\noindent{\itshape NOTE: These references are only for demonstration. They are
  not real citations in the original text.}

\begin{translationbib}
\item Donald E. Knuth. The \TeX book. Addison-Wesley, 1984. ISBN: 0-201-13448-9
\item Paul W. Abrahams, Karl Berry and Kathryn A. Hargreaves. \TeX\ for the
  Impatient. Addison-Wesley, 1990. ISBN: 0-201-51375-7
\item David Salomon. The advanced \TeX book.  New York : Springer, 1995. ISBN:0-387-94556-3
\end{translationbib}

\chapter{外文资料的调研阅读报告或书面翻译}

\title{英文资料的中文标题}

{\heiti 摘要:} 本章为外文资料翻译内容。如果有摘要可以直接写上来,这部分好像没有
明确的规定。

\section{单目标规划}
北冥有鱼,其名为鲲。鲲之大,不知其几千里也。化而为鸟,其名为鹏。鹏之背,不知其几
千里也。怒而飞,其翼若垂天之云。是鸟也,海运则将徙于南冥。南冥者,天池也。
\begin{equation}\tag*{(123)}
 p(y|\mathbf{x}) = \frac{p(\mathbf{x},y)}{p(\mathbf{x})}=
\frac{p(\mathbf{x}|y)p(y)}{p(\mathbf{x})}
\end{equation}

吾生也有涯,而知也无涯。以有涯随无涯,殆已!已而为知者,殆而已矣!为善无近名,为
恶无近刑,缘督以为经,可以保身,可以全生,可以养亲,可以尽年。

\subsection{线性规划}
庖丁为文惠君解牛,手之所触,肩之所倚,足之所履,膝之所倚,砉然响然,奏刀騞然,莫
不中音,合于桑林之舞,乃中经首之会。
\begin{table}[ht]
\centering
  \centering
  \caption*{表~1\hskip1em 这是手动编号但不出现在索引中的一个表格例子}
  \label{tab:badtabular3}
  \begin{tabular}[c]{|m{1.5cm}|c|c|c|c|c|c|}\hline
    \multicolumn{2}{|c|}{Network Topology} & \# of nodes &
    \multicolumn{3}{c|}{\# of clients} & Server \\\hline
    GT-ITM & Waxman Transit-Stub & 600 &
    \multirow{2}{2em}{2\%}&
    \multirow{2}{2em}{10\%}&
    \multirow{2}{2em}{50\%}&
    \multirow{2}{1.2in}{Max. Connectivity}\\\cline{1-3}
    \multicolumn{2}{|c|}{Inet-2.1} & 6000 & & & &\\\hline
    & \multicolumn{2}{c|}{ABCDEF} &\multicolumn{4}{c|}{} \\\hline
\end{tabular}
\end{table}

文惠君曰:“嘻,善哉!技盖至此乎?”庖丁释刀对曰:“臣之所好者道也,进乎技矣。始臣之
解牛之时,所见无非全牛者;三年之后,未尝见全牛也;方今之时,臣以神遇而不以目视,
官知止而神欲行。依乎天理,批大郤,导大窾,因其固然。技经肯綮之未尝,而况大坬乎!
良庖岁更刀,割也;族庖月更刀,折也;今臣之刀十九年矣,所解数千牛矣,而刀刃若新发
于硎。彼节者有间而刀刃者无厚,以无厚入有间,恢恢乎其于游刃必有余地矣。是以十九年
而刀刃若新发于硎。虽然,每至于族,吾见其难为,怵然为戒,视为止,行为迟,动刀甚微,
謋然已解,如土委地。提刀而立,为之而四顾,为之踌躇满志,善刀而藏之。”

文惠君曰:“善哉!吾闻庖丁之言,得养生焉。”


\subsection{非线性规划}
孔子与柳下季为友,柳下季之弟名曰盗跖。盗跖从卒九千人,横行天下,侵暴诸侯。穴室枢
户,驱人牛马,取人妇女。贪得忘亲,不顾父母兄弟,不祭先祖。所过之邑,大国守城,小
国入保,万民苦之。孔子谓柳下季曰:“夫为人父者,必能诏其子;为人兄者,必能教其弟。
若父不能诏其子,兄不能教其弟,则无贵父子兄弟之亲矣。今先生,世之才士也,弟为盗
跖,为天下害,而弗能教也,丘窃为先生羞之。丘请为先生往说之。”

柳下季曰:“先生言为人父者必能诏其子,为人兄者必能教其弟,若子不听父之诏,弟不受
兄之教,虽今先生之辩,将奈之何哉?且跖之为人也,心如涌泉,意如飘风,强足以距敌,
辩足以饰非。顺其心则喜,逆其心则怒,易辱人以言。先生必无往。”

孔子不听,颜回为驭,子贡为右,往见盗跖。

\subsection{整数规划}
盗跖乃方休卒徒大山之阳,脍人肝而餔之。孔子下车而前,见谒者曰:“鲁人孔丘,闻将军
高义,敬再拜谒者。”谒者入通。盗跖闻之大怒,目如明星,发上指冠,曰:“此夫鲁国之
巧伪人孔丘非邪?为我告之:尔作言造语,妄称文、武,冠枝木之冠,带死牛之胁,多辞缪
说,不耕而食,不织而衣,摇唇鼓舌,擅生是非,以迷天下之主,使天下学士不反其本,妄
作孝弟,而侥幸于封侯富贵者也。子之罪大极重,疾走归!不然,我将以子肝益昼餔之膳。”


\chapter{其它附录}
前面两个附录主要是给本科生做例子。其它附录的内容可以放到这里,当然如果你愿意,可
以把这部分也放到独立的文件中,然后将其到主文件中。
%本科生翻译论文
%\end{appendix}
%%%%%%%%%%%%%%%%%%%%%%%%%%%%%%%%%%%%%%%%%%%%%%%%%%%%%%%%%%%%%%%%%%%%%%%%%%%%%%%%
% 本科书序(威海校区)
%%%%%%%%%%%%%%%%%%%%%%%%%%%%%%%%%%%%%%%%%%%%%%%%%%%%%%%%%%%%%%%%%%%%%%%%%%%%%%%%
% \authorization %授权
% % \authorization[scan.pdf] %添加扫描页的命令,与上互斥
% \bibliography{reference} % 参考文献 
% % !Mode:: "TeX:UTF-8"
\begin{acknowledgements}
衷心感谢导师~XXX~教授对本人的精心指导。他的言传身教将使我终生受益。

……

感谢哈工大\LaTeX\ 论文模板\hithesis\ !

\end{acknowledgements}
 %致谢
% \begin{appendix}%附录 
% \chapter{外文资料原文}
\label{cha:engorg}

\title{The title of the English paper}

\textbf{Abstract:} As one of the most widely used techniques in operations
research, \emph{ mathematical programming} is defined as a means of maximizing a
quantity known as \emph{bjective function}, subject to a set of constraints
represented by equations and inequalities. Some known subtopics of mathematical
programming are linear programming, nonlinear programming, multiobjective
programming, goal programming, dynamic programming, and multilevel
programming$^{[1]}$.

It is impossible to cover in a single chapter every concept of mathematical
programming. This chapter introduces only the basic concepts and techniques of
mathematical programming such that readers gain an understanding of them
throughout the book$^{[2,3]}$.


\section{Single-Objective Programming}
The general form of single-objective programming (SOP) is written
as follows,
\begin{equation}\tag*{(123)} % 如果附录中的公式不想让它出现在公式索引中,那就请
                             % 用 \tag*{xxxx}
\left\{\begin{array}{l}
\max \,\,f(x)\\[0.1 cm]
\mbox{subject to:} \\ [0.1 cm]
\qquad g_j(x)\le 0,\quad j=1,2,\cdots,p
\end{array}\right.
\end{equation}
which maximizes a real-valued function $f$ of
$x=(x_1,x_2,\cdots,x_n)$ subject to a set of constraints.

\newtheorem{mpdef}{Definition}[chapter]
\begin{mpdef}
In SOP, we call $x$ a decision vector, and
$x_1,x_2,\cdots,x_n$ decision variables. The function
$f$ is called the objective function. The set
\begin{equation}\tag*{(456)} % 这里同理,其它不再一一指定。
S=\left\{x\in\Re^n\bigm|g_j(x)\le 0,\,j=1,2,\cdots,p\right\}
\end{equation}
is called the feasible set. An element $x$ in $S$ is called a
feasible solution.
\end{mpdef}

\newtheorem{mpdefop}[mpdef]{Definition}
\begin{mpdefop}
A feasible solution $x^*$ is called the optimal
solution of SOP if and only if
\begin{equation}
f(x^*)\ge f(x)
\end{equation}
for any feasible solution $x$.
\end{mpdefop}

One of the outstanding contributions to mathematical programming was known as
the Kuhn-Tucker conditions\ref{eq:ktc}. In order to introduce them, let us give
some definitions. An inequality constraint $g_j(x)\le 0$ is said to be active at
a point $x^*$ if $g_j(x^*)=0$. A point $x^*$ satisfying $g_j(x^*)\le 0$ is said
to be regular if the gradient vectors $\nabla g_j(x)$ of all active constraints
are linearly independent.

Let $x^*$ be a regular point of the constraints of SOP and assume that all the
functions $f(x)$ and $g_j(x),j=1,2,\cdots,p$ are differentiable. If $x^*$ is a
local optimal solution, then there exist Lagrange multipliers
$\lambda_j,j=1,2,\cdots,p$ such that the following Kuhn-Tucker conditions hold,
\begin{equation}
\label{eq:ktc}
\left\{\begin{array}{l}
    \nabla f(x^*)-\sum\limits_{j=1}^p\lambda_j\nabla g_j(x^*)=0\\[0.3cm]
    \lambda_jg_j(x^*)=0,\quad j=1,2,\cdots,p\\[0.2cm]
    \lambda_j\ge 0,\quad j=1,2,\cdots,p.
\end{array}\right.
\end{equation}
If all the functions $f(x)$ and $g_j(x),j=1,2,\cdots,p$ are convex and
differentiable, and the point $x^*$ satisfies the Kuhn-Tucker conditions
(\ref{eq:ktc}), then it has been proved that the point $x^*$ is a global optimal
solution of SOP.

\subsection{Linear Programming}
\label{sec:lp}

If the functions $f(x),g_j(x),j=1,2,\cdots,p$ are all linear, then SOP is called
a {\em linear programming}.

The feasible set of linear is always convex. A point $x$ is called an extreme
point of convex set $S$ if $x\in S$ and $x$ cannot be expressed as a convex
combination of two points in $S$. It has been shown that the optimal solution to
linear programming corresponds to an extreme point of its feasible set provided
that the feasible set $S$ is bounded. This fact is the basis of the {\em simplex
  algorithm} which was developed by Dantzig as a very efficient method for
solving linear programming.
\begin{table}[ht]
\centering
  \centering
  \caption*{Table~1\hskip1em This is an example for manually numbered table, which
    would not appear in the list of tables}
  \label{tab:badtabular2}
  \begin{tabular}[c]{|m{1.5cm}|c|c|c|c|c|c|}\hline
    \multicolumn{2}{|c|}{Network Topology} & \# of nodes &
    \multicolumn{3}{c|}{\# of clients} & Server \\\hline
    GT-ITM & Waxman Transit-Stub & 600 &
    \multirow{2}{2em}{2\%}&
    \multirow{2}{2em}{10\%}&
    \multirow{2}{2em}{50\%}&
    \multirow{2}{1.2in}{Max. Connectivity}\\\cline{1-3}
    \multicolumn{2}{|c|}{Inet-2.1} & 6000 & & & &\\\hline
    & \multicolumn{2}{c|}{ABCDEF} &\multicolumn{4}{c|}{} \\\hline
\end{tabular}
\end{table}

Roughly speaking, the simplex algorithm examines only the extreme points of the
feasible set, rather than all feasible points. At first, the simplex algorithm
selects an extreme point as the initial point. The successive extreme point is
selected so as to improve the objective function value. The procedure is
repeated until no improvement in objective function value can be made. The last
extreme point is the optimal solution.

\subsection{Nonlinear Programming}

If at least one of the functions $f(x),g_j(x),j=1,2,\cdots,p$ is nonlinear, then
SOP is called a {\em nonlinear programming}.

A large number of classical optimization methods have been developed to treat
special-structural nonlinear programming based on the mathematical theory
concerned with analyzing the structure of problems.

Now we consider a nonlinear programming which is confronted solely with
maximizing a real-valued function with domain $\Re^n$.  Whether derivatives are
available or not, the usual strategy is first to select a point in $\Re^n$ which
is thought to be the most likely place where the maximum exists. If there is no
information available on which to base such a selection, a point is chosen at
random. From this first point an attempt is made to construct a sequence of
points, each of which yields an improved objective function value over its
predecessor. The next point to be added to the sequence is chosen by analyzing
the behavior of the function at the previous points. This construction continues
until some termination criterion is met. Methods based upon this strategy are
called {\em ascent methods}, which can be classified as {\em direct methods},
{\em gradient methods}, and {\em Hessian methods} according to the information
about the behavior of objective function $f$. Direct methods require only that
the function can be evaluated at each point. Gradient methods require the
evaluation of first derivatives of $f$. Hessian methods require the evaluation
of second derivatives. In fact, there is no superior method for all
problems. The efficiency of a method is very much dependent upon the objective
function.

\subsection{Integer Programming}

{\em Integer programming} is a special mathematical programming in which all of
the variables are assumed to be only integer values. When there are not only
integer variables but also conventional continuous variables, we call it {\em
  mixed integer programming}. If all the variables are assumed either 0 or 1,
then the problem is termed a {\em zero-one programming}. Although integer
programming can be solved by an {\em exhaustive enumeration} theoretically, it
is impractical to solve realistically sized integer programming problems. The
most successful algorithm so far found to solve integer programming is called
the {\em branch-and-bound enumeration} developed by Balas (1965) and Dakin
(1965). The other technique to integer programming is the {\em cutting plane
  method} developed by Gomory (1959).

\hfill\textit{Uncertain Programming\/}\quad(\textsl{BaoDing Liu, 2006.2})

\section*{References}
\noindent{\itshape NOTE: These references are only for demonstration. They are
  not real citations in the original text.}

\begin{translationbib}
\item Donald E. Knuth. The \TeX book. Addison-Wesley, 1984. ISBN: 0-201-13448-9
\item Paul W. Abrahams, Karl Berry and Kathryn A. Hargreaves. \TeX\ for the
  Impatient. Addison-Wesley, 1990. ISBN: 0-201-51375-7
\item David Salomon. The advanced \TeX book.  New York : Springer, 1995. ISBN:0-387-94556-3
\end{translationbib}

\chapter{外文资料的调研阅读报告或书面翻译}

\title{英文资料的中文标题}

{\heiti 摘要:} 本章为外文资料翻译内容。如果有摘要可以直接写上来,这部分好像没有
明确的规定。

\section{单目标规划}
北冥有鱼,其名为鲲。鲲之大,不知其几千里也。化而为鸟,其名为鹏。鹏之背,不知其几
千里也。怒而飞,其翼若垂天之云。是鸟也,海运则将徙于南冥。南冥者,天池也。
\begin{equation}\tag*{(123)}
 p(y|\mathbf{x}) = \frac{p(\mathbf{x},y)}{p(\mathbf{x})}=
\frac{p(\mathbf{x}|y)p(y)}{p(\mathbf{x})}
\end{equation}

吾生也有涯,而知也无涯。以有涯随无涯,殆已!已而为知者,殆而已矣!为善无近名,为
恶无近刑,缘督以为经,可以保身,可以全生,可以养亲,可以尽年。

\subsection{线性规划}
庖丁为文惠君解牛,手之所触,肩之所倚,足之所履,膝之所倚,砉然响然,奏刀騞然,莫
不中音,合于桑林之舞,乃中经首之会。
\begin{table}[ht]
\centering
  \centering
  \caption*{表~1\hskip1em 这是手动编号但不出现在索引中的一个表格例子}
  \label{tab:badtabular3}
  \begin{tabular}[c]{|m{1.5cm}|c|c|c|c|c|c|}\hline
    \multicolumn{2}{|c|}{Network Topology} & \# of nodes &
    \multicolumn{3}{c|}{\# of clients} & Server \\\hline
    GT-ITM & Waxman Transit-Stub & 600 &
    \multirow{2}{2em}{2\%}&
    \multirow{2}{2em}{10\%}&
    \multirow{2}{2em}{50\%}&
    \multirow{2}{1.2in}{Max. Connectivity}\\\cline{1-3}
    \multicolumn{2}{|c|}{Inet-2.1} & 6000 & & & &\\\hline
    & \multicolumn{2}{c|}{ABCDEF} &\multicolumn{4}{c|}{} \\\hline
\end{tabular}
\end{table}

文惠君曰:“嘻,善哉!技盖至此乎?”庖丁释刀对曰:“臣之所好者道也,进乎技矣。始臣之
解牛之时,所见无非全牛者;三年之后,未尝见全牛也;方今之时,臣以神遇而不以目视,
官知止而神欲行。依乎天理,批大郤,导大窾,因其固然。技经肯綮之未尝,而况大坬乎!
良庖岁更刀,割也;族庖月更刀,折也;今臣之刀十九年矣,所解数千牛矣,而刀刃若新发
于硎。彼节者有间而刀刃者无厚,以无厚入有间,恢恢乎其于游刃必有余地矣。是以十九年
而刀刃若新发于硎。虽然,每至于族,吾见其难为,怵然为戒,视为止,行为迟,动刀甚微,
謋然已解,如土委地。提刀而立,为之而四顾,为之踌躇满志,善刀而藏之。”

文惠君曰:“善哉!吾闻庖丁之言,得养生焉。”


\subsection{非线性规划}
孔子与柳下季为友,柳下季之弟名曰盗跖。盗跖从卒九千人,横行天下,侵暴诸侯。穴室枢
户,驱人牛马,取人妇女。贪得忘亲,不顾父母兄弟,不祭先祖。所过之邑,大国守城,小
国入保,万民苦之。孔子谓柳下季曰:“夫为人父者,必能诏其子;为人兄者,必能教其弟。
若父不能诏其子,兄不能教其弟,则无贵父子兄弟之亲矣。今先生,世之才士也,弟为盗
跖,为天下害,而弗能教也,丘窃为先生羞之。丘请为先生往说之。”

柳下季曰:“先生言为人父者必能诏其子,为人兄者必能教其弟,若子不听父之诏,弟不受
兄之教,虽今先生之辩,将奈之何哉?且跖之为人也,心如涌泉,意如飘风,强足以距敌,
辩足以饰非。顺其心则喜,逆其心则怒,易辱人以言。先生必无往。”

孔子不听,颜回为驭,子贡为右,往见盗跖。

\subsection{整数规划}
盗跖乃方休卒徒大山之阳,脍人肝而餔之。孔子下车而前,见谒者曰:“鲁人孔丘,闻将军
高义,敬再拜谒者。”谒者入通。盗跖闻之大怒,目如明星,发上指冠,曰:“此夫鲁国之
巧伪人孔丘非邪?为我告之:尔作言造语,妄称文、武,冠枝木之冠,带死牛之胁,多辞缪
说,不耕而食,不织而衣,摇唇鼓舌,擅生是非,以迷天下之主,使天下学士不反其本,妄
作孝弟,而侥幸于封侯富贵者也。子之罪大极重,疾走归!不然,我将以子肝益昼餔之膳。”


\chapter{其它附录}
前面两个附录主要是给本科生做例子。其它附录的内容可以放到这里,当然如果你愿意,可
以把这部分也放到独立的文件中,然后将其到主文件中。
%本科生翻译论文
% \end{appendix} 
%%%%%%%%%%%%%%%%%%%%%%%%%%%%%%%%%%%%%%%%%%%%%%%%%%%%%%%%%%%%%%%%%%%%%%%%%%%%%%%%
% 硕博书序
%%%%%%%%%%%%%%%%%%%%%%%%%%%%%%%%%%%%%%%%%%%%%%%%%%%%%%%%%%%%%%%%%%%%%%%%%%%%%%%% 
\bibliography{reference} % 参考文献 
%\nocite{*} %显示所有文献  , 后续一定要删掉这个!!!!!!!!!!!!!!!!!!!!!!!!!!!!!!!!!!!!!!!!!! 
%\begin{appendix}%附录
%	% -*-coding: utf-8 -*-
%%%%%%%%%%%%%%%%%%%%%%%%%%%%%%%%%%%%%%%%%%%%%%%%%%%%%%%%%
\chapter{带章节的附录}[Full Appendix]%
完整的附录内容,包含章节,公式,图表等

%%%%%%%%%%%%%%%%%%%%%%%%%%%%%%%%%%%%%%%%%%%%%%%%%%%%%%%%%
\section{附录节的内容}[Section in Appendix]
这是附录的节的内容

附录中图的示例:
\begin{figure}[htbp]
\centering
\includegraphics[width = 0.4\textwidth]{golfer}
%\bicaption[golfer5]{}{\xiaosi[0]打高尔夫球的人}{Fig.$\!$}{The person playing golf}\vspace{-1em}
\caption{\xiaosi[0]打高尔夫球的人}
\end{figure}

附录中公式的示例:
\begin{align}
a & = b \times c \\
E & = m c^2
\label{eq}
\end{align}

\chapter{这个星球上最好的免费Linux软件列表}[List of the Best Linux Software in our Planet]
\section{系统}

\href{http://fvwm.org/}{FVWM自从上世纪诞生以来,此星球最强大的窗口管理器。}
推荐基于FVWM的桌面设计hifvwm:\href{https://github.com/dustincys/hifvwm}{https://github.com/dustincys/hifvwm}。

\subsection{hifvwm的优点}

\begin{enumerate}
	\item 即使打开上百个窗口也不会“蒙圈”,对比win或mac都无法做到。计算机性能越来越强大,窗口任务的管理必须要升级到打怪兽级别。
	\item 二维可视化任务栏。
	\item 自动同步Bing搜索主页的壁纸。每次电脑开机,午夜零点自动更新,用户
		也可以手动更新,从此审美再也不疲劳。
	\item 切换窗口自动聚焦到最上面的窗口。使用键盘快捷键切换窗口时候,减少
		操作过程,自动聚焦到目标窗口。这一特性是虚拟窗口必须的人性化设
		计。
	\item 类似window右下角的功能的最小化窗口来显示桌面的功能此处类似
		win7/win10,实现在一个桌面之内操作多个任务。
	\item 任务栏结合标题栏。采用任务栏和标题栏结合,节省空间。
	\item 同类窗口切换。可以在同类窗口之内类似alt-tab的方式切换。
	\item ……
\end{enumerate}

\section{其他}

\href{https://github.com/goldendict/goldendict}{goldendict 星球最强大的桌面字典。}

\href{https://github.com/yarrick/iodine}{iodine,“HIT-WLAN + 锐捷”时代的福音。}

\href{http://www.aircrack-ng.org/}{aircrack,Wifi“安全性评估”工具,自由上网,
  就是隔壁寝室网络会变慢一点。}

\href{https://www.ledger-cli.org/}{ledger,前“金融区块链”时代最好的复式记账系统。}

\href{https://orgmode.org/}{orgmode,最强大的笔记系统,从来没有之一。}

\href{https://www.jianguoyun.com/}{坚果云,国内一款支持WebDav的云盘系统,国内真正的云盘没有之一。}

\href{https://notmuchmail.org/}{notmuch, 目前最好的邮件管理工具,还在为每天几百
  个email苦恼?几百个这些都不算多,notmuch。}

\section{vim}
实现中英文每一句一行,以及实现每一句折叠断行的简单正则式,tex源码更加乖乖。
\begin{lstlisting}
vnoremap <leader>fae J:s/[.!?]\zs\s\+/\="\r".matchstr(getline('.'), '^\s*')/g<CR>
vnoremap <leader>fac J:s/[。!?]/\=submatch(0)."\n".matchstr(getline('.'), '^\s*')/g<CR>
vnoremap <leader>fle :!fmt -80 -s<CR>
\end{lstlisting}

%\end{appendix}  
% !Mode:: "TeX:UTF-8" 
\begin{publication}
\noindent\textbf{发表的相关论文}
\begin{publist}
\item	XXX,XXX. Static Oxidation Model of Al-Mg/C Dissipation Thermal Protection Materials[J]. Rare Metal Materials and Engineering, 2010, 39(Suppl. 1): 520-524.(SCI~收录,IDS号为~669JS,IF=0.16)
\item XXX,XXX. 精密超声振动切削单晶铜的计算机仿真研究[J]. 系统仿真学报,2007,19(4):738-741,753.(EI~收录号:20071310514841)
\item XXX,XXX. 局部多孔质气体静压轴向轴承静态特性的数值求解[J]. 摩擦学学报,2007(1):68-72.(EI~收录号:20071510544816)
\item XXX,XXX. 硬脆光学晶体材料超精密切削理论研究综述[J]. 机械工程学报,2003,39(8):15-22.(EI~收录号:2004088028875)
\item XXX,XXX. 基于遗传算法的超精密切削加工表面粗糙度预测模型的参数辨识以及切削参数优化[J]. 机械工程学报,2005,41(11):158-162.(EI~收录号:2006039650087)
\item XXX,XXX. Discrete Sliding Mode Cintrok with Fuzzy Adaptive Reaching Law on 6-PEES Parallel Robot[C]. Intelligent System Design and Applications, Jinan, 2006: 649-652.(EI~收录号:20073210746529)
\end{publist}

\noindent\textbf{(二)申请及已获得的专利}
\begin{publist}
\item 张兆心,李超,程亚楠,郭长勇,杜跃进,\textbf{门浩}. 一种获取网络数据时网络阻塞造成的噪声数据消除方法: 中国,CN202110121032.7[P]. 2022-04-15. (已授权)

\item 张兆心,李超,柴婷婷,程亚楠,陆柯羽,郭长勇,杜跃进,\textbf{门浩}. 基于网络服务商国别标注的域名国家可控性评估方法:中国, CN202110258091.9[P]. 2021-06-01.(已受理)

\item 张兆心,李超,程亚楠,陆柯羽,\textbf{门浩}. 一种基于多源信息定位域名根镜像节点地理位置的方法: 中国,CN202110856090.4[P]. 2021-11-02. (已受理)


\end{publist}

\noindent\textbf{(三)参与的科研项目及获奖情况}
\begin{publist}
\item 域名安全分析系统,奇安信合作项目,2020年10月1日-2021年10月1日.50W
\item	XXX,XXX. XX~气体静压轴承技术研究, XX~省自然科学基金项目.课题编号:XXXX.
\item XXX,XXX. XX~静载下预应力混凝土房屋结构设计统一理论. 黑江省科学技术二等奖, 2007.
\end{publist}
%\vfill
%\hangafter=1\hangindent=2em\noindent
%\setlength{\parindent}{2em}
\end{publication}
    % 所发文章
%\begin{ceindex}
  %如果想要手动加索引,注释掉以下这一样,用wordlist环境
\printsubindex*
\end{ceindex}
    % 索引, 根据自己的情况添加或者不添加,选择自动添加或者手工添加。
%\authorization %授权
%\authorization[scan.pdf] %添加扫描页的命令,与上互斥
% !Mode:: "TeX:UTF-8"
\begin{acknowledgements}
衷心感谢导师~XXX~教授对本人的精心指导。他的言传身教将使我终生受益。

……

感谢哈工大\LaTeX\ 论文模板\hithesis\ !

\end{acknowledgements}
 %致谢  
%\include{body/regu}  % 格式
%\include{body/introduction} %格式
% % !Mode:: "TeX:UTF-8" 

\begin{resume}
XXXX~年~XX~月~XX~日出生于~XXXX。

XXXX~年~XX~月考入~XX~大学~XX~院(系)XX~专业,XXXX~年~XX~月本科毕业并获得~XX~学学士学位。

XXXX~年~XX~月------XXXX~年~XX~月在~XX~大学~XX~院(系)XX~学科学习并获得~XX~学硕士学位。

XXXX~年~XX~月------XXXX~年~XX~月在~XX~大学~XX~院(系)XX~学科学习并获得~XX~学博士学位。

获奖情况:如获三好学生、优秀团干部、X~奖学金等(不含科研学术获奖)。

工作经历:

\textbf{( 除全日制硕士生以外,其余学生均应增列此项。个人简历一般应包含教育经历和工作经历。)}
\end{resume}
          % 博士学位论文有个人简介
%%%%%%%%%%%%%%%%%%%%%%%%%%%%%%%%%%%%%%%%%%%%%%%%%%%%%%%%%%%%%%%%%%%%%%%%%%%%%%%%
% 博后书序
%%%%%%%%%%%%%%%%%%%%%%%%%%%%%%%%%%%%%%%%%%%%%%%%%%%%%%%%%%%%%%%%%%%%%%%%%%%%%%%%
% \bibliography{reference} % 参考文献
% % !Mode:: "TeX:UTF-8"
\begin{acknowledgements}
衷心感谢导师~XXX~教授对本人的精心指导。他的言传身教将使我终生受益。

……

感谢哈工大\LaTeX\ 论文模板\hithesis\ !

\end{acknowledgements}
 %致谢
% % !Mode:: "TeX:UTF-8" 

\begin{doctorpublication}
\noindent\textbf{(一)发表的学术论文}
\begin{publist}
\item	XXX,XXX. Static Oxidation Model of Al-Mg/C Dissipation Thermal Protection Materials[J]. Rare Metal Materials and Engineering, 2010, 39(Suppl. 1): 520-524.(SCI~收录,IDS号为~669JS,IF=0.16)
\item XXX,XXX. 精密超声振动切削单晶铜的计算机仿真研究[J]. 系统仿真学报,2007,19(4):738-741,753.(EI~收录号:20071310514841)
\item XXX,XXX. 局部多孔质气体静压轴向轴承静态特性的数值求解[J]. 摩擦学学报,2007(1):68-72.(EI~收录号:20071510544816)
\item XXX,XXX. 硬脆光学晶体材料超精密切削理论研究综述[J]. 机械工程学报,2003,39(8):15-22.(EI~收录号:2004088028875)
\item XXX,XXX. 基于遗传算法的超精密切削加工表面粗糙度预测模型的参数辨识以及切削参数优化[J]. 机械工程学报,2005,41(11):158-162.(EI~收录号:2006039650087)
\item XXX,XXX. Discrete Sliding Mode Cintrok with Fuzzy Adaptive Reaching Law on 6-PEES Parallel Robot[C]. Intelligent System Design and Applications, Jinan, 2006: 649-652.(EI~收录号:20073210746529)
\end{publist}

\noindent\textbf{(二)申请及已获得的专利(无专利时此项不必列出)}
\begin{publist}
\item XXX,XXX. 一种温热外敷药制备方案:中国,88105607.3[P]. 1989-07-26.
\end{publist}

\noindent\textbf{(三)参与的科研项目及获奖情况}
\begin{publist}
\item	XXX,XXX. XX~气体静压轴承技术研究, XX~省自然科学基金项目.课题编号:XXXX.
\item XXX,XXX. XX~静载下预应力混凝土房屋结构设计统一理论. 黑江省科学技术二等奖, 2007.
\end{publist}
%\vfill
%\hangafter=1\hangindent=2em\noindent
%\setlength{\parindent}{2em}
\end{doctorpublication}
    % 所发文章
% % !Mode:: "TeX:UTF-8" 
\begin{publication}
\noindent\textbf{发表的相关论文}
\begin{publist}
\item	XXX,XXX. Static Oxidation Model of Al-Mg/C Dissipation Thermal Protection Materials[J]. Rare Metal Materials and Engineering, 2010, 39(Suppl. 1): 520-524.(SCI~收录,IDS号为~669JS,IF=0.16)
\item XXX,XXX. 精密超声振动切削单晶铜的计算机仿真研究[J]. 系统仿真学报,2007,19(4):738-741,753.(EI~收录号:20071310514841)
\item XXX,XXX. 局部多孔质气体静压轴向轴承静态特性的数值求解[J]. 摩擦学学报,2007(1):68-72.(EI~收录号:20071510544816)
\item XXX,XXX. 硬脆光学晶体材料超精密切削理论研究综述[J]. 机械工程学报,2003,39(8):15-22.(EI~收录号:2004088028875)
\item XXX,XXX. 基于遗传算法的超精密切削加工表面粗糙度预测模型的参数辨识以及切削参数优化[J]. 机械工程学报,2005,41(11):158-162.(EI~收录号:2006039650087)
\item XXX,XXX. Discrete Sliding Mode Cintrok with Fuzzy Adaptive Reaching Law on 6-PEES Parallel Robot[C]. Intelligent System Design and Applications, Jinan, 2006: 649-652.(EI~收录号:20073210746529)
\end{publist}

\noindent\textbf{(二)申请及已获得的专利}
\begin{publist}
\item 张兆心,李超,程亚楠,郭长勇,杜跃进,\textbf{门浩}. 一种获取网络数据时网络阻塞造成的噪声数据消除方法: 中国,CN202110121032.7[P]. 2022-04-15. (已授权)

\item 张兆心,李超,柴婷婷,程亚楠,陆柯羽,郭长勇,杜跃进,\textbf{门浩}. 基于网络服务商国别标注的域名国家可控性评估方法:中国, CN202110258091.9[P]. 2021-06-01.(已受理)

\item 张兆心,李超,程亚楠,陆柯羽,\textbf{门浩}. 一种基于多源信息定位域名根镜像节点地理位置的方法: 中国,CN202110856090.4[P]. 2021-11-02. (已受理)


\end{publist}

\noindent\textbf{(三)参与的科研项目及获奖情况}
\begin{publist}
\item 域名安全分析系统,奇安信合作项目,2020年10月1日-2021年10月1日.50W
\item	XXX,XXX. XX~气体静压轴承技术研究, XX~省自然科学基金项目.课题编号:XXXX.
\item XXX,XXX. XX~静载下预应力混凝土房屋结构设计统一理论. 黑江省科学技术二等奖, 2007.
\end{publist}
%\vfill
%\hangafter=1\hangindent=2em\noindent
%\setlength{\parindent}{2em}
\end{publication}
    % 所发文章
% % !Mode:: "TeX:UTF-8" 

\begin{resume}
XXXX~年~XX~月~XX~日出生于~XXXX。

XXXX~年~XX~月考入~XX~大学~XX~院(系)XX~专业,XXXX~年~XX~月本科毕业并获得~XX~学学士学位。

XXXX~年~XX~月------XXXX~年~XX~月在~XX~大学~XX~院(系)XX~学科学习并获得~XX~学硕士学位。

XXXX~年~XX~月------XXXX~年~XX~月在~XX~大学~XX~院(系)XX~学科学习并获得~XX~学博士学位。

获奖情况:如获三好学生、优秀团干部、X~奖学金等(不含科研学术获奖)。

工作经历:

\textbf{( 除全日制硕士生以外,其余学生均应增列此项。个人简历一般应包含教育经历和工作经历。)}
\end{resume}
          % 博士学位论文有个人简介
% % !Mode:: "TeX:UTF-8"
\begin{correspondingaddr}
  \heiti\xiaosi
  \noindent 永久通讯地址: \par
  \noindent email: \par
  \noindent 电话: \par
\end{correspondingaddr}
 %通信地址
%%%%%%%%%%%%%%%%%%%%%%%%%%%%%%%%%%%%%%%%%%%%%%%%%%%%%%%%%%%%%%%%%%%%%%%%%%%%%%%%
\end{document}
% Local Variables:
% TeX-engine: xetex
% End:
