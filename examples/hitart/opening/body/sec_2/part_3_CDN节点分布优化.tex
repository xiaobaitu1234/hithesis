% !Mode:: "TeX:UTF-8"
 
\subsection{CDN放置优化策略研究现状}

CDN放置策略主要从边缘服务器放置,内容缓存\cite{Sahoo2016}两方面进行研究。前者用于确定新位置来放置边缘服务器,内容缓存方面,则是考虑一组内容对象与已部署边缘服务器的关系,称为缓存/复制策略。传统CDN放置策略期望达到QoS约束下,使成本最小\cite{salahuddin2017survey,Pathan-survey-2007,Tang-2018}。
 

\textbf{边缘服务器放置方面}, 
Huang\cite{Huang2008}在对CDN进行测绘时,发现CDN分布的两种设计理念,一种是深入互联网服务提供商(Internet Service Provider,ISP)的设计,其更接近终端用户,在时延和吞吐量方面有较大优化。但集群较为分散,管理困难。另一种是ISP入户设计,即只在几个关键点建立大型内容分发中心,并使用高速链路互联,与第一种相比,在维护和管理费用方面有所降低,但时延性能有一定损失。
清华大学朱文武\cite{zhu-2021}指出CDN设置逐渐边缘化,靠近终端用户。针对优酷部署的路由宝入户现象进行分析,发现其智能路由器可配置多达1TB的存储空间,起到辅助CDN进行内容分发的功能。并研究一种边缘网络视频内容分发的新架构,使智能路由器拥有轻量级的内容分发功能,达到CDN下沉的目的。
Sahoo\cite{Sahoo2016}指出,传统CDN边缘服务器放置是非确定性多项式困难问题(Non-deterministic Polynomial Hard,NP-Hard)。其问题定义如下:给定一组服务器侯选位置,一组用户位置,边缘服务器放置涉及从候选集中找到边缘服务器的最佳数量和位置,使得每个终端用户分配到其中一个服务器,且CDN服务商的成本最小化。同理,基于云的CDN使用租用的虚拟化资源建立边缘服务器,需要结合云提供商考虑,基于网络功能虚拟化(Network Functions Virtualization,NFV)的CDN中,服务器放置需要虚拟化网络功能(Virtual Network Functions,VNF)实现的CDN节点,称为VNF放置问题。基于虚拟化程度分类,传统CDN放置依赖于离线算法实现,基于云和VNF的放置则支持在线算法。
Benkacem\cite{benkacem2018optimal}等人介绍了一种CDN即服务(CDN as a Service,CDNaaS)平台,支持跨云平台对基于VNF的CDN进行动态部署和管理。相应的,使用两个整数线性规划(Integer Linear Programming,ILP),分别用于降低成本和提高服务质量。并使用博弈论提供权衡方案。
Tang\cite{Tang-2018}等人在部署优化方面,额外考虑了动态用户流量,并将流量分发成本降维至ILP问题。随后,通过贪心算法来解决ILP问题的可扩展形式。

\textbf{内容缓存方面},
清华大学的葛志诚\cite{葛志诚2018一种移动内容分发网络的分层协同缓存机制}等人由移动CDN缓存入手,针对资源缓存策略以及资源请求路径,研究多层缓存协调配合问题,基于贪心思想提出启发式分层协作缓存策略。
Liu\cite{liu2018}重点关注移动CDN基站协作问题,使用随机优化模型对网络稳定约束下的长时间平均传输成本进行优化。有效决定了内容放置和请求重定向问题,在拥塞避免和降低传输成本方面拥有高性能。
Fu\cite{fu2018}则针对任播CDN入手,通过使用软件定义网络(Software Defined Network,SDN)结合NFV控制细粒度流量重定向,使依赖BGP路由的任播CDN可根据实际情况重定向,通过控制路径来提高CDN的服务质量。
此外,也有少数研究从结构方面入手,进行CDN服务的优化,如
Li\cite{Li-2019-AI-attention-CDN}通过构建通用人工智能(Artificial Intelligence,AI)注意力网络对CDN服务的性能进行预测。
Fayed\cite{Fayed-IpUnbind-2021}使用可编程套接字sk\_lookup,使IP与主机资源解绑,实现一种更灵活的CDN服务,节约IP资源。


% Local Variables:
% TeX-master: "../section_2_研究现状"
% TeX-engine: xetex
% End: