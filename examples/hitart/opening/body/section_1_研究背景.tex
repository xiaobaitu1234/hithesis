% !Mode:: "TeX:UTF-8"

\section{课题来源及研究的背景和意义}
	本节对研究的课题来源及研究目的和意义进行介绍。
\subsection{课题的来源}
本课题来源于哈尔滨工业大学(威海)网络与信息安全技术研究中心。
\subsection{课题研究的背景和意义}
随着计算机和网络技术的快速发展、互联网及移动互联网的普及,形成了以Windows、Mac OS、Linux为典型的桌面操作系统以及以Android、iOS为典型的移动操作系统占据了操作系统绝大多数市场份额。同时基于这些操作系统平台的包括办公、社交、娱乐在内的系统软件、应用软件、应用程序都已经经历过各自井喷式的大发展而变得更加成熟,热门应用软件和应用程序的下载量和用户量都已经积累了非常惊人的数量。

与之相伴的是网络上出现了越来越多的恶意程序或软件。恶意软件是一种破坏受害者的工作站、移动设备、服务器和网关的程序。常见的恶意软件程序有病毒、蠕虫、木马、间谍软件、勒索软件、恐吓软件、僵尸程序、Rootkit、广告、后门等。罪犯利用恶意软件攻击个人和组织,实现破坏操作系统、破坏电脑或网络、窃取机密数据、收集用户隐私信息、劫持或加密敏感数据等目标。这不仅给普通用户带来了极大的困扰,同时也给企业与政府部门带来了不可小觑的损失。恶意程序或软件存在于各种操作系统中,并且随着(移动)互联网的飞速发展而广泛传播。据国家互联网应急中心(CNCERT/CC)监测数据显示,恶意程序逐年在数量上呈几何级数增长。在种类繁多的恶意软件中,程序后门、间谍程序、广告等在未明确提示用户或未经用户许可的情况下恶意收集、发送用户的信息,与传统的病毒不同,它们通常是用户自愿使用的软件,并且攻击者事先在程序中植入了一些恶意代码来盗取用户的数据。
移动互联网方面,据2022年6月CNCERT报告\cite{CNCERT2022},其向应用商店、个人网站、广告平台、云平台等传播渠道通报下架移动互联网恶意程序达 2,075 个。
 
针对上述问题,研究人员已经提出了一些软件恶意性检测方法。但随着恶意软件的数量和多样性的不断增加,传统的恶意软件检测方法已然失效。

Windows、安卓等操作系统更加关注个人娱乐和办公,拥有庞大的用户群,Office、WPS等文档处理软件,CAD、MATLAB等科研设计辅助软件短期具备不可替代性。
因此,为保障用户隐私及使用安全性,针对典型系统与软件的通信行为进行把控,分析研究流量的产生原因,如流量用于实现心跳、版本验证、正版验证等必要功能,或传输隐私数据、攻击流量、涉密流量等可疑或存在恶意性的行为。
继续研究更加快速、高效的软件通信行为分析方法,发现其可疑通信行为并分析预警,是本文典型系统与软件可疑行为研究的重点目标(系统软件隶属于软件,下文统称软件)。

% 因此,针对典型系统与软件的可疑行为(包括通信行为和操作行为)继续研究更加快速、高效的(恶意)软件检测方法是十分有必要的。

%Quick Heal 报告\cite{quick2016Report} 显示,2016年第一季度,恶意软件被检测3亿多次。

% Android占据了智能终端设备的大部分市场,市场份额从2009年的1.6\%迅速增长到2016年的86.2\%[2];同时,Android设备上的恶意软件数量也在快速增加,Intel在Mobile Threat Report 2016 [3]中指出,全球大部分地区每小时都有100~1000个恶意软件被检出,恶意软件高发区——中国每小时有超过6000个恶意程序被检测出来。 

 
% Local Variables:
% TeX-master: "../report"
% TeX-engine: xetex
% End: