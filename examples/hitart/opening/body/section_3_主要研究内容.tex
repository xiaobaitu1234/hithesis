% !Mode:: "TeX:UTF-8"

\section{主要研究内容}
本节对CDN服务商识别技术、CDN时延特征测量技术以及CDN放置优化策略相关研究内容进行介绍。


\subsection{基于探测空间约减的CDN服务商识别技术研究}

为完成CDN服务商的识别过程,首先要确定其识别算法的输入与识别算法执行流程。根据奥卡姆剃刀定律(Occam's Razor),本文试图剔除对识别精度影响较小、或识别效率较低的特征因子,从而以最小代价达到较优的识别精度。对当前主流识别算法分析可知,CNAME记录的匹配结果\cite{Huang2008,Adhikari2014,Guo2018},是判定域名是否使用CDN的主要手段之一。同时,在匹配规则基础上引入其他辅助特征,对识别算法的精度提高有一定帮助。Hou\cite{Hou2021}指出,使用CNAME、IP WHOIS以及HTTP响应头可以较好识别CDN服务商。然而,现网环境下,域名与IP的映射关系存在一定误差\cite{Ma2021}。引入IP的相关变量信息,势必会对识别效率和精度产生影响。因此,本文重点考虑HTTP特征来辅助识别。通常,针对使用CDN服务的站点,其经CDN边缘服务器交付给用户的HTTP响应,会在头部信息字典包含一些特征,如Cache、Hit from、Miss from、CDN等标识字段。因此,本文拟采用HTTP的响应头部字典,对CNAME记录匹配识别进行一定扩充,以期使用较少的信息源达到较优的识别精度。


同时,针对特征的探测空间约减,需要对探测对象和探测方法两方面进行处理。针对探测对象,本文主要考虑重点域名列表,或流量较大的域名,因为这些域名有较高概率使用CDN服务。在探测方法上,目前存在两种方式,一种为借助地域分布广泛的探针对域名进行解析,以获取多地域的解析结果
。该方案对探针规模及位置合理性有较高的要求。另一种方式为借助公开区域递归服务器(LDNS open)作为探测媒介\cite{Huang2008},从而获取多地解析结果。该方法借助现有DNS系统,但LDNS open规模有限,无法保证覆盖率。因此,本文拟采用转发DNS服务器(FDNS)作为媒介,从而覆盖更多的区域递归服务器(LDNS),期望使用较少资源和发包量,获取较全面的特征数据,降低对实网环境的污染。

综上所述,本文拟提出一种基于探测空间约减的CDN服务商识别技术,期望使用较少探测资源,以及实网环境低污染情况下,完成CDN服务商的识别。


\subsection{基于分布式代理技术的CDN时延特征测量技术研究}
为完成CDN时延性能测量,需要完成CDN边缘服务器发现。根据CDN就近服务的原则,CDN在DNS解析过程中,会综合考量请求方的ISP、地理位置、网络拥塞情况等因素,返回较优的边缘服务器IP地址。同时,当用户选择OpenDNS和Google Public DNS等公共DNS解析器。其解析结果不可能接近所有用户\cite{fu2018},因此,本研究拟尝试FDNS的转发功能或代理服务器的方式模拟多地解析请求过程。由于DNS拥有缓存机制\cite{Moura2019},在TTL时间内,即时首选边缘服务器存在更新,也无法及时传播到终端用户。因此,还需要通过间隔多次连续解析消除瞬时网络拥塞等情况引起的边缘服务器更新,导致服务器收集不全面的问题。

针对CDN边缘服务器时延测量问题,本文拟采用Socket代理的方式进行测量。胡治国\cite{胡治国2017IP}等人指出,当前网络测量的精准度有待提高,通用操作性系统无法测量高速网络的时延情况。通常,使用代理的测量方式可以保证覆盖面请求的覆盖面问题,理想情况下,使用单点结合分布广泛的代理即可完成探测任务。然而,由于单点不可能临近所有代理,与探测点地理位置过远,或网络距离较远的代理,其探测结果势必存在较大误差。因此,本研究拟采取构建分布式系统结合代理探测方式进行时延测量,并对测量结果进行分析。

\subsection{基于多目标优化的CDN放置策略研究}

CDN放置优化可以视为以降低CDN成本,保证用户服务质量为目的的多目标优化问题。当前多数研究从服务器放置位置,或者是缓存策略方面独立进行研究。服务器放置问题通常是一个NP-Hard问题,多采用设施位置、K-median、最小K中心、K-Cache等模型,进行整数线性规划,并结合贪心算法等启发式或近似算法降低计算量。缓存策略可以由内容分发路径进行考量,选取路径长度较短、拥塞成本低的分发路径。本研究拟从服务器放置和缓存内容分发路径两方面进行综合考量,基于多目标优化技术权衡成本和服务质量。


% Local Variables:
% TeX-master: "../report"
% TeX-engine: xetex
% End: