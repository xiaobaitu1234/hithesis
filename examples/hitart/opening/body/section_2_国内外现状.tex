% !Mode:: "TeX:UTF-8"

\section{国内外研究现状及分析}

目前,各操作系统及其平台上的应用软件或程序主要面临着两个层面上的安全问题,其一是系统安全机制本身的漏洞和缺陷,主要是各系统有可能遭受破坏导致安装了某些恶意应用程序,造成了用户隐私数据泄露等问题;其二是安装了操作系统的设备本身安装的应用程序或软件存在恶意行为,由此造成用户隐私数据泄露等问题。因而国内外学者对于各操作系统及其平台上的应用软件或程序的安全性研究也主要集中在这两个方面。 

在对于各操作系统及其平台应用软件或程序的研究分析上,一方面是通过对系统自身的权限控制,限制非系统软件对系统资源的访问权限,根据应用程序或软件是否超出操作权限范围来判断其安全性,这种方法虽然在一定程度上保护了设备和应用系统的安全,但是对于比较详细且利用Web服务或远程API接口来进行恶意损害用户利益的行为还是无法检测和阻止;另一方面是通过对软件内在的操作流或者数据流进行分析,检测出敏感、易于对用户信息造成泄露的功能函数,进而推断软件行为。软件行为的检测主要有静态检测、动态检测、混合检测(同时使用静态和动态检测技术)。

(1)\textbf{静态检测法}。静态检测一般不需要将程序放在真实的环境中运行,静态检测可以帮助人们理解程序的执行流程。针对不同操作系统及其上的应用软件,学者们研发了不同的检测工具。比如:针对iOS,静态检测是通过iOS逆向工程技术\cite{沙梓社2014IOS},利用IDA、class-dump、otool等一系列逆向工具\cite{zdziarski2012hacking},对应用程序或软件二进制文件进行分析检测;Manuel Egele等\cite{egele2011pios}利用静态检测方法,绘制出程序流程图,检测是否存在泄露用户隐私的用户信息交互。针对Android,有利用抽象解释技术自动静态分析Java和Android应用的工具Julia\cite{payet2012static}、SCanDal\cite{kim2012scandal};利用污点分析技术的AppSealer\cite{2014AppSealer}、FlowDroid\cite{arzt2014flowdroid}等。   
   
静态检测法相对容易,一部分静态检测工具依赖于程序源码。通过静态检测可以对程序进行比较全面的分析,分析范围更广,分析的路径也更多,易于找出隐蔽的执行路径,但可能会产生路径爆炸的问题。由于在静态检测过程中无法确定程序的各种输入以及运行环境,因此其检测的精度不高。

同时,有研究学者认为同一恶意软件家族代码复用导致恶意软件作者或团队编码具有编码相似性\cite{宋文纳2019恶意代码演化与溯源技术研究},因此当同一家族恶意软件载入内存执行时其结构信息和数据也应该具有一定的相似性。鉴于上述相似性,静态检测法通过对恶意软件本身二进制文件、可执行文件或者通过反编译文件提取到的静态特征进行分析,对比恶意软件与正常软件的静态特征的异同来发现恶意软件。通过将 EXE 文件以 PE 文件格式解析,提取样本的文本、全局、头部、导入导出表、节特征,再通过对比分析即可精准地识别出恶意软件。 

Schultz 等\cite{schultz2000data}通过朴素贝叶斯分类算法,将数据挖掘的算法首次应用于恶意软件分析,检测结果比基于特征匹配的传统
静态检测法更准确。Conti G等\cite{2010Automated}首次提出将恶意软件的二进制文件转化为灰度图像。 
Santos 等\cite{2013Opcode}根据操作码的出现频率以及操作码之间的关联性来识别和分类恶意软件。Zhang等\cite{2017IRMD} 通过将可执行文件反编译得到操作码序列转换成图像,输入卷积神经网络CNN来进行识别。这使得图像的概念被引入了恶意软件检测领域。 
      
静态检测法可以准确地捕捉到恶意软件的静态特征,但是由于特征类型单一,混淆或加壳等技术可以让恶意软件逃过检测,使得检测效果下降。

(2)\textbf{动态检测法}。动态检测法主要是通过在虚拟环境中执行恶意软件样本,记录恶意软件的行为特征,如行为日志、系统调用名称、上下文参数、环境变量等\cite{周杨2020基于线程融合特征的}。恶意软件在运行时会做出各类威胁行为,包括修改文件系统 (如写入设备驱动程序、更改系统配置文件) 、修改注册表(如修改注册表键值、更改防火墙设置)、网络行为(如解析域名、发出HTTP请求)等。在独立、安全的沙(Sandbox) 环境中运行PE文件,通过行为分析来判定其是否为恶意软件。动态检测通常与可视化技术相结合,便于分析动态行为轨迹。 
 
郑锐等\cite{郑锐2020一种基于深度学习的恶意软件家族分类模型}使用双向LSTM模型和Cuckoo Sandbox平台收集样本的API调用序列,对6681个恶意软件样本进行分类,取得了99.28\%的准确率。Rieck等\cite{2011Automatic} 通过沙箱采集了恶意软件运行时的行为特征,并基于这些行为特征使用机器学习算法进行识别和分类。Du 等\cite{Du2019A}通过构建API特征数据库,利用对比API序列特征的方法来判断是否属于恶意软件。
 
但动态检测法需要恶意软件完整运行,再进行判断,所以恶意软件的检测时效性较差。

(3)\textbf{混合检测法}。随着恶意软件的数量剧增,类型也逐渐变得多样和复杂,传统的技术显得效率不足。因此研究者逐渐趋向于使用机器学习技术,来应对恶意软件难以预测的变种和日益庞大的数量。

基于机器学习的恶意软件检测法关键在于特征和算法的选择。静态特征和动态特征都可用于机器学习,但无关特征和噪声特征会影响模型的准确性。利用数据挖掘选择数据、特征,再结合机器学习技术完成检测,是现有研究中常见的解决方案。一般分为四个步骤:数据准备、特征提取及特征选择、训练机器学习模型、获取检测结果。目前,大部分安全平台公开了大量恶意软件的数据集,其中包括可以在Windows平台,Linux平台以及移动端等各个环境下运行的恶意样本。而样本类型多种多样,例如木马,蠕虫,后门等,可以根据当前需求进行筛选。获取到数据集后,进行特征提取。特征提取分为静态特征(如PE头特征、二进制内容特征等)以及动态特征(如API调用特征、系统修改特征和网络行为特征等)。然后选择一种机器学习算法,使用提取的特征集合训练模型。最终使用模型进行恶意软件的检测与判定。将用于测试的恶意样本以相同的方式进行特征提取,把测试特征集合送入已经训练完成的模型中,模型会自动判定测试样本的分类,是否属于恶意样本,从而实现恶意软件的检测。 

赵中军等\cite{赵中军2018基于优化}通过优化的K-means算法,快速有效地识别出恶意软件;张莹等\cite{张莹2018基于网络行为特征聚类分析的恶意代码检测技术研究}解决了传统K-means选择初始质心不稳定的问题,提出一种基于PSO-K-means的恶意代码检测方法。杨宏宇等\cite{杨宏宇2017基于改进随机森林算法的}提出的改进随机森林算法在恶意软件检测实验中准确率达到98\%。目前硬件计算能力的大幅提高,使得深度学习的普及成为可能,逐渐步入人们的视野。在恶意软件检测中,CNN、RNN以及两者的结合应用较多,PE文件的二进制字节内容可以直接作为深度神经网络的输入\cite{王蕊2012基于语义的恶意代码行为特征提取及检测方法},也可以提取序列化的特征作为输入\cite{李鹏2012基于空间关系特征的未知恶意代码自动检测技术研究,watson2015malware,毛蔚轩2017一种基于主动学习的恶意代码检测方法}。
 
混合检测法分类准确度高,可以检测多形态的恶意软件,但是时间复杂度高,可扩展性较低。
 
 
% Local Variables:
% TeX-master: "../report"
% TeX-engine: xetex
% End: