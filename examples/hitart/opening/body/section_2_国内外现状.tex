% !Mode:: "TeX:UTF-8"

\section{国内外研究现状及分析}

本节对国内外与CDN服务商识别、CDN节点分布及时延测量、CDN节点分布优化策略相关的研究进行介绍。

\subsection{CDN服务商识别研究现状}
目前,学术界和工业界对CDN厂商的识别可概括为下述三种方式。
 
(1) 关键字匹配。许多研究\cite{Huang2008,Adhikari2014,Guo2018}使用CNAME关键字作为基础匹配机制,对域名的CNAME记录与CDN服务商提供的特征后缀匹配。其次,根据HTTP错误日志\cite{Huang2008,Guo2018}也可以辅助识别,根据响应报文检验,判定其是否包含CDN服务商的关键字。也有利用域名或IP的WHOIS\cite{Adhikari2014}信息、PTR\cite{Chen2019}记录与CDN服务商关键字进行匹配。或是根据域名解析的IP地址,是否归属于CDN服务商公开的IP范围进行识别\cite{Choffnes2017}。特别的,猜测CDN边缘服务器的命名规则以构建主机名,并进行DNS探测以获得更多的节点IP\cite{Hohlfeld2018,Timm2018},也是一种解决方案。

(2) 机器学习等方法。CDN识别与机器学习等方法融合也成为一种趋势。CNCERT\cite{Li2020}的研究表明使用CDN的域名,其在IP和TTL的特征有明显区分度,且样本识别精度可以达到90\%以上。Ma\cite{Ma2021}提出了基于半监督GNN模型的CDN节点发现,将域名与IP二部图的关联进一步细化,降低误判。Chen{Chen2019}提出基于LSTM的FCDR方法,将域名字符、经验信息、地理和时间相关特征作为输入,对域名进行CDN服务判定。Hou\cite{Hou2021}也提出了基于Naive Bayes Tree\cite{Liang2006},结合CNAME、WHOIS、HTTP响应头的统计学习方法\cite{Hou2021}。

(3) 使用互联网开源工具或公开服务接口。如CDN Finder\footnote{https://www.cdnplanet.com/tools/cdnfinder/}、CDN云观测\footnote{https://cdn.chinaz.com/}
或findCDN\footnote{https://github.com/cisagov/findcdn}等工具。其中,CDN Finder使用页面资源列表来识别每个唯一主机名的CDN,通过获取页面资源并检查响应标头、执行DNS解析并检查完整的CNAME链获取待判定特征,根据
其维护的Header-to-CDN和CNAME-to-CDN列表来匹配。若匹配失败,则根据其服务IP所属AS来匹配CDN服务商。findCDN则是通过HTTPS服务器响应头、CNAME记录结合IP WHOIS信息来匹配CDN服务商。 

% \begin{enumerate}[label={(\arabic*)}]

% 	\item 
% 	\item 
% 	\item 
% \end{enumerate}

从CDN识别精度而言,不同研究发现的数量不尽相同。基于匹配机制的识别方法一般精度在85\%以上,对于当前公开的CDN识别服务,如CDN Finder、CDN云观测以及findCDN等,其内在机制也是使用特征库对所选取特征进行匹配操作。集成机器学习或深度学习技术后,精度有进一步提高,如使用域名IP特征和TTL特征的方案可以达到90\%的精度,对于使用更多信息源的深度学习方案\cite{Ma2021},其样本精度可以达到98\%。相应的,高精度的代价是使用更多信息源。

从解析开销而言,Adhikari\cite{Adhikari2014}基于主被动测量,使用规范主机名(CNAME)、IP地址的WHOIS进行所有者识别,从而判定Hulu和Netflix使用的CDN服务商。Huang\cite{Huang2008}基于获取大量Windows Live搜索日志,从主机名中提取CNAME记录,并判定其是否属于特定目标CDN。以此来构建其特征识别库。Ma\cite{Ma2021}的图神经网络方法,更是集成了地理、时间等信息,在获得高精度的同时也极大增加了数据探测成本。 

综上所述,本研究需要选择一种较为高效的识别分类算法,并挑选识别收益较高的特征。同时,仍需要解决数据探测对实网环境的影响,降低特征的解析开销。



\subsection{CDN节点分布及时延测量研究现状}



说明当前主要存在的测量手段,测量内容,以及效果等。
 

研究\cite{Pathan-survey-2007}表明,CDN的业务目标至少包括可扩展性、安全性、可靠性、响应性和性能。由于测量基于DNS,吞吐量较难测量\cite{Huang2008}。



Johnson是较早研究CDN服务质量测量\cite{Johnson-2001-cdn-measure}之一,并指出CDN的价值并非提供最佳的分发点,而是避免给出性能明显不行的。

根据以往的研究\cite{Huang2008},CDN主要存在两方面时延:DNS解析时延,也即CDN内部DNS系统向终端用户提供“最佳”CDN边缘节点地址的时间;以及内容服务器时延,及终端用户和所选CDN服务器之间的往返时间。Krishna等人提出了King\cite{King-2002}方法来测量端到端估计距离,Huang\cite{Huang2008}在对CDN进行测绘时采取的增强king方法便由此改进而来。但由于其存在缓存污染、流量开销较大问题,后续有人提出了一种轻量化测量平台\cite{Zhang-2021-Scale-platform},总之这里要给出一个测量方法。

同时,根据阿里云\footnote{https://help.aliyun.com/document\_detail/140425.html}及腾讯云\footnote{https://cloud.tencent.com/document/product/228/1198}公开的CDN性能衡量指标,解析时延和可用性也是两个重要方面。

在DNS解析过程中,需要对LDNS和ODNS进行约减,减少发包,还要对CNAME进行筛选,减少发包。主要看获取到的IP是否量级相同。




\subsection{CDN节点分布优化策略研究现状}
当前,学术界对CDN的研究多种多样,
有对CDN边缘服务器放置算法的研究\cite{Sahoo2016},同时,更有学者大胆提出将IP与主机名解绑\cite{Fayed-IpUnbind-2021},实现一种更灵活的CDN服务(此处可以参考HTTP3.0弃用TCP,或者是Oracle的自动存储管理(ASM),基本做到与OS独立,直接操作磁盘卷)(基于层级的穿透,基于位置的调整)


Huang\cite{Huang2008}在对CDN进行测绘时,发现CDN分布的两种设计理念,一种是深入ISP的设计,其更接近终端用户,在时延和吞吐量方便有较大优化。但集群较为分散,管理困难。另一种是ISP入户设计,即只在几个关键点建立大型内容分发中心,并使用高速链路互联,与第一种相比,在维护和管理费用方面有所降低,但时延有一定损失。考虑后续分布策略

\subsection{国内外研究现状分析}
由上文可知,研究人员在CDN服务商识别、节点分布及时延测量,节点分布优化策略方面展开了研究,并提出了一些有效的理论和方法,然而仍存在下述问题。

(1) 针对CDN服务商识别,不同研究达到的识别精度与探测开销不尽相同,CDN服务识别在基于匹配规则时有良好表现,尤其是在CNAME特征匹配时,精度较高。在使用多种匹配规则进行联合识别时,识别分类的精度略有提高。在此基础上,引入机器学习和深度学习,同时增加多种信息源(如地理位置、TTL特征、IP特征等)后,其识别精度进一步提高,但已产生边际效应,探测资源再投入对精度的提高收益显著减少。



给出当前识别方面的问题,服务质量测量方面存在的不足。说明我自己要做一个分类器来识别,做一个方面的测量指标


因此,为对CDN服务的性能进行测量和改进,如何对CDN服务进行高效、准确的发现与识别,如何对...........................是什么什么需要解决的问题,也是本文的研究目标。







% Local Variables:
% TeX-master: "../report"
% TeX-engine: xetex
% End: