% !Mode:: "TeX:UTF-8"

\section{研究方案及进度安排,预期达到的目标和已取得的研究成果}
本节对研究方案及进度安排,预期达到的目标和已取得的研究成果进行介绍。
\subsection{研究方案}
为了实现面向时延特征的CDN服务质量测绘及部署优化策略,首先需要获取国内CDN服务商的特征及使用情况,该数据需要通过实现CDN服务商识别系统得到;然后需要根据CDN使用情况,选取测量目标并进行测绘。该步骤需要使用分布式代理测量平台实现,并构建服务质量评估方法给出评价值。最后需要根据CDN放置效果设计CDN放置优化策略,得到一种成本与服务质量均衡的放置方案。本文研究方案的整体结构如图 \ref{fig:研究整体结构图}所示。

 
\begin{figure}[ht]
	\centering
	\begin{tikzpicture}[grow cyclic, text width=2.7cm, align=flush center,
    level 1/.style={level distance=5cm,sibling angle=120},
    level 2/.style={level distance=3cm,sibling angle=90},
		rotate=90
		]
  
    \node{研究整体结构图}
    child { node {这里需要放一张图} 	}
		child { node {这里需要放一张图} 	}
		child { node {这里需要放一张图}  };
  \end{tikzpicture}

	\caption{研究整体结构图}
	\label{fig:研究整体结构图}
\end{figure}


\subsubsection{基于探测空间约减的CDN服务商识别系统}
基于探测空间约减的CDN服务商识别系统包括探测空间约减和CDN识别特征集扩充模块,系统模块内容及关系如图 \ref{fig:基于探测空间约减的CDN服务商识别系统模块关系}所示:


\begin{figure}[ht]
	\centering
	\begin{tikzpicture}[grow cyclic, text width=2.7cm, align=flush center,
    level 1/.style={level distance=5cm,sibling angle=120},
    level 2/.style={level distance=3cm,sibling angle=90},
		rotate=90
		]
  
    \node{基于探测空间约减的CDN服务商识别系统模块关系}
    child { node {这里需要放一张图} 	}
		child { node {这里需要放一张图} 	}
		child { node {这里需要放一张图}  };
  \end{tikzpicture}

	\caption{基于探测空间约减的CDN服务商识别系统模块关系}
	\label{fig:基于探测空间约减的CDN服务商识别系统模块关系}
\end{figure}



(1)\textbf{基于FDNS聚类的探测空间约减方法研究}  

本文研究识别数据获取过程中,发包量主要集中在DNS解析部分,且主要考虑其覆盖面和解析目标选取合理性。针对覆盖面问题,本文考虑使用基于FDNS聚类的探测方法,对于解析目标,本文将选取重点域名排名,具体介绍如下。

1)根据CDN就近解析原则,在模拟多地域探测时(即使用较少探针或单探针解析,获取与多个不同地域探针解析覆盖效果相同的结果集),需要借助模拟解析地区的区域递归DNS服务器(LDNS),并舍弃EDNS扩展选项,因为EDNS可能会携带请求方IP地址,使得CDN根据智能DNS返回重复结果。针对这种使用LDNS作为请求目标服务器的方法,本文称之为基于LDNS聚类的探测方法。Huang\cite{Huang2008}使用基于LDNS聚类的探测方法,利用公开LDNS作为解析器,获取多个区域的解析结果。但由于许多LDNS被配置非公开,即只为其内部用户进行DNS解析,因此覆盖面并不完整\cite{吴金福2014中国大陆}。哈尔滨工业大学陆柯羽研究表明,转发DNS(FDNS)占据开放DNS比重达40\%左右DNS\cite{lukeyu-2021-DNS},远高于递归DNS占比。且转发DNS采用将解析请求转达递归等有解析能力的DNS服务器进行解析,即FDNS必然对应一个或多个递归DNS。因此,本文拟采用FDNS作为解析的请求目标,故称为基于FDNS聚类的探测方法。
 

2)重点域名数据集选取,由于Alexa排名目前处于停止更新状态,本文考虑当前较为出色的域名排行:Traco\cite{Pochat2018}排名以及SecRank\cite{Xie2022}排名。其中,Traco融合了多种国际排名,如Umbrella、Majestic以及Farsight,具有较好的研究价值。但由于其提供域名多为主域名,不适合用来做CNAME检测及HTTP获取。因此,本文选择具有包含多种标签长度域名的SecRank排名作为研究对象。进一步的,针对排名域名选择,本文考虑使用www域名以及SLD域名作为研究对象,因为它们大概率是站点。从减少探测目标的角度来约减DNS发包量,从而降低探测对实网环境的污染。 


(2) \textbf{面向HTTP特征的CDN识别特征集扩充方法研究}

本研究使用域名的HTTP特征进行是否使用CDN服务的判定。并提取使用CDN服务的HTTP特征对应的CNAME记录,并根据CNAME记录的特征进行人工判定,将判定为CDN服务商的CNAME特征串扩充到CDN特征库中。

1) 数据集构建

通常,分类器需要提供一定规模的正例和反例进行训练,基于DNS解析得到的域名CNAME、以及HTTP Head命令解析得到的域名HTTP信息,本文给出如图\ref{fig:解析数据预分类}所示的分类过程,保守得将数据划分为使用CDN服务、未使用CDN服务、待分类数据、无效数据四类。该分类依据算法需要,过滤掉信息不完整内容,即将不含HTTP信息或不含CNAME信息得数据作为无效数据。随后,将明确匹配成功的数据作为使用CDN服务数据,将HTTP信息不含跳转特征,并且多地解析结果相同的数据,视为未使用CDN服务数据。此后,剩余数据为待分类数据,CDN特征库将根据待分类数据的识别结果进行选择性扩充。
 
\begin{figure}[ht]
	\centering
	\includegraphics{解析数据预分类.pdf}
	\caption{解析数据预分类}
	\label{fig:解析数据预分类}
\end{figure}
\FloatBarrier

2) HTTP分类器构建
该部分利用前述使用CDN服务和未使用CDN服务的HTTP信息进行特征提取,主要考虑使用TF-IDF方法对关键字进行分类效益排序,提取主要的特征作为编码依据,将HTTP进行特征编码。后续也将尝试PCA、LDA等主成分分析方法进行特征提取,需要根据实验结果进行探讨。针对分类器构建问题,当前需求明确为二分类问题,可以考虑使用支持向量机(SVM)来进行解决。同时,由于当前数据的标签较为保守,待分类数据中可能包含较多未明确标签的数据,因此,本研究拟采用TSVM\cite{Joachims1999}(一种经典半监督 SVM)方法来对待分类数据进行判定。


4) CDN特征库扩充

作为辅助识别特征,HTTP分类得主要目的为增强主特征的识别精度,即学习新的CNAME特征串,扩充到CDN特征库中。针对待分类数据进行分类判定后,将主要针对分类器判定为使用CDN服务的域名数据,针对其CNAME记录进行判定。该阶段中,可以使用WHOIS信息或搜索引擎对当前CNAME的归属进行判定。若判定其隶属于某CDN服务商,则将其扩充到CDN特征库中。
 
\subsubsection{基于分布式代理技术的CDN时延特征测量技术}
本研究拟针对特定CDN服务商进行探讨,使用大流量域名基于Socket代理技术(后简称代理)从多地进行持续探测,完成边缘服务器发现工作。发现过程中,使用链路往返时间探测技术,得到往返时延属性,从而给出CDN缓存服务器的时延特征。

(1) 探测基本流程
针对特定CDN服务商,根据CDN识别技术得到的CNAME特征串寻找对应的大流量域名作为探测信息构造依据。探测时,利用代理首先发送DNS解析报文,得到CDN服务商提供的IP集合。随后,立刻向IP集合中每一个IP连续多次发送RTT探测报文,计算得出代理与CDN边缘服务器的时延特征。


(2) 分布式探测系统构建 
针对探测点与代理的长距离误差情况,本研究拟构建分布式探测系统,将代理根据AS、链路时延等信息指派至较优探测点。具体执行流程如图 \ref{fig:分布式代理系统原型}中探测流程所示,其中,系统节点位置拟通过代理服务器进行Canopy结合K-Means聚类得出节点租用数量和位置。随后,系统将根据分布式节点与代理的距离关系(通过时延情况、AS情况等因素计算得出)得出代理的调度顺序,即当前代理的指派顺序。根据大流量域名构建的探测任务,将通过任务分发,广播到所有系统节点,并转发至指派代理节点中。代理节点负责对探测任务具体执行过程,主要分为域名的解析过程以及IP时延测量过程。

(3) 测量结果分析
对测量得到的时延特征,本文拟从三个角度进行分析,如图 \ref{fig:测量结果分析维度}所示。其中,可用性主要体现在长期监测丢包率情况,反映CDN提供无间断服务的能力。使用稳定性从用户出发,给出CDN服务质量波动情况。结构稳定性从空间结构角度,检验CDN各地区服务质量差异情况。

在系统鲁棒性方面,本研究主要从健康度监测、代理存活以及代理链路可达三方面进行考量。如图 \ref{fig:分布式代理系统原型}中系统鲁棒性所示, 
健康度监测面向分布式系统的所有探测节点,保证对节点异常情况的快速发现,如服务器宕机、硬盘容量、内存占用情况等。
代理链路可达性角度,构建自适应调度算法,主要根据代理的指派情况,以及探测汇总结果进行分析,针对未探测区域即时调度,保证数据的完整性和时效性。
代理存活解决同样保证数据的全面性问题,主要通过增加代理冗余,即同区域选取多个代理,保证区域存在探测代理。

\begin{figure}[ht]
	\centering
	\begin{tikzpicture}[grow cyclic, text width=2.7cm, align=flush center,
    level 1/.style={level distance=3cm,sibling angle=120},
    level 2/.style={level distance=3cm,sibling angle=90},
		rotate=90
		]
  
    \node{时延分析}
    child { node {可用性} 	}
		child { node {使用稳定性} 	}
		child { node {结构稳定性}  };
  \end{tikzpicture}

	\caption{测量结果分析维度}
	\label{fig:测量结果分析维度}
\end{figure}



\begin{figure}[ht]
	\centering
	\begin{tikzpicture}[grow cyclic, text width=2.7cm, align=flush center,
    level 1/.style={level distance=4cm,sibling angle=90},
    level 2/.style={level distance=2cm,sibling angle=90},
		rotate=270
		]
  
    \node{分布式代理系统}
    child { node {探测流程}  
			child { node {代理指派}}
			child { node {任务分发}}
			child { node {探测解析}}
		}
		child { node {系统鲁棒性}  
			child { node {链路可达}}
			child { node {代理存活}}
			child { node {健康度监测}}
		};
  \end{tikzpicture}

	\caption{分布式代理系统原型}
	\label{fig:分布式代理系统原型}
\end{figure}
\FloatBarrier

\subsubsection{基于多目标优化博弈的CDN放置策略研究}
本文设计基于多目标优化的CDN放置策略,如图 \ref{fig:基于多目标优化的CDN放置策略研究方案}所示。首先,采用
\begin{figure}[ht]
  \centering
  \includegraphics{基于多目标优化的CDN放置策略研究方案.pdf}
 
  \caption{基于多目标优化的CDN放置策略研究方案}
  \label{fig:基于多目标优化的CDN放置策略研究方案}
\end{figure}

本研究拟使用多目标优化技术,从服务器放置成本,缓存路径选择两方面进行考量。尝试使用包括但不限于以下方式进行研究。

(1) 整数线性规划结合博弈论方法, 服务器放置,以及路径选择,通常可转变为图模型,并使用整数线性规划进行求解。从单一方面来看,针对服务器放置进行规划,可以得到较低成本部署方案,从路径选择规划,可以给出部署方案下的较优服务质量。为得到成本与服务质量均衡的解决方案,需要使用博弈思想对两者进行权衡。因此,可以尝试使用整数线性规划集合博弈论的方法对CDN放置策略进行研究。

(2) 基于强化学习方法, 对服务器部署和路径选择组合学习,通过优化成本和服务质量目标对CDN放置策略进行调优。


对于优化结果验证方面,拟采用仿真环境验证,通常,网络方面实验可使用大规模网络实验环境PlanetLab\footnote{https://planetlab.cs.princeton.edu/}进行仿真。作为备选方案,Stamos\cite{stamos2010cdnsim}提出的CDN仿真软件CDNsim\footnote{https://sourceforge.net/projects/cdnsim/}或后续可能发现的其他仿真环境,均在本研究的考虑范围之内。

\subsection{预期达到的目标和取得的研究成果}
本节对预期达到的目标和取得的研究成果进行介绍。
\subsubsection{预期达到的目标}

通过前期方案准备,本次拟达到的目标如下:

(1) 设计实现基于探测空间约减的CDN服务商识别系统,实现CDN识别数据集的扩充功能。

(2) 设计基于分布式代理技术的CDN时延特征测量平台,并根据测量结果进行时间、空间及可用性角度分析。

(3) 设计一种CDN放置优化策略,提供CDN放置成本与CDN服务质量均衡的部署策略。

\subsubsection{已取得的研究成果}

	\noindent\textbf{(一)已发表论文}
	\begin{enumerate}
	\item Li C, Cheng Y, \textbf{Men H}, et al. Performance Analysis of Root Anycast Nodes Based on Active Measurement[J]. Electronics, 2022, 11(8): 1194.(EI检索)
\end{enumerate}
\noindent\textbf{(二)申请及已获得的专利}
\begin{enumerate}
\item 张兆心,李超,程亚楠,郭长勇,杜跃进,\textbf{门浩}. 一种获取网络数据时网络阻塞造成的噪声数据消除方法: 中国,CN202110121032.7[P]. 2022-04-15. (已授权)
\item 张兆心,李超,柴婷婷,程亚楠,陆柯羽,郭长勇,杜跃进,\textbf{门浩}. 基于网络服务商国别标注的域名国家可控性评估方法:中国, CN202110258091.9[P]. 2021-06-01.(已受理)
\item 张兆心,李超,程亚楠,陆柯羽,\textbf{门浩}. 一种基于多源信息定位域名根镜像节点地理位置的方法: 中国,CN202110856090.4[P]. 2021-11-02. (已受理)
\item 张兆心,\textbf{门浩},程亚楠,梁浩宇,郭长勇,李超,赵东. 一种多设备网页中内嵌广告获取以及恶意性识别的方法: 中国,202210606165.8[P]. 2022-08-09. (已受理)
\end{enumerate}

 
\subsection{进度安排}
论文进度安排如表 \ref{table:jinduanpai}所示:


\begin{table}[htbp]
	\centering
	\caption{进度安排}\label{table:jinduanpai}
	\vspace{0.5em}\wuhao
	\begin{tabular}{ccc}
		\toprule
		起始日期        & 截止日期     & 进度安排    \\
		\midrule
		2022年05月           & 2022年07月           & 阅读文献、调研、数据验证      \\
		2022年07月          & 2022年08月  & 开题报告撰写,算法设计  \\
		2022年08月         & 2022年11月   & CDN识别及分布式探测平台构建  \\
		2022年11月         & 2022年2月   & CDN放置优化策略实现  \\
		2022年2月           & 2023年03月  & 算法验证与优化 \\
		2023年03月          & 2023年05月 &      撰写毕业论文     \\


		\bottomrule
	\end{tabular}
\end{table}


\FloatBarrier




% Local Variables:
% TeX-master: "../report"
% TeX-engine: xetex
% End: