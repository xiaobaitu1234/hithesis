% !Mode:: "TeX:UTF-8"
\section{主要研究内容}

为保障用户隐私及安全性,针对软件通信行为进行把控,研究流量用途。
本文首先研究基于动态分布式的设备多通信协议实时流量采集方法,提供演练与实网环境下的流量采集支持;
其次,研究基于多维特征协同计算的软件正常行为特征建模,构建软件正常行为特征库,作为可疑行为匹配机制检测依据;
最后研究基于相异行为特征的软件可疑行为检测方法,面向实网环境采集的灰色流量,实现正常行为特征库扩充及可疑流量行为分析。
研究内容组织结构如图 \ref{fig:研究内容组织结构图}所示。

% 为针对典型系统与软件可疑通信行为分析,本文主要分为通信流量采集、正常行为特征库构建以及系统与软件流量可疑行为检测分析三部分。
% 首先,设计实现设备多通信协议实时流量采集系统,提供演练与实网环境下的流量采集支持;
% 其次,根据演练环境采集的高正常流量,基于多为特征协同构建软件正常行为特征库;
% 最后,面向实网环境采集的灰色流量,基于正常行为特征库匹配与可疑分类方法分离正常流量与可疑流量,并提供可疑流量的行为分析。

 
\FloatBarrier

\begin{figure}[ht]
  \centering
  \includegraphics[width = 1\textwidth]{主要研究内容.pdf}
  \caption{研究内容组织结构图}
  \label{fig:研究内容组织结构图}
\end{figure} 

\FloatBarrier
 
\subsection{基于动态分布式的设备多通信协议实时流量采集方法}

为了对通信流量进行实时、准确、完整、高效的采集,
本文研究基于动态分布式的设备多通信协议实时流量采集方法,
支持动态扩展目标协议,动态增加探测设备的分布式流量采集,
实现对多设备多通信协议采集任务统一采集调度,精准采集实时流量目的。

\subsection{基于多维特征协同计算的软件正常行为特征建模}

为了对软件正常行为特征进行全面、透明和可靠地画像,
本文研究基于多维特征协同计算的软件正常行为特征建模方法,构建软件正常行为特征库。
在可控正常流量环境下,利用时间窗口和空间关联的正常性验证机制,基于多特征维度协议,研究流量的实际效用,
实现对软件多协议流量目标定位和协议内容覆盖。

\subsection{基于相异行为特征的软件可疑行为检测方法}

为了对实网环境中软件产生的可疑行为进行准确、及时发现与分析,
本文研究基于相异行为特征的软件可疑行为检测方法,
面向实网环境下的灰色流量,基于正常行为特征匹配及可疑分类两种机制进行检测,分析灰色流量的数据流向和内容可疑类别。
并实现对软件正常行为特征库补充,对软件可疑行为进行分析及预警。
 
% Local Variables:
% TeX-master: "../report"
% TeX-engine: xetex
% End: