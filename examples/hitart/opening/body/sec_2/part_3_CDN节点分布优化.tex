% !Mode:: "TeX:UTF-8"
 
\subsection{CDN放置优化策略研究现状}

CDN放置策略可以从节点放置,节点内容缓存\cite{Sahoo2016}两大方面进行研究。前者是确定新位置来放置边缘服务器,内容缓存方面,则是考虑一组内容对象与已部署边缘服务器的关系,称为缓存/复制策略,期望达到服务质量约束(QoS,Quality of Service)下,使成本最小\cite{salahuddin2017survey,Pathan-survey-2007,Tang-2018}。
 
缓存方面,
清华大学的葛志诚\cite{葛志诚2018一种移动内容分发网络的分层协同缓存机制}等人由移动CDN缓存入手,针对资源缓存策略以及资源请求路径,研究多层缓存协调配合问题研究,基于贪心思想提出了一种启发式分层协作缓存策略。
Fu\cite{fu2018}则针对任播CDN入手,通过使用SDN结合NFV(网络功能虚拟化)控制细粒度流量重定向,使依赖BGP路由的任播CDN可根据实际情况重定向,通过控制路径来提高CDN的服务质量。
Liu\cite{liu2018}重点关注移动CDN基站的协作问题,使用随机优化模型对网络稳定约束下的长时间平均传输成本进行优化。有效决定了内容放置和请求重定向问题,在拥塞避免和传输成本方面拥有高性能。此外,也有少数研究从结构方面入手,进行CDN服务的优化,如
Fayed使用可编程套接字sk\_lookup,使IP与主机资源解绑\cite{Fayed-IpUnbind-2021},实现一种更灵活的CDN服务,节约IP资源。
Li\cite{Li-2019-AI-attention-CDN}通过构建一种通用AI注意力网络对CDN服务的性能进行预测。
   
节点放置方面, 
Huang\cite{Huang2008}在对CDN进行测绘时,发现CDN分布的两种设计理念,一种是深入ISP的设计,其更接近终端用户,在时延和吞吐量方便有较大优化。但集群较为分散,管理困难。另一种是ISP入户设计,即只在几个关键点建立大型内容分发中心,并使用高速链路互联,与第一种相比,在维护和管理费用方面有所降低,但时延性能有一定损失。
Zhu\cite{zhu-2021}指出CDN设置逐渐边缘化,靠近终端用户。针对优酷部署的路由宝入户等现象进行分析,发现其智能路由器可配置多达1TB的存储空间,起到辅助CDN进行内容分发的功能。并研究一种边缘网络视频内容分发的新架构,使智能路由器拥有轻量级的内容分发功能,达到CDN下沉的目的。
Sahoo\cite{Sahoo2016}指出,传统CDN的边缘服务器放置是NP-Hard问题。其问题定义如下:给定一组服务器侯选位置,一组用户位置,边缘服务器放置涉及从候选集中找到边缘服务器的最佳数量和位置,使得每个最终用户分配到其中一个服务器,且CDN服务商的成本最小化。同理,基于云的CDN使用基于租用的虚拟化资源建立边缘服务器,需要结合云提供商考虑,基于NFV的CDN中,服务器放置需要VNF(虚拟化网络功能)实现的CDN节点,称为VNF放置问题。基于虚拟化程度分类,传统CDN放置依赖于离线算法实现,基于云和VNF的放置则支持在线算法。
Benkacem\cite{benkacem2018optimal}等人介绍了一种CDN as a Service平台,支持跨云平台对基于VNF的CDN进行动态部署和管理。相应的,使用两个整数线性规划,分别用于降低成本和提高服务质量。并使用博弈论提供权衡方案。
Tang\cite{Tang-2018}等人在部署优化方面,额外考虑了动态用户流量,并将流量分发成本降维至整数线性规划(ILP)问题。随后,通过贪心算法来解决ILP问题的可扩展形式。


% Local Variables:
% TeX-master: "../section_2_研究现状"
% TeX-engine: xetex
% End: